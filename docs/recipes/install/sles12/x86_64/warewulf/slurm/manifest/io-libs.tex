\newcolumntype{C}[1]{>{\centering}p{#1}} 
\newcolumntype{L}[1]{>{\raggedleft}p{#1}} 
\small
\begin{tabularx}{\textwidth}{L{\firstColWidth{}}|C{\secondColWidth{}}|X}
\toprule
{\bf RPM Package Name} & {\bf Version} & {\bf Info/URL}  \\ 
\midrule

% <-- begin entry for adios
adios-gnu-impi-ohpc & 
\multirow{8}{*}{1.11.0} & 
\multirow{8}{\linewidth}{The Adaptable IO System (ADIOS). \newline {\color{logoblue} \url{http://www.olcf.ornl.gov/center-projects/adios}}} \\ 
adios-gnu-mpich-ohpc & 
& \\ 
adios-gnu-mvapich2-ohpc & 
& \\ 
adios-gnu-openmpi-ohpc & 
& \\ 
adios-intel-impi-ohpc & 
& \\ 
adios-intel-mpich-ohpc & 
& \\ 
adios-intel-mvapich2-ohpc & 
& \\ 
adios-intel-openmpi-ohpc & 
& \\ 
\hline
% <-- end entry for adios

% <-- begin entry for hdf5
hdf5-gnu-ohpc & 
\multirow{2}{*}{1.8.17} & 
\multirow{2}{\linewidth}{A general purpose library and file format for storing scientific data.  {\color{logoblue} \url{http://www.hdfgroup.org/HDF5}}} \\ 
hdf5-intel-ohpc & 
& \\ 
\hline
% <-- end entry for hdf5

% <-- begin entry for netcdf-cxx
netcdf-cxx-gnu-impi-ohpc & 
\multirow{8}{*}{4.3.0} & 
\multirow{8}{\linewidth}{C++ Libraries for the Unidata network Common Data Form. \newline {\color{logoblue} \url{http://www.unidata.ucar.edu/software/netcdf}}} \\ 
netcdf-cxx-gnu-mpich-ohpc & 
& \\ 
netcdf-cxx-gnu-mvapich2-ohpc & 
& \\ 
netcdf-cxx-gnu-openmpi-ohpc & 
& \\ 
netcdf-cxx-intel-impi-ohpc & 
& \\ 
netcdf-cxx-intel-mpich-ohpc & 
& \\ 
netcdf-cxx-intel-mvapich2-ohpc & 
& \\ 
netcdf-cxx-intel-openmpi-ohpc & 
& \\ 
\hline
% <-- end entry for netcdf-cxx

% <-- begin entry for netcdf-fortran
netcdf-fortran-gnu-impi-ohpc & 
\multirow{8}{*}{4.4.4} & 
\multirow{8}{\linewidth}{Fortran Libraries for the Unidata network Common Data Form. \newline {\color{logoblue} \url{http://www.unidata.ucar.edu/software/netcdf}}} \\ 
netcdf-fortran-gnu-mpich-ohpc & 
& \\ 
netcdf-fortran-gnu-mvapich2-ohpc & 
& \\ 
netcdf-fortran-gnu-openmpi-ohpc & 
& \\ 
netcdf-fortran-intel-impi-ohpc & 
& \\ 
netcdf-fortran-intel-mpich-ohpc & 
& \\ 
netcdf-fortran-intel-mvapich2-ohpc & 
& \\ 
netcdf-fortran-intel-openmpi-ohpc & 
& \\ 
\hline
% <-- end entry for netcdf-fortran

% <-- begin entry for netcdf
netcdf-gnu-impi-ohpc & 
\multirow{8}{*}{4.4.1.1} & 
\multirow{8}{\linewidth}{C Libraries for the Unidata network Common Data Form. \newline {\color{logoblue} \url{http://www.unidata.ucar.edu/software/netcdf}}} \\ 
netcdf-gnu-mpich-ohpc & 
& \\ 
netcdf-gnu-mvapich2-ohpc & 
& \\ 
netcdf-gnu-openmpi-ohpc & 
& \\ 
netcdf-intel-impi-ohpc & 
& \\ 
netcdf-intel-mpich-ohpc & 
& \\ 
netcdf-intel-mvapich2-ohpc & 
& \\ 
netcdf-intel-openmpi-ohpc & 
& \\ 
\hline
% <-- end entry for netcdf

% <-- begin entry for phdf5
phdf5-gnu-impi-ohpc & 
\multirow{8}{*}{1.8.17} & 
\multirow{8}{\linewidth}{A general purpose library and file format for storing scientific data. \newline {\color{logoblue} \url{http://www.hdfgroup.org/HDF5}}} \\ 
phdf5-gnu-mpich-ohpc & 
& \\ 
phdf5-gnu-mvapich2-ohpc & 
& \\ 
phdf5-gnu-openmpi-ohpc & 
& \\ 
phdf5-intel-impi-ohpc & 
& \\ 
phdf5-intel-mpich-ohpc & 
& \\ 
phdf5-intel-mvapich2-ohpc & 
& \\ 
phdf5-intel-openmpi-ohpc & 
& \\ 
\hline
% <-- end entry for phdf5

\bottomrule
\end{tabularx}
