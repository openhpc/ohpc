\newcolumntype{C}[1]{>{\centering}p{#1}} 
\newcolumntype{L}[1]{>{\raggedleft}p{#1}} 
\small
\begin{tabularx}{\textwidth}{L{\firstColWidth{}}|C{\secondColWidth{}}|X}
\toprule
{\bf RPM Package Name} & {\bf Version} & {\bf Info/URL}  \\ 
\midrule

% <-- begin entry for R_base-fsp
\multirow{2}{*}{R\_base-fsp} & 
\multirow{2}{*}{3.2.0} & 
R is a language and environment for statistical computing and graphics (S-Plus like).  { \color{blue} http://www.r-project.org} 
\\ \hline 
% <-- end entry for %name_base

% <-- begin entry for autoconf-fsp
\multirow{2}{*}{autoconf-fsp} & 
\multirow{2}{*}{2.69} & 
A GNU tool for automatically configuring source code. \newline { \color{blue} http://www.gnu.org/software/autoconf} 
\\ \hline 
% <-- end entry for %name_base

% <-- begin entry for automake-fsp
\multirow{2}{*}{automake-fsp} & 
\multirow{2}{*}{1.15} & 
A GNU tool for automatically creating Makefiles. \newline { \color{blue} http://www.gnu.org/software/automake} 
\\ \hline 
% <-- end entry for %name_base

% <-- begin entry for intel-inspector-fsp
\multirow{2}{*}{intel-inspector-fsp} & 
\multirow{2}{*}{16.0.1.407184} & 
Intel(R) Inspector XE. \newline { \color{blue} http://www.intel.com/software/products} 
\\ \hline 
% <-- end entry for %name_base

% <-- begin entry for libtool-fsp
\multirow{2}{*}{libtool-fsp} & 
\multirow{2}{*}{2.4.6} & 
The GNU Portable Library Tool. \newline { \color{blue} http://www.gnu.org/software/libtool} 
\\ \hline 
% <-- end entry for %name_base

% <-- begin entry for python-numpy
python-numpy-gnu-fsp & 
\multirow{2}{*}{1.9.2} & 
\multirow{2}{\linewidth}{NumPy array processing for numbers, strings, records and objects.  {\color{blue} http://sourceforge.net/projects/numpy}} \\ 
python-numpy-intel-fsp & 
& \\ 
\hline
% <-- end entry for python-numpy

% <-- begin entry for python-scipy
python-scipy-gnu-impi-fsp & 
\multirow{3}{*}{0.15.1} & 
\multirow{3}{\linewidth}{Scientific Tools for Python. \newline {\color{blue} http://www.scipy.org}} \\ 
python-scipy-gnu-mvapich2-fsp & 
& \\ 
python-scipy-gnu-openmpi-fsp & 
& \\ 
\hline
% <-- end entry for python-scipy

% <-- begin entry for valgrind-fsp
\multirow{2}{*}{valgrind-fsp} & 
\multirow{2}{*}{3.10.1} & 
Valgrind Memory Debugger. \newline { \color{blue} http://www.valgrind.org} 
\\ \hline 
% <-- end entry for %name_base

\bottomrule
\end{tabularx}
