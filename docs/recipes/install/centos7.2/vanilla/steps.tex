\documentclass[letterpaper]{article}
\usepackage{../../common/ohpc-doc}
\setcounter{secnumdepth}{5}
\setcounter{tocdepth}{5}

% Include git variables
\input{vc.tex}

% Define Base OS and other local macros
\newcommand{\baseOS}{CentOS7.2}
\newcommand{\OSRepo}{CentOS\_7.2}
\newcommand{\baseos}{centos7.2}

\newcommand{\install}{yum -y install}
\newcommand{\chrootinstall}{yum -y --installroot=\$CHROOT install}
\newcommand{\groupinstall}{yum -y groupinstall}
\newcommand{\groupchrootinstall}{yum -y --installroot=\$CHROOT groupinstall}

% boolean for os-specific formatting
\toggletrue{isCentOS}

\begin{document}
\graphicspath{{../../common/figures/}}
\thispagestyle{empty}

% Title Page

\lhead{ \small {\color{logodarkgrey}\fontfamily{phv}\selectfont { Install Guide} - {\baseOS{} Version} (v\OHPCVersion{}) } \vspace*{0pt} }

{\hspace*{4in} \includegraphics[width=1.7in]{ohpc_logo_blue.pdf}}

\vspace*{2cm}
\noindent {\LARGE \color{logodarkgrey} \fontfamily{phv}\selectfont OpenHPC (v\OHPCVersion{})} \vspace*{0.1cm} \\
\noindent {\LARGE \color{logodarkgrey} \fontfamily{phv}\selectfont Cluster Building Recipes} \\ 

{\color{logoblue}\noindent\rule{6.15in}{1.2pt}} \\ \vspace{0.2cm}

\noindent {\Large \color{logodarkgrey} \fontfamily{phv}\selectfont \baseOS{} Base OS} \vspace{0.2cm}

\noindent {\large \color{logodarkgrey} \fontfamily{phv}\selectfont {\em Base Linux* Edition }}

\vspace*{4in}


%\noindent{\normalsize \color{black} Intel Cluster Makers} \vspace*{0.1cm} \\
%{\normalsize \color{logodarkgrey} Copyright~{\small\copyright}~2014-2015 Intel Corporation} \vspace*{0.1cm} \\ 
\noindent{\small \color{black} Document Last Update: \VCDateISO} \vspace*{0.1cm} \\ 
{\small \color{black} Document Revision: \VCRevision} \\ \vspace*{0.1cm}

% Disclaimer  ----------------------------------------------------
\newpage

\vspace*{3.0cm}
\noindent {\Large \color{logoblue} \fontfamily{phv}\selectfont Legal Notice} \\ 

\vspace*{0.5cm}

\noindent Copyright {\small\copyright} 2016-2018, OpenHPC, a Linux Foundation
Collaborative Project. All rights reserved. \\

\vspace*{0.1cm}

\noindent \begin{tabular}{cp{10cm}}
\raisebox{-.75\height}{\includegraphics[width=0.22\textwidth]{cc_by}} &
This documentation is licensed under the Creative Commons Attribution 4.0 International
License. To view a copy of this license, visit
\href{http://creativecommons.org/licenses/by/4.0}{\color{blue}{http://creativecommons.org/licenses/by/4.0}}. \\
\end{tabular}


\vspace*{1.5cm}

{\footnotesize

\noindent Intel, the Intel logo, and other Intel marks are trademarks of Intel
Corporation in the U.S. and/or other countries. \\
\iftoggleverb{ispbs}
\noindent Altair, the Altair logo, PBS Professional, and other Altair marks are
trademarks of Altair Engineering, Inc. in the U.S. and/or other countries. \\
\fi
\noindent *Other names and brands may be claimed as the property of others. \\



}
 

\newpage
\tableofcontents
\newpage

% Introduction  --------------------------------------------------

\section{Introduction} \label{sec:introduction}
% begin_ohpc_run
% ohpc_validation_comment -----------------------------------------------------------------------------------------
% ohpc_validation_comment  Example Installation Script Template
% ohpc_validation_comment
% ohpc_validation_comment  This convenience script encapsulates command-line instructions highlighted in
% ohpc_validation_comment  an OpenHPC Install Guide that can be used as a starting point to perform a local
% ohpc_validation_comment  cluster install beginning with bare-metal. Necessary inputs that describe local
% ohpc_validation_comment  hardware characteristics, desired network settings, and other customizations
% ohpc_validation_comment  are controlled via a companion input file that is used to initialize variables
% ohpc_validation_comment  within this script.
% ohpc_validation_comment
% ohpc_validation_comment  Please see the OpenHPC Install Guide(s) for more information regarding the
% ohpc_validation_comment  procedure. Note that the section numbering included in this script refers to
% ohpc_validation_comment  corresponding sections from the companion install guide.
% ohpc_validation_comment -----------------------------------------------------------------------------------------
% ohpc_validation_newline

% ohpc_command inputFile=${OHPC_INPUT_LOCAL:-/opt/ohpc/pub/doc/recipes/BOSSHORT/input.local}
% ohpc_validation_newline
% ohpc_command if [ ! -e ${inputFile} ];then
% ohpc_command    echo "Error: Unable to access local input file -> ${inputFile}"
% ohpc_command    exit 1
% ohpc_command else
% ohpc_command    . ${inputFile} || { echo "Error sourcing ${inputFile}"; exit 1; }
% ohpc_command fi

% ohpc_validation_newline
% ohpc_validation_comment ---------------------------- Begin OpenHPC Recipe ---------------------------------------
% ohpc_validation_comment Commands below are extracted from an OpenHPC install guide recipe and are intended for
% ohpc_validation_comment execution on the master SMS host.
% ohpc_validation_comment -----------------------------------------------------------------------------------------

% end_ohpc_run

This guide presents a simple cluster installation procedure using components
from the \OHPC{} software stack. \OHPC{} represents an aggregation of a number
of common ingredients required to deploy and manage an HPC Linux* cluster
including provisioning tools, resource management, I/O clients, development
tools, and a variety of scientific libraries. These packages have been
pre-built with HPC integration in mind while conforming to common \Linux{}
distribution standards.
The documentation herein is intended to
be reasonably generic, but uses the underlying motivation of a small, 4-node
\iftoggleverb{isxCATstateful} stateful \else stateless \fi
cluster installation to define a step-by-step process. Several
optional customizations are included and the intent is that these collective
instructions can be modified as needed for local site customizations.
 \\

\noindent {\bf Base Linux Edition}: this edition of the guide highlights
installation without the use of a companion configuration management system and
directly uses distro-provided package management tools for component
selection. The steps that follow also highlight specific changes to system
configuration files that are required as part of the cluster install
process.
%Other editions of this guide provide similar install steps when using
%specific configuration management systems that can simplify the installation
%and configuration process.

\subsection{Target Audience}

This guide is targeted at experienced \Linux{} system administrators for HPC
environments. Knowledge of software package management, system networking, and
PXE booting is assumed. Command-line input examples are highlighted throughout
this guide via the following syntax:

\begin{lstlisting}[language=bash,literate={-}{-}1,keywords={},upquote=true]
[sms](*\#*) echo "OpenHPC hello world"
\end{lstlisting}

Unless specified otherwise, the examples presented are executed with
elevated (root) privileges. The examples also presume use of the BASH login
shell, though the equivalent commands in other shells can be substituted.
In addition to specific command-line instructions called out in this guide, an
alternate convention is used to highlight potentially useful tips or optional
configuration options. These tips are highlighted via the following format:

\begin{center}
\begin{tcolorbox}[]
\small  Life is a tale told by an idiot, full of sound and fury signifying nothing. --Willy Shakes
\end{tcolorbox}
\end{center}


%\noindent {\bf Requirements/Assumptions}: 
\subsection{Requirements/Assumptions}
This installation recipe assumes the availability of a single head node {\em
 master}, and four {\em compute} nodes. The {\em master} node serves as the
overall system management server (SMS) and is provisioned with \baseOS{} and is
subsequently configured to provision the remaining {\em compute} nodes with
\provisioner{} in a 
\iftoggleverb{isxCATstateful} stateful \else stateless \fi 
configuration. The terms {\em master} and SMS are
used interchangeably in this guide. For power management, we assume that
the compute node baseboard management controllers (BMCs) are available via IPMI
from the chosen master host. For file systems, we assume that the chosen master
server will host an \NFS{} file system that is made available to the compute
nodes.
\iftoggleverb{isx86}
Installation information is also discussed to optionally mount a
parallel file system and in this case, the parallel file system is assumed to
exist previously.
\fi

\begin{figure}[hbt]
\center
\iftoggleverb{isx86}
\includegraphics[width=0.85\linewidth]{ohpc-arch-small3.pdf}
\fi
\iftoggleverb{isaarch}
\includegraphics[width=0.85\linewidth]{ohpc-arch-small-eth.pdf}
\fi
\vspace*{-0.2cm}
\caption{Overview of physical cluster architecture.} \label{fig:physical_arch}
\end{figure}
\mbox{}

\vspace*{0.5cm}

An outline of the physical architecture discussed is shown in
Figure~\ref{fig:physical_arch} and highlights the high-level networking
configuration. The {\em master} host requires at least two Ethernet interfaces
with {\em eth0} connected to the local data center network and {\em eth1} used
to provision and manage the cluster backend (note that these interface names
are examples and may be different depending on local settings and OS
conventions). Two logical IP interfaces are expected to each compute node: the
first is the standard Ethernet interface that will be used for provisioning and
resource management. The second is used to connect to each host's BMC and is
used for power management and remote console access. Physical connectivity for
these two logical IP networks is often accommodated via separate cabling and
switching infrastructure; however, an alternate configuration can also be
accommodated via the use of a shared NIC, which runs a packet filter to divert
management packets between the host and BMC.

\iftoggleverb{isx86}
In addition to the IP networking, there is an optional high-speed network
(\InfiniBand{} or \OmniPath{} in this recipe) that is also connected to each of the
hosts. This high speed network is used for application message passing and
optionally for parallel file system connectivity as well (e.g. to
existing \Lustre{} or BeeGFS storage targets).
\fi

% -*- mode: latex; fill-column: 120; -*-

\subsection{Inputs} \label{sec:inputs}
As this recipe details installing a cluster starting from bare-metal, there is a requirement to define IP addresses and
gather hardware MAC addresses in order to support a controlled provisioning process. These values are necessarily unique
to the hardware being used, and this document uses variable substitution (\texttt{\$\{variable\}}) in the command-line
examples that follow to highlight where local site inputs are required. A summary of the required and optional variables
used throughout this recipe are presented below. Note that while the example definitions above correspond to a small
4-node compute subsystem, the compute parameters are defined in array format to accommodate logical extension to larger
node counts. \\

\vspace*{0.2cm}
\begin{tabular}{@{}>{\textbullet}l p{7cm} l}
& \texttt{\$\{sms\_name\}} & {\small \# Hostname for SMS server} \\
& \texttt{\$\{sms\_ip\}} & {\small \# Internal IP address on SMS server}  \\
\iftoggleverb{isxCAT}
& \texttt{\$\{domain\_name\}} & {\small \# Local network domain name}  \\
\fi
& \texttt{\$\{sms\_eth\_internal\}} & {\small \# Internal Ethernet interface on SMS} \\
\iftoggleverb{isWarewulf}
& \texttt{\$\{eth\_provision\}} & {\small \# Provisioning interface for computes} \\
\fi
& \texttt{\$\{internal\_network\}} & {\small \# Subnet network address for internal network} \\
& \texttt{\$\{internal\_netmask\}} & {\small \# Subnet netmask for internal network} \\
& \texttt{\$\{ntp\_server\}} & {\small \# Local ntp server for time synchronization} \\
& \texttt{\$\{bmc\_username\}} & {\small \# BMC username for use by IPMI} \\
& \texttt{\$\{bmc\_password\}} & {\small \# BMC password for use by IPMI} \\
& \texttt{\$\{num\_computes\}} & {\small \# Total \# of desired compute nodes} \\
& \texttt{\$\{c\_ip[0]\}}, \, \texttt{\$\{c\_ip[1]\}}, ... & {\small \# Desired compute node addresses} \\
& \texttt{\$\{c\_bmc[0]\}}, \texttt{\$\{c\_bmc[1]\}}, ... & {\small \# BMC addresses for computes} \\
& \texttt{\$\{c\_mac[0]\}}, \texttt{\$\{c\_mac[1]\}}, ... & {\small \# MAC addresses for computes} \\
& \texttt{\$\{c\_name[0]\}}, \texttt{\$\{c\_name[1]\}}, ... & {\small \# Host names for computes} \\
& \texttt{\$\{compute\_regex\}} & {\small \# Regex matching all compute node names (e.g. ``c*'')} \\
& \texttt{\$\{compute\_prefix\}} & {\small \# Prefix for compute node names (e.g. ``c'')} \\
\iftoggleverb{isxCAT}
& \texttt{\$\{iso\_path\}} & {\small \# Directory location of OS iso for \xCAT{} install} \\
\nottoggle{isxCATstateful}
{& \texttt{\$\{synclist\}} & {\small \# Filesystem location of \xCAT{} synclist file} \\}
\fi
\iftoggleverb{isxCATstateful}
& \texttt{\$\{ohpc\_repo\_dir\}} & {\small \# Directory location of local \OHPC{} repository mirror} \\
& \texttt{\$\{epel\_repo\_dir\}} & {\small \# Directory location of local EPEL repository
mirror} \\
\fi
\end{tabular}

\vspace*{0.2cm}
\noindent {Optional:}
\vspace*{0.1cm}

\begin{tabular}{@{}>{\textbullet}l p{7cm} l}
\iftoggleverb{isx86}
& \texttt{\$\{sysmgmtd\_host\}} & {\small \# BeeGFS System Management host name} \\
& \texttt{\$\{mgs\_fs\_name\}} & {\small \# Lustre MGS mount name} \\
& \texttt{\$\{sms\_ipoib\}} & {\small \# IPoIB address for SMS server} \\
& \texttt{\$\{ipoib\_netmask\}} & {\small \# Subnet netmask for internal IPoIB} \\
& \texttt{\$\{c\_ipoib[0]\}}, \texttt{\$\{c\_ipoib[1]\}}, ... & {\small \# IPoIB addresses for computes} \\
\fi
\iftoggleverb{isWarewulf}
& \texttt{\$\{kargs\}} & {\small \# Kernel boot arguments} \\
\fi
\end{tabular}




% Base Operating System --------------------------------------------

\section{Install Base Operating System (BOS)}
In an external setting, installing the desired BOS on a
{\em master} SMS host typically involves booting from a DVD ISO image on a new
server. With this approach, insert the \baseOS{} DVD, power cycle the host, and
follow the distro provided directions to install the BOS on your chosen {\em
master} host.  Alternatively, if choosing to use a pre-installed server, please
verify that it is provisioned with the required \baseOS{} distribution. \\

Prior to beginning the installation process of \OHPC{} components, several additional
considerations are noted here for the SMS host configuration. First, 
the installation recipe herein assumes that
the SMS host name is resolvable locally. Depending on the manner in which you
installed the BOS, there may be an adequate entry already defined
in \path{/etc/hosts}. If not, the following addition can be used to identify
your SMS host.
\begin{lstlisting}[language=bash,keywords={}]
[sms](*\#*) echo ${sms_ip} ${sms_name} >> /etc/hosts
\end{lstlisting}

\iftoggleverb{isWarewulf}
While it is theoretically possible to enable SELinux on a cluster provisioned
with \provisioner{}, 
doing so is beyond the scope of this document. Even the use of
permissive mode can be problematic and we therefore recommend disabling SELinux on the {\em
master} SMS host. If SELinux components are installed locally,
the \texttt{selinuxenabled} command can be used to determine if SELinux is
currently enabled. If enabled, consult the distro documentation for information
on how to disable. \\
\fi

\iftoggleverb{isxCAT} Another important consideration is SELinux. This security
mechanism is not supported by \xCAT{} and gets automatically disabled upon
\xCAT{} installation.  Thus this document assumes that SELinux is fully
deactivated. \\ \fi


Finally, provisioning services rely on DHCP, TFTP, and HTTP network protocols.
Depending on the local BOS configuration on the SMS host, default firewall
rules may prohibit these services. Consequently, this recipe assumes that the local
firewall running on the SMS host is disabled. If installed, the default
firewall service can be disabled as follows:


% ------------------------------------------------------------------

\section{Install \OHPC{} Components} \label{sec:basic_install}
\iftoggleverb{isxCATstateful}
With the BOS installed and booted on both the {\em master} server and the {\em
compute} nodes, the next step is to add desired \OHPC{} packages into the
cluster. The following subsections highlight this process.
\else
With the BOS installed and booted, the next step is to add desired \OHPC{} packages
onto the {\em master} server in order to provide provisioning and resource
management services for the rest of the cluster. The following subsections
highlight this process.
\fi


\subsection{Enable \OHPC{} repository for local use} \label{sec:enable_repo}
To begin, enable use of the \OHPC{} repository by adding it to the local list
of available package repositories. Note that this requires network access from
your {\em master} server to the \OHPC{} repository, or alternatively, that
the \OHPC{} repository be mirrored locally.  In cases where network external
connectivity is available, \OHPC{} provides an \texttt{ohpc-release} package
that includes GPG keys for package signing and enabling the repository.  The
example which follows illustrates installation of the \texttt{ohpc-release}
package directly from the \OHPC{} build server.

\iftoggleverb{isCentOS}
% CentOS
\begin{lstlisting}[language=bash,keywords={},basicstyle=\fontencoding{T1}\fontsize{7.6}{10}\ttfamily,
	literate={VER}{\OHPCVerTree{}}1 {OSREPO}{\OSTree{}}1 {TAG}{\OSTag{}}1 {ARCH}{\arch{}}1 {-}{-}1]
[sms](*\#*) yum install http://repos.openhpc.community/OpenHPC/VER/OSREPO/ARCH/ohpc-release-VER-1.TAG.ARCH.rpm
\end{lstlisting}
\else
% non-CentOS
\begin{lstlisting}[language=bash,keywords={},basicstyle=\fontencoding{T1}\fontsize{7.8}{10}\ttfamily,
	literate={VER}{\OHPCVerTree{}}1 {OSREPO}{\OSTree{}}1 {TAG}{\OSTag{}}1 {ARCH}{\arch{}}1 {-}{-}1]
[sms](*\#*) rpm -ivh http://repos.openhpc.community/OpenHPC/VER/OSREPO/ARCH/ohpc-release-VER-1.TAG.ARCH.rpm
\end{lstlisting}
\fi

\begin{center}
\begin{tcolorbox}[]
\small Many sites may find it useful or necessary to maintain a local copy of the
\OHPC{} repositories. To facilitate this need, standalone tar 
archives are provided -- one containing a repository of binary packages as well as any
available updates, and one containing a repository of source RPMS. The tar files
also contain a simple bash script to configure the package manager to use the
local repository after download. To use, simply unpack the tarball where you
would like to host the local repository and execute the \texttt{make\_repo.sh} script.
Tar files for this release can be found at \href{http://repos.openhpc.community/dist/\OHPCVersion}
        {\color{blue}{http://repos.openhpc.community/dist/\OHPCVersion}}
\end{tcolorbox}
\end{center}


% begin_ohpc_run
% ohpc_validation_newline
% ohpc_validation_comment Verify OpenHPC repository has been enabled before proceeding
% ohpc_validation_newline
% ohpc_command yum repolist | grep -q OpenHPC
% ohpc_command if [ $? -ne 0 ];then
% ohpc_command    echo "Error: OpenHPC repository must be enabled locally"
% ohpc_command    exit 1
% ohpc_command fi
% end_ohpc_run

In addition to the \OHPC{} package repository, the {\em master} host also
requires access to the standard base OS distro repositories in order to resolve
necessary dependencies. For \baseOS{}, the requirements are to have access to
both the base OS and EPEL repositories for which mirrors are freely available online:

\begin{itemize*}
\item CentOS-7 - Base 7.2.1511
  (e.g. \href{http://mirror.centos.org/centos-7/7.2.1511/os/x86\_64}
             {\color{blue}{http://mirror.centos.org/centos-7/7.2.1511/os/x86\_64}} )
\item EPEL 7 (e.g. \href{http://download.fedoraproject.org/pub/epel/7/x86\_64}
                        {\color{blue}{http://download.fedoraproject.org/pub/epel/7/x86\_64}} )
\end{itemize*}

\noindent The public EPEL repository will be enabled automatically upon installation of the 
\texttt{ohpc-release} package. Note that this requires the CentOS Extras
repository, which is shipped with CentOS and is enabled by default.

\subsection{Installation template}
The collection of command-line instructions that follow in this guide, when
combined with local site inputs, can be used to implement a
bare-metal system installation and configuration. The format of these commands
is intended to be usable via direct cut and paste (with variable substitution
for site-specific settings). Alternatively, the \OHPC{} documentation package
(\texttt{docs-ohpc}) includes a template script which includes a summary of all
of the commands used herein. This script can be used in conjunction with a
simple text file to define the local site variables defined in the previous
section (\S~\ref{sec:inputs}) and is provided as a convenience for
administrators. For additional information on accessing this script, please see
Appendix~\ref{appendix:template_script}.




\subsection{Add provisioning services on {\em master} node} \label{sec:add_provisioning}
With the \OHPC{} repository enabled, we can now begin adding desired components onto the
{\em master} server. This repository provides a number of aliases that group
logical components together in order to help aid in this process. For
reference, a complete list of available group aliases and RPM packages available
via \OHPC{} are provided in Appendix~\ref{appendix:manifest}. To add
support for provisioning services, the following commands illustrate addition
of a common base package followed by the Warewulf provisioning system.

%\nottoggle{isCentOS}{\clearpage}

% begin_ohpc_run
% ohpc_comment_header Add baseline OpenHPC and provisioning services \ref{sec:add_provisioning}
\begin{lstlisting}[language=bash,keywords={}]
# Install base meta-packages
[sms](*\#*) (*\install*) ohpc-base
[sms](*\#*) (*\install*) ohpc-warewulf
[sms](*\#*) (*\install*) hwloc-ohpc
\end{lstlisting}
% end_ohpc_run



\input{../../common/firewall}

% begin_ohpc_run
% ohpc_validation_comment Disable firewall 
\begin{lstlisting}[language=bash,keywords={}]
[sms](*\#*) systemctl disable firewalld
[sms](*\#*) systemctl stop firewalld
\end{lstlisting}
% end_ohpc_run

\input{../../common/enable_pxe}

HPC systems rely on synchronized clocks throughout the system and the
NTP protocol can be used to facilitate this synchronization. To enable NTP
services on the SMS host with a specific server \texttt{\$\{ntp\_server\}}, and
allow this server to serve as a local time server for the cluster,
issue the following:

% begin_ohpc_run
% ohpc_validation_comment Enable NTP services on SMS host
\begin{lstlisting}[language=bash,literate={-}{-}1,keywords={},upquote=true,keepspaces]
[sms](*\#*) systemctl enable chronyd.service
[sms](*\#*) echo "local stratum 10" >> /etc/chrony.conf
[sms](*\#*) echo "server ${ntp_server}" >> /etc/chrony.conf
[sms](*\#*) echo "allow all" >> /etc/chrony.conf
[sms](*\#*) systemctl restart chronyd
\end{lstlisting}
% end_ohpc_run

\begin{center}
\begin{tcolorbox}[]
\small Note that the ``allow all'' option specified for the chrony time daemon
allows all servers on the local network to be able to synchronize with the SMS
host. Alternatively, you can restrict access to fixed IP ranges and an example
config line allowing access to a local class B subnet is as follows:
\begin{lstlisting}[language=bash]
allow 192.168.0.0/16
\end{lstlisting}
\end{tcolorbox}
\end{center}


\subsection{Add resource management services on {\em master} node} \label{sec:add_rm}
\OHPC{} provides multiple options for distributed resource management. 
The following command adds the \SLURM{} workload manager server components to the
chosen {\em master} host. Note that client-side components will be added to
the corresponding compute image in a subsequent step.

% begin_ohpc_run
% ohpc_comment_header Add resource management services on master node \ref{sec:add_rm}
\begin{lstlisting}[language=bash,keywords={}]
# Install slurm server meta-package
[sms](*\#*) (*\install*) ohpc-slurm-server

# Use ohpc-provided file for starting SLURM configuration
[sms](*\#*) cp /etc/slurm/slurm.conf.ohpc /etc/slurm/slurm.conf
# Setup default cgroups file
[sms](*\#*) cp /etc/slurm/cgroup.conf.example /etc/slurm/cgroup.conf

# Identify resource manager hostname on master host
[sms](*\#*) perl -pi -e "s/SlurmctldHost=\S+/SlurmctldHost=${sms_name}/" /etc/slurm/slurm.conf
\end{lstlisting}
% end_ohpc_run

There are a wide variety of configuration options and plugins available
for \SLURM{} and the example config file illustrated above targets a fairly
basic installation. In particular, job completion data will be stored in a text
file (\texttt{/var/log/slurm\_jobcomp.log)} that can be used to log simple
accounting information. Sites who desire more detailed information, or want to
aggregate accounting data from multiple clusters, will likely want to enable the
database accounting back-end.  This requires a number of additional local modifications
(on top of installing \texttt{slurm-slurmdbd-ohpc}), and users are advised to
consult the online \href{https://slurm.schedmd.com/accounting.html}{\color{blue}{documentation}}
for more detailed information on setting up a database configuration for \SLURM{}.

\begin{center}
\begin{tcolorbox}[]
  \small SLURM requires enumeration of the physical hardware characteristics for
  compute nodes under its control. In particular, three configuration parameters
  combine to define consumable compute resources: {\em Sockets}, {\em
  CoresPerSocket}, and {\em ThreadsPerCore}. The default configuration file
  provided via \OHPC{} assumes the nodes are named c1-c4 and are dual-socket, 8
  cores per socket, and two threads per core for this 4-node example. If this
  does not reflect your local hardware, please update the configuration file at
  \path{/etc/slurm/slurm.conf} accordingly to match your nodes names and
  particular hardware. Be sure to run \texttt{scontrol reconfigure} to notify
  SLURM of the changes. Note that the SLURM project provides an easy-to-use
  online configuration tool that can be accessed
  \href{https://slurm.schedmd.com/configurator.html}{\color{blue} here}.
\end{tcolorbox}
\end{center}

% begin_ohpc_run
% ohpc_comment_header Update node configuration for slurm.conf
% ohpc_command if [[ ${update_slurm_nodeconfig} -eq 1 ]];then
% ohpc_indent 5
% ohpc_command perl -pi -e "s/^NodeName=.+$/#/" /etc/slurm/slurm.conf
% ohpc_command perl -pi -e "s/ Nodes=c\S+ / Nodes=${compute_prefix}[1-${num_computes}] /" /etc/slurm/slurm.conf
% ohpc_command echo -e ${slurm_node_config} >> /etc/slurm/slurm.conf
% ohpc_indent 0
% ohpc_command fi
% end_ohpc_run

Other versions of this guide are available that describe installation of alternate
resource management systems, and they can be found in the \texttt{docs-ohpc}
package.



\subsection{Add \InfiniBand{} support services on {\em master} node} \label{sec:add_ofed}

The following command adds OFED and PSM support using base distro-provided drivers
to the chosen {\em master} host.

% begin_ohpc_run
% ohpc_comment_header Add InfiniBand support services on master node \ref{sec:add_ofed}
\begin{lstlisting}[language=bash,keywords={}]
[sms](*\#*) (*\groupinstall*) "InfiniBand Support"
[sms](*\#*) (*\install*) infinipath-psm

# Load IB drivers
[sms](*\#*) systemctl start rdma
\end{lstlisting}
% end_ohpc_run

With the \InfiniBand{} drivers included, you can also enable (optional) IPoIB functionality
which provides a mechanism to send IP packets over the IB network. If you plan
to mount a \Lustre{} file system over \InfiniBand{} (see \S\ref{sec:lustre_client}
for additional details), then having IPoIB enabled is a requirement for the
\Lustre{} client. \OHPC{} provides a template configuration file to aid in setting up
an {\em ib0} interface on the {\em master} host. To use, copy the template
provided and update the \texttt{\$\{sms\_ipoib\}} and
\texttt{\$\{ipoib\_netmask\}} entries to match local desired settings (alter ib0
naming as appropriate if system contains dual-ported or multiple HCAs). 

% begin_ohpc_run
% ohpc_validation_newline
% ohpc_command if [[ ${enable_ipoib} -eq 1 ]];then
% ohpc_indent 5
% ohpc_validation_comment Enable ib0
\begin{lstlisting}[language=bash,literate={-}{-}1,keywords={},upquote=true]
[sms](*\#*) cp /opt/ohpc/pub/examples/network/centos/ifcfg-ib0 /etc/sysconfig/network-scripts

# Define local IPoIB address and netmask
[sms](*\#*) perl -pi -e "s/master_ipoib/${sms_ipoib}/" /etc/sysconfig/network-scripts/ifcfg-ib0
[sms](*\#*) perl -pi -e "s/ipoib_netmask/${ipoib_netmask}/" /etc/sysconfig/network-scripts/ifcfg-ib0

# Initiate ib0
[sms](*\#*) ifup ib0
\end{lstlisting}
% ohpc_indent 0
% ohpc_command fi
% end_ohpc_run

Each infiniband network requires a subnet manager. It is possible to have more
than one, in which case, one will act as master, and any other subnet managers 
will act as slaves that will take over should the master subnet manager fail.

There are two types of subnet managers, software based and hardware based.
Hardware based subnet managers are typically part of the firmware of the attached
Infiniband switch. A software subnet manager is not necessary if a hardware based
subnet manager is active.


\begin{center}
\begin{tcolorbox}[]
\small
The default opensm package which is available as an optional package in the
package group "InfiniBand Support" can be installed using the command,
\begin{lstlisting}[language=bash]
[sms](*\#*) yum install `repoquery -g -l --grouppkgs=all "Infiniband Support"`
[sms](*\#*) yum -y --installroot=/opt/ohpc/admin/images/centos7.2 install `repoquery -g -l --grouppkgs=all "Infiniband Support"`
\end{lstlisting}
Once the opensm package is installed on master node and at the chroot
environment,the opensm service can be started and enabled at boot time on the 
designated node(s) inside of the cluster.  

\end{tcolorbox}
\end{center}

\vspace*{-0.15cm}
\subsection{Complete basic Warewulf setup for {\em master} node} \label{sec:setup_ww}
\input{../../common/warewulf_setup}

% begin_ohpc_run
% ohpc_comment_header Complete basic Warewulf setup for master node \ref{sec:setup_ww}
%\begin{verbatim}

\begin{lstlisting}[language=bash,literate={-}{-}1,keywords={},upquote=true,keepspaces]
# Configure Warewulf provisioning to use desired internal interface
[sms](*\#*) perl -pi -e "s/device = eth1/device = ${sms_eth_internal}/" /etc/warewulf/provision.conf

# Enable tftp service for compute node image distribution
[sms](*\#*) perl -pi -e "s/^\s+disable\s+= yes/ disable = no/" /etc/xinetd.d/tftp

# Enable http access for Warewulf cgi-bin directory to support newer apache syntax
[sms](*\#*) export MODFILE=/etc/httpd/conf.d/warewulf-httpd.conf
[sms](*\#*) perl -pi -e "s/cgi-bin>\$/cgi-bin>\n Require all granted/" $MODFILE

[sms](*\#*) perl -pi -e "s/Allow from all/Require all granted/" $MODFILE
[sms](*\#*) perl -ni -e "print unless /^\s+Order allow,deny/" $MODFILE

# Enable internal interface for provisioning
[sms](*\#*) ifconfig ${sms_eth_internal} ${sms_ip} netmask ${internal_netmask} up

# Restart/enable relevant services to support provisioning
[sms](*\#*) systemctl restart xinetd
[sms](*\#*) systemctl enable mariadb.service
[sms](*\#*) systemctl restart mariadb
[sms](*\#*) systemctl enable httpd.service
[sms](*\#*) systemctl restart httpd
\end{lstlisting}
%\end{verbatim}
% end_ohpc_run


\subsection{Define {\em compute} image for provisioning}

With the provisioning services enabled, the next step is to define and
customize a system image that can subsequently be used to provision one or more
{\em compute} nodes. The following subsections highlight this process.

\subsubsection{Build initial BOS image} \label{sec:assemble_bos}

The \OHPC{} build of \Warewulf{} includes specific enhancements enabling support for
\baseOS{}. The following steps illustrate the process to build a minimal, default
image for use with \Warewulf{}. We begin by defining a directory structure on the 
{\em master} host that will represent the root filesystem of the compute node. The 
default location for this example is in
\texttt{/opt/ohpc/admin/images/\baseos{}}.

\begin{center}
  \begin{tcolorbox}[]
    \small Note that \Warewulf{} is configured by default to access an external repository
    (vault.centos.org) during the \texttt{wwmkchroot} process.
    If the master host cannot reach the public CentOS mirrors, or if you
    prefer to access a locally cached mirror, set the
    \texttt{\$\{BOS\_MIRROR\}} environment variable to your desired repo
    location and update the template file 
    {\em prior} to running the \texttt{wwmkchroot} command below. For
    example:

% begin_ohpc_run
% ohpc_command if [ ! -z ${BOS_MIRROR+x} ]; then
% ohpc_indent 5
\begin{lstlisting}[language=bash,keywords={}]
# Override default OS repository (optional) - set BOS_MIRROR variable to desired repo location
[sms](*\#*) perl -pi -e "s#^YUM_MIRROR=(\S+)#YUM_MIRROR=${BOS_MIRROR}#" \
   /usr/libexec/warewulf/wwmkchroot/centos-7.tmpl
\end{lstlisting}
% ohpc_indent 0
% ohpc_command fi
% end_ohpc_run

\end{tcolorbox}
\end{center}

% begin_ohpc_run
% ohpc_comment_header Create compute image for Warewulf \ref{sec:assemble_bos}
\begin{lstlisting}[language=bash,literate={-}{-}1,keywords={},upquote=true,keepspaces]
# Define chroot location 
[sms](*\#*) export CHROOT=/opt/ohpc/admin/images/centos7.2

# Build initial chroot image
[sms](*\#*) wwmkchroot centos-7 $CHROOT
\end{lstlisting}
% end_ohpc_run

\subsubsection{Add \OHPC{} components} \label{sec:add_components}

The \texttt{wwmkchroot} process used in the previous step is designed to
provide a minimal \baseOS{} configuration. Next, we add additional components
to include resource management client services, NTP support, and
other additional packages to support the default \OHPC{} environment. This
process augments the chroot-based install performed by \texttt{wwmkchroot} to
modify the base provisioning image and will access the BOS and \OHPC{}
repositories to resolve package install requests. We begin by installing a few
common base packages:

% begin_ohpc_run
% ohpc_comment_header Add OpenHPC base components to compute image \ref{sec:add_components}
\begin{lstlisting}[language=bash,literate={-}{-}1,keywords={},upquote=true]
# Install compute node base meta-package
[sms](*\#*) (*\chrootinstall*) ohpc-base-compute
\end{lstlisting}
% end_ohpc_run

To access the remote
repositories by hostname (and not IP addresses), the chroot environment needs
to be updated to enable DNS resolution. Assuming that the {\em master} host has
a working DNS configuration in place, the chroot environment can be updated
with a copy of the configuration as follows:

% begin_ohpc_run
% ohpc_comment_header Add OpenHPC components to compute image \ref{sec:add_components}
\begin{lstlisting}[language=bash,literate={-}{-}1,keywords={},upquote=true]
[sms](*\#*) cp -p /etc/resolv.conf $CHROOT/etc/resolv.conf
\end{lstlisting}
% end_ohpc_run

\noindent Now, we can include additional required components to the compute
instance including resource manager client, NTP, and development environment modules support.



\noindent Now, we can include additional components to the compute instance using
\texttt{yum} to install into the chroot location defined previously:

% begin_ohpc_run
% ohpc_validation_comment Add OpenHPC components to compute instance
\begin{lstlisting}[language=bash,literate={-}{-}1,keywords={},upquote=true]
# Add Slurm client support
[sms](*\#*) (*\groupchrootinstall*) ohpc-slurm-client

# Add IB support and enable
[sms](*\#*) (*\groupchrootinstall*) "InfiniBand Support"
[sms](*\#*) (*\chrootinstall*) infinipath-psm
[sms](*\#*) chroot $CHROOT systemctl enable rdma

# Add Network Time Protocol (NTP) support
[sms](*\#*) (*\chrootinstall*) ntp

# Add kernel drivers
[sms](*\#*) (*\chrootinstall*) kernel

# Include modules user environment
[sms](*\#*) (*\chrootinstall*) lmod-ohpc
\end{lstlisting}
% end_ohpc_run

\subsubsection{Customize system configuration} \label{sec:master_customization}

Prior to assembling the image, it is advantageous to perform any additional
customization within the chroot environment created for the desired {\em
 compute} instance. The following steps document the process to add a local
{\em ssh} key created by \Warewulf{} to support remote access, identify the
resource manager server, configure NTP for compute resources, and enable \NFS{}
mounting of a \$HOME file system and the public \OHPC{} install path
(\texttt{/opt/ohpc/pub}) that will be hosted by the {\em master} host in this
example configuration.
%The \NFS{} exporting options use an address/netmask
%combination to limit the export scope to the defined compute nodes.

% begin_ohpc_run
% ohpc_comment_header Customize system configuration \ref{sec:master_customization}
\begin{lstlisting}[language=bash,literate={-}{-}1,keywords={},upquote=true]
# add new cluster key to base image
[sms](*\#*) wwinit ssh_keys
[sms](*\#*) cat ~/.ssh/cluster.pub >> $CHROOT/root/.ssh/authorized_keys

# add NFS client mounts of /home and /opt/ohpc/pub to base image
[sms](*\#*) echo "${sms_ip}:/home /home nfs nfsvers=3,rsize=1024,wsize=1024,cto 0 0" >> $CHROOT/etc/fstab
[sms](*\#*) echo "${sms_ip}:/opt/ohpc/pub /opt/ohpc/pub nfs nfsvers=3 0 0" >> $CHROOT/etc/fstab

# Identify resource manager host name on master host
[sms](*\#*) perl -pi -e "s/ControlMachine=\S+/ControlMachine=${sms_name}/" /etc/slurm/slurm.conf
\end{lstlisting}
% end_ohpc_run

% begin_ohpc_run
\begin{lstlisting}[language=bash,literate={-}{-}1,keywords={},upquote=true]
# Export /home and OpenHPC public packages from master server to cluster compute nodes
[sms](*\#*) echo "/home *(rw,no_subtree_check,fsid=10,no_root_squash)" >> /etc/exports
[sms](*\#*) echo "/opt/ohpc/pub *(ro,no_subtree_check,fsid=11)" >> /etc/exports
[sms](*\#*) exportfs -a
[sms](*\#*) systemctl restart nfs
[sms](*\#*) systemctl enable nfs-server
# Enable NTP time service on computes and identify master host as local NTP server
[sms](*\#*) chroot $CHROOT systemctl enable ntpd
[sms](*\#*) echo "server ${sms_ip}" >> $CHROOT/etc/ntp.conf
\end{lstlisting}
% end_ohpc_run


\begin{center}
\begin{tcolorbox}[]
  \small Slurm requires enumeration of the physical hardware characteristics
  for compute nodes under its control. In particular, three configuration
  parameters combine to define consumable compute resources: {\em Sockets},
  {\em CoresPerSocket}, and {\em ThreadsPerCore}. The default configuration
  file provided via \OHPC{} assumes dual-socket, 8 cores per socket, and two
  threads per core for this 4-node example. If this does not reflect your local
  hardware, please update the configuration file at
  \path{/etc/slurm/slurm.conf} accordingly to match your particular hardware.
\end{tcolorbox}
\end{center}

% Additional commands when additional computes are requested

% begin_ohpc_run
% ohpc_validation_newline
% ohpc_validation_comment Update basic slurm configuration if additional computes defined
% ohpc_command if [ ${num_computes} -gt 4 ];then
% ohpc_command    perl -pi -e "s/^NodeName=(\S+)/NodeName=c[1-${num_computes}]/" /etc/slurm/slurm.conf
% ohpc_command    perl -pi -e "s/^PartitionName=normal Nodes=(\S+)/PartitionName=normal Nodes=c[1-${num_computes}]/" /etc/slurm/slurm.conf

% ohpc_command    perl -pi -e "s/^NodeName=(\S+)/NodeName=c[1-${num_computes}]/" $CHROOT/etc/slurm/slurm.conf
% ohpc_command    perl -pi -e "s/^PartitionName=normal Nodes=(\S+)/PartitionName=normal Nodes=c[1-${num_computes}]/" $CHROOT/etc/slurm/slurm.conf
% ohpc_command fi
% end_ohpc_run

%\clearpage
\subsubsection{Additional Customization ({\em optional})} \label{sec:addl_customizations}
This section highlights common additional customizations that can {\em
optionally} be applied to the local cluster environment. These customizations
include:

\begin{multicols}{2}
\begin{itemize*}
\iftoggleverb{isx86}
\item Add InfiniBand or Omni-Path drivers
\item Increase memlock limits
\fi

\nottoggle{ispbs}{\item Restrict ssh access to compute resources}

\iftoggleverb{isx86}
\item Add \beegfs{} client
\item Add \Lustre{} client
\fi

\iftoggle{isWarewulf}{\item Enable syslog forwarding}

\item Add \Nagios{} Core monitoring
\item Add \Sensys{} monitoring
\item Add \clustershell{}
\item Add \mrsh{}
\item Add \genders{}
%%\item Add \powerman{}
\item Add \conman{}  
\item Add \GEOPM{}
\end{itemize*}
\end{multicols}

\noindent Details on the steps required for each of these customizations are
discussed further in the following sections.


\paragraph{Increase locked memory limits}
In order to utilize \InfiniBand{} as the underlying high speed interconnect, it is
generally necessary to increase the locked memory settings for system
users. This can be accomplished by updating
the \texttt{/etc/security/limits.conf} file and this should be performed within
the {{\em compute} image and on all job submission hosts. In this recipe, jobs
are submitted from the {\em master} host, and the following commands can be
used to update the maximum locked memory settings on both the master host and
the compute image:

% begin_ohpc_run
% ohpc_comment_header Additional customizations \ref{sec:addl_customizations}
\begin{lstlisting}[language=bash,keywords={},upquote=true]
# Update memlock settings on master
[sms](*\#*) perl -pi -e 's/# End of file/\* soft memlock unlimited\n$&/s' /etc/security/limits.conf
[sms](*\#*) perl -pi -e 's/# End of file/\* hard memlock unlimited\n$&/s' /etc/security/limits.conf

# Update memlock settings within compute image
[sms](*\#*) perl -pi -e 's/# End of file/\* soft memlock unlimited\n$&/s' $CHROOT/etc/security/limits.conf
[sms](*\#*) perl -pi -e 's/# End of file/\* hard memlock unlimited\n$&/s' $CHROOT/etc/security/limits.conf
\end{lstlisting}
% end_ohpc_run




\paragraph{Enable ssh control via resource manager} 
\input{../../common/slurm_pam}

\paragraph{Add \Lustre{} client} \label{sec:lustre_client}
To add \Lustre{} client support on the cluster, it necessary to install the client
and associated modules on each host needing to access a \Lustre{} file system. In
this recipe, it is assumed that the \Lustre{} file system is hosted by servers
that are pre-existing and are not part of the install process. Outlining the
variety of \Lustre{} client mounting options is beyond the scope of this document,
%(please consult \Lustre{} documentation for more details on failover configuration
%support and networking options),
but the general requirement is to add a mount entry for the desired file system
that defines the management server (MGS) and underlying network transport
protocol. To add client mounts on both the {\em master} server and {\em
compute} image, the following commands can be used. Note that the \Lustre{} file
system to be mounted is identified by the \texttt{\$\{mgs\_fs\_name\}} variable.
In this example, the file system is configured to be mounted locally
as \path{/mnt/lustre}.


% begin_ohpc_run
% ohpc_validation_newline 
% ohpc_validation_comment Enable Optional packages
% ohpc_validation_newline
% ohpc_command if [[ ${enable_lustre_client} -eq 1 ]];then
% ohpc_indent 5

% ohpc_validation_comment Install Lustre client on master
\begin{lstlisting}[language=bash,keywords={},upquote=true]
# Add Lustre client software to master host
[sms](*\#*) (*\install*) lustre-client-ohpc lustre-client-ohpc-modules
\end{lstlisting}
% end_ohpc_run

% begin_ohpc_run
% ohpc_validation_comment Enable lustre in WW compute image
\begin{lstlisting}[language=bash,keywords={},upquote=true]
# Include Lustre client software in compute image
[sms](*\#*) (*\chrootinstall*) lustre-client-ohpc lustre-client-ohpc-modules

# Include mount point and file system mount in compute image
[sms](*\#*) mkdir $CHROOT/mnt/lustre
[sms](*\#*) echo "${mgs_fs_name} /mnt/lustre lustre defaults,_netdev,localflock 0 0" >> $CHROOT/etc/fstab
\end{lstlisting}
% end_ohpc_run

The default underlying network type used by \Lustre{} is {\em tcp}. If your
external \Lustre{} file system is to be mounted using a network type other than
{\em tcp}, additional configuration files are necessary to identify the desired
network type. The example below illustrates creation of modprobe configuration files
instructing \Lustre{} to use an \InfiniBand{} network with the \textbf{o2ib} LNET driver
attached to \texttt{ib0}. Note that these modifications are made to both the
{\em master} host and {\em compute} image.

%x\clearpage
% begin_ohpc_run
% ohpc_validation_comment Enable o2ib for Lustre
\begin{lstlisting}[language=bash,keywords={},upquote=true]
[sms](*\#*) echo "options lnet networks=o2ib(ib0)" >> /etc/modprobe.d/lustre.conf
[sms](*\#*) echo "options lnet networks=o2ib(ib0)" >> $CHROOT/etc/modprobe.d/lustre.conf
\end{lstlisting}
% end_ohpc_run

With the \Lustre{} configuration complete, the client can be mounted on the {\em master}
host as follows:
% begin_ohpc_run
% ohpc_validation_comment mount Lustre client on master
\begin{lstlisting}[language=bash,keywords={},upquote=true]
[sms](*\#*) mkdir /mnt/lustre
[sms](*\#*) mount -t lustre -o localflock ${mgs_fs_name} /mnt/lustre
\end{lstlisting}
% ohpc_indent 0
% ohpc_command fi
% ohpc_validation_newline
% end_ohpc_run

\paragraph{Add \Nagios{} monitoring}
\Nagios{} is an open source infrastructure monitoring package that monitors
servers, switches, applications, and services and offers user-defined alerting
facilities. As provided by \OHPC{}, it consists of a base monitoring daemon and
a set of plug-ins for monitoring various aspects of an HPC cluster. The
following commands can be used to install and configure a \Nagios{} server on the {\em
master} node, and add the facility to run tests and gather metrics from
provisioned {\em compute} nodes.

% begin_ohpc_run
% ohpc_command if [[ ${enable_nagios} -eq 1 ]];then
% ohpc_indent 5
% ohpc_validation_comment Install Nagios on master and vnfs image
\begin{lstlisting}[language=bash,keywords={},upquote=true]
# Install Nagios base, Remote Plugin Engine, and Plugins on master host
[sms](*\#*) (*\groupinstall*) ohpc-nagios

# Also install in compute node image
[sms](*\#*) (*\chrootinstall*) nagios-plugins-all-ohpc nrpe-ohpc

# Enable and configure NRPE in compute image
[sms](*\#*) chroot $CHROOT systemctl enable nrpe
[sms](*\#*) perl -pi -e "s/^allowed_hosts=/# allowed_hosts=/" $CHROOT/etc/nagios/nrpe.cfg
[sms](*\#*) echo "nrpe 5666/tcp # NRPE"         >> $CHROOT/etc/services
[sms](*\#*) echo "nrpe : ${sms_ip}  : ALLOW"    >> $CHROOT/etc/hosts.allow
[sms](*\#*) echo "nrpe : ALL : DENY"            >> $CHROOT/etc/hosts.allow
[sms](*\#*) chroot $CHROOT /usr/sbin/useradd -c "NRPE user for the NRPE service" -d /var/run/nrpe \
        -r -g nrpe -s /sbin/nologin nrpe
[sms](*\#*) chroot $CHROOT /usr/sbin/groupadd -r nrpe

# Configure remote services to test on compute nodes
[sms](*\#*) mv /etc/nagios/conf.d/services.cfg.example /etc/nagios/conf.d/services.cfg

# Define compute nodes as hosts to monitor
[sms](*\#*) mv /etc/nagios/conf.d/hosts.cfg.example /etc/nagios/conf.d/hosts.cfg
[sms](*\#*) for ((i=0; i<$num_computes; i++)) ; do
              perl -pi -e "s/HOSTNAME$(($i+1))/${c_name[$i]}/ || s/HOST$(($i+1))_IP/${c_ip[$i]}/" \
              /etc/nagios/conf.d/hosts.cfg
           done

# Update location of mail binary for alert commands
[sms](*\#*) perl -pi -e "s/ \/bin\/mail/ \/usr\/bin\/mailx/g" /etc/nagios/objects/commands.cfg

# Update email address of contact for alerts
[sms](*\#*) perl -pi -e "s/nagios\@localhost/root\@${sms_name}/" /etc/nagios/objects/contacts.cfg

# Add check_ssh command for remote hosts
[sms](*\#*) echo command[check_ssh]=/usr/lib64/nagios/plugins/check_ssh localhost \
        >> $CHROOT/etc/nagios/nrpe.cfg

# Enable Nagios on master, and configure
[sms](*\#*) chkconfig nagios on
[sms](*\#*) systemctl start nagios
[sms](*\#*) chmod u+s `which ping`
\end{lstlisting}
% ohpc_indent 0
% ohpc_command fi
% end_ohpc_run



\clearpage
\paragraph{Add \Ganglia{} monitoring}
\Ganglia{} is a scalable distributed system monitoring tool for high-performance
computing systems such as clusters and grids. It allows the user to remotely
view live or historical statistics (such as CPU load averages or network
utilization) for all machines running the {\em gmond} daemon. The following
commands can be used to enable \Ganglia{} to monitor both the {\em master} and
{\em compute} hosts.

% begin_ohpc_run
% ohpc_validation_newline
% ohpc_command if [[ ${enable_ganglia} -eq 1 ]];then
% ohpc_indent 5
% ohpc_validation_comment Install Ganglia on master
\begin{lstlisting}[language=bash,keywords={},upquote=true]
# Install Ganglia meta-package on master
[sms](*\#*) (*\install*) ohpc-ganglia

# Install Ganglia compute node daemon
[sms](*\#*) (*\chrootinstall*) ganglia-gmond-ohpc

# Use example configuration script to enable unicast receiver on master host
[sms](*\#*) cp /opt/ohpc/pub/examples/ganglia/gmond.conf /etc/ganglia/gmond.conf
[sms](*\#*) perl -pi -e "s/<sms>/${sms_name}/" /etc/ganglia/gmond.conf

# Add configuration to compute image and provide gridname
[sms](*\#*) cp /etc/ganglia/gmond.conf $CHROOT/etc/ganglia/gmond.conf
[sms](*\#*) echo "gridname MySite" >> /etc/ganglia/gmetad.conf

# Start and enable Ganglia services
[sms](*\#*) systemctl enable gmond
[sms](*\#*) systemctl enable gmetad
[sms](*\#*) systemctl start gmond
[sms](*\#*) systemctl start gmetad
[sms](*\#*) chroot $CHROOT systemctl enable gmond

# Restart web server
[sms](*\#*) systemctl try-restart httpd
\end{lstlisting}
% ohpc_indent 0
% ohpc_command fi
% end_ohpc_run

\noindent Once enabled and running, Ganglia should provide access to a web-based
monitoring console on the {\em master} host. Read access to monitoring metrics
will be enabled by default and can be accessed via a web browser. When running
a web browser directly on the {\em master} host, the Ganglia top-level overview
is available
at \href{http://localhost/ganglia}{\color{blue}{http://localhost/ganglia}}.
When accessing remotely, replace {\em localhost} with the chosen name of your
master host (\texttt{\$\{sms\_name\}}).




\paragraph{Add \clustershell{}}
\clustershell{} is an event-based Python library to execute commands in parallel
across cluster nodes. Installation and basic configuration defining three node
groups ({\em adm}, {\em compute}, and {\em all}) is as follows:

% begin_ohpc_run
% ohpc_validation_newline
% ohpc_command if [[ ${enable_clustershell} -eq 1 ]];then
% ohpc_indent 5
% ohpc_validation_comment Install clustershell
\begin{lstlisting}[language=bash,keywords={},upquote=true]
# Install ClusterShell
[sms](*\#*) (*\install*) clustershell-ohpc

# Setup node definitions
[sms](*\#*) cd /etc/clustershell/groups.d
[sms](*\#*) mv local.cfg local.cfg.orig
[sms](*\#*) echo "adm: ${sms_name}" > local.cfg
[sms](*\#*) echo "compute: ${compute_prefix}[1-${num_computes}]" >> local.cfg
[sms](*\#*) echo "all: @adm,@compute" >> local.cfg
\end{lstlisting}
% ohpc_indent 0
% ohpc_command fi
% end_ohpc_run



\paragraph{Add \mrsh{}}
\mrsh{} is a secure remote shell utility, like ssh, which uses \MUNGE{} 
for authentication and encryption. By using the \MUNGE{} package, \mrsh{} provides 
shell access to systems using the same \MUNGE{} key  without having to track {ssh} keys. 
Like {ssh}, \mrsh{} provides a  remote copy command, {\em mrcp}, and can be used as 
a {\em rcmd} by {\em pdsh}. Example installation and configuration is as follows:

%\clearpage
% begin_ohpc_run
% ohpc_validation_newline
% ohpc_command if [[ ${enable_mrsh} -eq 1 ]];then
% ohpc_indent 5
% ohpc_validation_comment Install mrsh
\begin{lstlisting}[language=bash,keywords={},upquote=true]
# Install mrsh
[sms](*\#*) (*\install*) mrsh-ohpc mrsh-rsh-compat-ohpc
[sms](*\#*) (*\chrootinstall*) mrsh-ohpc mrsh-rsh-compat-ohpc mrsh-server-ohpc

# Identify mshell and mlogin in services file
[sms](*\#*) echo "mshell          21212/tcp                  # mrshd" >> /etc/services
[sms](*\#*) echo "mlogin            541/tcp                  # mrlogind" >> /etc/services

# Enable xinetd in compute node image
[sms](*\#*) chroot $CHROOT systemctl enable xinetd
\end{lstlisting}
% ohpc_indent 0
% ohpc_command fi
% end_ohpc_run



\paragraph{Add \genders{}}
\genders{} is a static cluster configuration database or node typing database
used for cluster configuration management. Other tools and users can access the
\genders{} database in order to make decisions about where an action, or even
what action, is appropriate based on associated types or "\genders{}."

Values may also be assigned to and retrieved from a {\em gender} to provide
further granularity.

% begin_ohpc_run
% ohpc_validation_newline
% ohpc_command if [ ${enable_genders} -eq 1 ];then
% ohpc_indent 5
% ohpc_validation_comment Install genders
\begin{lstlisting}[language=bash,keywords={},upquote=true]
# generate a sample genders file
[sms](*\#*) echo -e "${master_name}\tsms" > /etc/genders
[sms](*\#*) for ((i=0; i<$num_computes; i++)) ; do
              echo -e "${c_name[$i]}\tcompute,bmc=${c_bmc[$i]}"
           done >> /etc/genders
\end{lstlisting}
% ohpc_indent 0
% ohpc_command fi
% end_ohpc_run



%% \paragraph{Add \powerman{}}
%% \powerman{} abstracts many different kinds of power control interfaces (IPMI,
smart PDU, etc) into a single clean interface. \powerman{} accepts node ranges
for controlling sets of nodes, and is the default {\em resetcmd} for \conman{}
as configured by this recipe.

% begin_ohpc_run
% ohpc_validation_newline
% ohpc_command if [[ ${enable_powerman} -eq 1 ]];then
% ohpc_indent 5
% ohpc_validation_comment Optionally, install powerman
\begin{lstlisting}[language=bash,keywords={},upquote=true]
# Install powerman
[sms](*\#*) (*\install*) powerman-ohpc

# Create a basic powerman.conf
[sms](*\#*) cp /etc/powerman/powerman.conf{.example,}

[sms](*\#*) perl -pi -e 's/^\#(tcpwrappers yes)/$1/' /etc/powerman/powerman.conf
[sms](*\#*) perl -pi -e 's/^\#(listen "0.0.0.0:10101")/$1/' /etc/powerman/powerman.conf
[sms](*\#*) perl -pi -e 's/^\#(include "\/etc\/powerman\/ipmipower\.dev")/$1/' \
            /etc/powerman/powerman.conf
[sms](*\#*) for ((i=0; i<$num_computes; i++)) ; do
            perl -pi -e 'print "device \"ipmi'$i'\" \"ipmipower\" \"/usr/sbin/ipmipower -h ".
                "'${c_bmc[$i]}' -u '$bmc_username' -p ".
                "'${IPMI_PASSWORD:-undefined}'|&\"\n" if(/^\#device "ipmi1"/);' /etc/powerman/powerman.conf
        done
[sms](*\#*) for ((i=0; i<$num_computes; i++)) ; do
            perl -pi -e 'print "node \"'${c_name[$i]}'\" \"ipmi'$i'\" \"'${c_bmc[$i]}'\"\n"
                if(/^\#node "t1"/);' /etc/powerman/powerman.conf
        done

# Start powerman
[sms](*\#*) systemctl start powerman

# Check power status
[sms](*\#*) pm -q
\end{lstlisting}
% ohpc_indent 0
% ohpc_command fi
% end_ohpc_run



\paragraph{Add \conman{}} \label{sec:add_conman}
\conman{} is a serial console management program designed to support a large
number of console devices and simultaneous users. It supports logging console
device output and connecting to compute node consoles via IPMI
serial-over-lan. Installation and example configuration is outlined below.

% begin_ohpc_run
% ohpc_validation_newline
% ohpc_validation_comment Optionally, enable conman and configure
% ohpc_command if [[ ${enable_ipmisol} -eq 1 ]];then
% ohpc_indent 5
\begin{lstlisting}[language=bash,keywords={},upquote=true]
# Install conman to provide a front-end to compute consoles and log output
[sms](*\#*) (*\install*) conman-ohpc

# Configure conman for computes (note your IPMI password is required for console access)
[sms](*\#*) for ((i=0; i<$num_computes; i++)) ; do
              echo -n 'CONSOLE name="'${c_name[$i]}'" dev="ipmi:'${c_bmc[$i]}'" '
              echo 'ipmiopts="'U:${bmc_username},P:${IPMI_PASSWORD:-undefined},W:solpayloadsize'"'
        done >> /etc/conman.conf

# Enable and start conman
[sms](*\#*) systemctl enable conman
[sms](*\#*) systemctl start conman
\end{lstlisting}
% ohpc_indent 0
% ohpc_command fi
% end_ohpc_run

\iftoggleverb{isWarewulf}
\noindent Note that an additional kernel boot option is typically necessary to
enable serial console output. This option is highlighted in \S\ref{sec:optional_kargs} after
compute nodes have been registered with the provisioning system.
\fi

\iftoggleverb{isxCAT}
\noindent Note that additional options are typically necessary to
enable serial console output. These are setup during the node registration
process in \S\ref{sec:xcat_add_nodes} 
\fi




\clearpage
\paragraph{Enable forwarding of system logs} \label{sec:add_syslog}
It is often desirable to consolidate system logging information for the cluster in a
central location, both to provide easy access to the data, and to reduce the
impact of storing data inside the stateless compute node's memory footprint. The
following commands highlight the steps necessary to configure compute nodes to
forward their logs to the SMS, and to allow the SMS to accept these log requests.


% begin_ohpc_run
% ohpc_comment_header Configure rsyslog on SMS and computes \ref{sec:add_syslog}
\begin{lstlisting}[language=bash,keywords={}]
# Configure SMS to receive messages and reload rsyslog configuration
[sms](*\#*) echo 'module(load="imudp")' >> /etc/rsyslog.d/ohpc.conf
[sms](*\#*) echo 'input(type="imudp" port="514")' >> /etc/rsyslog.d/ohpc.conf
[sms](*\#*) systemctl restart rsyslog

# Define compute node forwarding destination
[sms](*\#*) echo "*.* @${sms_ip}:514" >> $CHROOT/etc/rsyslog.conf
[sms](*\#*) echo "Target=\"${sms_ip}\" Protocol=\"udp\"" >> $CHROOT/etc/rsyslog.conf

# Disable most local logging on computes. Emergency and boot logs will remain on the compute nodes
[sms](*\#*) perl -pi -e "s/^\*\.info/\\#\*\.info/" $CHROOT/etc/rsyslog.conf
[sms](*\#*) perl -pi -e "s/^authpriv/\\#authpriv/" $CHROOT/etc/rsyslog.conf
[sms](*\#*) perl -pi -e "s/^mail/\\#mail/" $CHROOT/etc/rsyslog.conf
[sms](*\#*) perl -pi -e "s/^cron/\\#cron/" $CHROOT/etc/rsyslog.conf
[sms](*\#*) perl -pi -e "s/^uucp/\\#uucp/" $CHROOT/etc/rsyslog.conf

\end{lstlisting}
% end_ohpc_run


\subsubsection{Import files} \label{sec:file_import}
The \Warewulf{} system includes functionality to import arbitrary files from
the provisioning server for distribution to managed hosts. This is one way to
distribute user credentials to {\em compute} nodes. To import local file-based
credentials, issue the following:

% begin_ohpc_run
% ohpc_comment_header Import files \ref{sec:file_import}
\begin{lstlisting}[language=bash,literate={-}{-}1,keywords={},upquote=true]
[sms](*\#*) wwsh file import /etc/passwd
[sms](*\#*) wwsh file import /etc/group
[sms](*\#*) wwsh file import /etc/shadow

\end{lstlisting}
% \end_ohpc_run


%----------------
% CentOS specific
%----------------

% begin_ohpc_run
% ohpc_validation_newline
% ohpc_command if [[ ${enable_ipoib} -eq 1 ]];then
% ohpc_indent 5
\begin{lstlisting}[language=bash,literate={-}{-}1,keywords={},upquote=true]
[sms](*\#*) wwsh file import /opt/ohpc/pub/examples/network/centos/ifcfg-ib0.ww
[sms](*\#*) wwsh -y file set ifcfg-ib0.ww --path=/etc/sysconfig/network-scripts/ifcfg-ib0
\end{lstlisting}
% ohpc_indent 0
% ohpc_command fi
% \end_ohpc_run

\subsection{Finalizing provisioning configuration} \label{sec:assemble_bootstrap}

\Warewulf{} employs a two-stage boot process for provisioning nodes via
creation of a bootstrap image that is used to initialize the process, and a virtual node
file system capsule containing the full system image. This section highlights
creation of the necessary provisioning images, followed by the registration of
desired compute nodes.

\subsubsection{Assemble bootstrap image}

The bootstrap image includes the runtime kernel and associated modules, as well
as some simple scripts to complete the provisioning process. The
following commands highlight the inclusion of additional drivers and creation
of the bootstrap image based on the running kernel.

%\iftoggle{isCentOS_ww_slurm_aarch}{\clearpage}

% begin_ohpc_run
% ohpc_comment_header Assemble bootstrap image \ref{sec:assemble_bootstrap}
\begin{lstlisting}[language=bash,literate={-}{-}1,keywords={},upquote=true]
# (Optional) Include drivers from kernel updates;  needed if enabling additional kernel modules on computes
[sms](*\#*) export WW_CONF=/etc/warewulf/bootstrap.conf
[sms](*\#*) echo "drivers += updates/kernel/" >> $WW_CONF

# Build bootstrap image
[sms](*\#*) wwbootstrap `uname -r`
\end{lstlisting}
% end_ohpc_run

\subsubsection{Assemble Virtual Node File System (VNFS) image}

With the local site customizations in place, the following step uses the
\texttt{wwvnfs} command to assemble a VNFS capsule from the chroot environment
defined for the {\em compute} instance.

% begin_ohpc_run
% ohpc_validation_comment Assemble VNFS
\begin{lstlisting}[language=bash,literate={-}{-}1,keywords={},upquote=true]
[sms](*\#*) wwvnfs --chroot $CHROOT
\end{lstlisting}
% end_ohpc_run

\iftoggle{isCentOS_ww_slurm_aarch}{\vspace*{0.4cm}}

\iftoggle{isSLES_ww_slurm_aarch}{\vspace*{-0.1cm}}

\subsubsection{Register nodes for provisioning}

In preparation for provisioning, we can now define the desired network settings
for four example compute nodes with the underlying provisioning system and
restart the \texttt{dhcp} service. Note the use of variable names for the
desired compute hostnames, node IPs, and MAC addresses which should be modified
to accommodate local settings and hardware.  By default, \Warewulf{} uses
network interface names of the \texttt{eth\#} variety and adds kernel boot
arguments to maintain this scheme on newer kernels. Consequently, when specifying
the desired provisioning interface via the \texttt{\$eth\_provision} variable,
it should follow this convention. Alternatively, if you prefer to use the
predictable network interface naming scheme (e.g. names like \texttt{en4s0f0}),
additional steps are included to alter the default kernel boot arguments and take
the \texttt{eth\#} named interface down after bootstrapping so the normal init
process can bring it up again using the desired name.

\iftoggleverb{isx86}
Also included in these steps are commands
to enable \Warewulf{} to manage IPoIB settings and corresponding definitions of
IPoIB addresses for the compute nodes. This is typically optional unless you
are planning to include a \Lustre{} client mount over \InfiniBand{}.
\fi
The final step
in this process associates the VNFS image assembled in previous steps with the
newly defined compute nodes, utilizing the user credential files and munge key
that were imported in \S\ref{sec:file_import}.




%\clearpage
% begin_ohpc_run
% ohpc_validation_comment Add hosts to cluster

\begin{lstlisting}[language=bash,keywords={},upquote=true,basicstyle=\footnotesize\ttfamily,]
# Set provisioning interface as the default networking device
[sms](*\#*) echo "GATEWAYDEV=${eth_provision}" > /tmp/network.$$
[sms](*\#*) wwsh -y file import /tmp/network.$$ --name network
[sms](*\#*) wwsh -y file set network --path /etc/sysconfig/network --mode=0644 --uid=0

# Add nodes to Warewulf data store
[sms](*\#*) for ((i=0; i<$num_computes; i++)) ; do
                wwsh -y node new ${c_name[i]} --ipaddr=${c_ip[i]} --hwaddr=${c_mac[i]} -D ${eth_provision}
        done
\end{lstlisting}
% end_ohpc_run

% begin_ohpc_run
% ohpc_validation_comment Add hosts to cluster (Cont.)
\begin{lstlisting}[language=bash,keywords={},upquote=true,basicstyle=\footnotesize\ttfamily,literate={BOSVER}{\baseos{}}1]
# Define provisioning image for hosts
[sms](*\#*) wwsh -y provision set "${compute_regex}" --vnfs=centos7.2 --bootstrap=`uname -r` \
    --files=dynamic_hosts,passwd,group,shadow,slurm.conf,munge.key,network 
\end{lstlisting}

% ohpc_validation_newline
% ohpc_validation_comment Optionally, add arguments to bootstrap kernel
% ohpc_command if [[ ${enable_kargs} ]]; then
% ohpc_command    wwsh provision set "${compute_regex}" --kargs=${kargs}
% ohpc_command fi

% ohpc_validation_newline
% ohpc_validation_comment Restart ganglia services to pick up hostfile changes
% ohpc_command if [[ ${enable_ganglia} -eq 1 ]];then
% ohpc_command   systemctl restart gmond
% ohpc_command   systemctl restart gmetad
% ohpc_command fi

% ohpc_validation_newline
% ohpc_validation_comment Optionally, define IPoIB network settings (required if planning to mount Lustre over IB)
% ohpc_command if [[ ${enable_ipoib} -eq 1 ]];then
% ohpc_indent 5
\begin{lstlisting}[language=bash,keywords={},upquote=true,basicstyle=\footnotesize\ttfamily]
# Optionally define IPoIB network settings (required if planning to mount Lustre over IB)
[sms](*\#*) for ((i=0; i<$num_computes; i++)) ; do
              wwsh -y node set ${c_name[$i]} -D ib0 --ipaddr=${c_ipoib[$i]} --netmask=${ipoib_netmask}
        done
[sms](*\#*) wwsh -y provision set "${compute_regex}" --fileadd=ifcfg-ib0.ww
\end{lstlisting}
% ohpc_indent 0
% ohpc_command fi
% ohpc_validation_newline
% end_ohpc_run

\begin{center}
\begin{tcolorbox}[]
\small \Warewulf{} includes a utility named \texttt{wwnodescan}
to automatically register new compute nodes versus the outlined node-addition approach
which requires hardware MAC addresses to be gathered in advance.  With
\texttt{wwnodescan}, nodes will be added to the \Warewulf{} database in the
order in which their DHCP requests are received by the master, so care must be
taken to boot nodes in the order one wishes to see preserved in the Warewulf
database. The IP address provided will be incremented after each node is found,
and the utility will exit after all specified nodes have been found. Example
usage is highlighted below:
\begin{lstlisting}[language=bash,keywords={},upquote=true,basicstyle=\footnotesize\ttfamily,literate={BOSVER}{\baseos{}}1]
[sms](*\#*) wwnodescan --netdev=${eth_provision} --ipaddr=${c_ip[0]} --netmask=${internal_netmask} \
    --vnfs=BOSVER --bootstrap=`uname -r` --listen=${sms_eth_internal} ${c_name[0]}-${c_name[3]}
\end{lstlisting}
\end{tcolorbox}
\end{center}


% begin_ohpc_run
\begin{lstlisting}[language=bash,keywords={},upquote=true,basicstyle=\footnotesize\ttfamily]
# Restart dhcp / update PXE
[sms](*\#*) systemctl restart dhcpd
[sms](*\#*) wwsh pxe update
\end{lstlisting}
% end_ohpc_run

\subsubsection{Optional kernel arguments} \label{sec:optional_kargs}
\input{../../common/conman_post}

\subsubsection{Optionally configure stateful provisioning}
Warewulf normally defaults to running the assembled VNFS image out of system
memory in a {\em stateless} configuration. Alternatively, Warewulf can also be
used to partition and format persistent storage such that the VNFS image can be
installed locally to disk in a {\em stateful} manner.  This does, however,
require that a boot loader (GRUB) be added to the image as follows:

\begin{lstlisting}[language=bash,literate={-}{-}1,keywords={},upquote=true]
# Add GRUB2 bootloader and re-assemble VNFS image
[sms](*\#*) (*\chrootinstall*) grub2
[sms](*\#*) wwvnfs  --chroot $CHROOT
\end{lstlisting}

\noindent Enabling stateful nodes also requires additional site-specific, disk-related
parameters in the Warewulf configuration. In the example that follows, the
compute node configuration is updated to define where to install the GRUB
bootloader, which disk to partition, which partition to format, and what the
filesystem layout will look like.

\begin{lstlisting}[language=bash,literate={-}{-}1,keywords={},upquote=true]
# Update node object parameters
[sms](*\#*) export sda1="mountpoint=/boot:dev=sda1:type=ext4:size=1000"
[sms](*\#*) export sda2="dev=sda2:type=swap:size=32768"
[sms](*\#*) export sda3="mountpoint=/:dev=sda3:type=ext4:size=fill"
[sms](*\#*) wwsh -y object modify -s bootloader=sda -t node "${compute_regex}" 
[sms](*\#*) wwsh -y object modify -s diskpartition=sda -t node "${compute_regex}" 
[sms](*\#*) wwsh -y object modify -s diskformat=sda1,sda2,sda3 -t node "${compute_regex}" 
[sms](*\#*) wwsh -y object modify -s filesystems="$sda1,$sda2,$sda3" -t node "${compute_regex}" 
\end{lstlisting}

\noindent Upon subsequent reboot of the modified nodes, Warewulf will partition
and format the disk to host the desired VNFS image.  Once installed to disk,
Warewulf can be instructed to subsequently boot from local storage
(alternatively, the BIOS boot option order could be updated to reflect a desire
to boot from disk):

\begin{lstlisting}[language=bash,literate={-}{-}1,keywords={},upquote=true]
# After provisioning, update node object parameters to boot from local storage
[sms](*\#*) wwsh -y object modify -s bootlocal=EXIT -t node "${compute_regex}"
\end{lstlisting}


\noindent Deleting the bootlocal object parameter will cause Warewulf to once
again reformat and re-install to local storage upon a new PXE boot request.


\vspace*{-0.1cm}
\subsection{Boot compute nodes} \label{sec:boot_computes}
At this point, the {\em master} server should be able to boot the newly defined
compute nodes. Assuming that the compute node BIOS settings are configured to
boot over PXE, all that is required to initiate the provisioning process is to
power cycle each of the desired hosts using IPMI access.
The following commands use the \texttt{ipmitool} utility to initiate power
resets on each of the four compute hosts. Note that the utility requires that
the \texttt{IPMI\_PASSWORD} environment variable be set with the local BMC password in
order to work interactively.

% begin_ohpc_run
% ohpc_comment_header Boot compute nodes \ref{sec:boot_computes}
\begin{lstlisting}[language=bash,keywords={},upquote=true]
[sms](*\#*) for ((i=0; i<${num_computes}; i++)) ; do
             ipmitool -E -I lanplus -H ${c_bmc[$i]} -U ${bmc_username} -P ${bmc_password} chassis power reset
        done
\end{lstlisting} 
% end_ohpc_run

Once kicked off, the boot process should take less than 5 minutes (depending on
BIOS post times) and you can verify that the compute hosts are available via
ssh, or via parallel ssh tools to multiple hosts. For example, to run a command
on the newly imaged compute hosts using \texttt{pdsh}, execute the following:

%\iftoggle{isCentOS_ww_pbs_x86}{\clearpage}
%\iftoggle{isCentOS}{\clearpage}
  
\begin{lstlisting}[language=bash]
[sms](*\#*) pdsh -w c[1-4] uptime
c1  05:03am  up   0:02,  0 users,  load average: 0.20, 0.13, 0.05
c2  05:03am  up   0:02,  0 users,  load average: 0.20, 0.14, 0.06
c3  05:03am  up   0:02,  0 users,  load average: 0.19, 0.15, 0.06
c4  05:03am  up   0:02,  0 users,  load average: 0.15, 0.12, 0.05
\end{lstlisting}

\iftoggleverb{isWarewulf}
\begin{center}
\begin{tcolorbox}[]
\small While the \texttt{pxelinux.0} and \texttt{lpxelinux.0} files that ship
with Warewulf to enable network boot support a wide range of hardware, some
hosts may boot more reliably or faster using the BOS versions provided via the
\texttt{\tftppkg{}} package. If you encounter PXE issues, consider
replacing the \texttt{pxelinux.0} and \texttt{lpxelinux.0} files supplied with
\texttt{warewulf-provision-ohpc} with versions from \texttt{\tftppkg{}}.
\end{tcolorbox}
\end{center}
\fi

 

%----------------
% CentOS specific
%----------------

\begin{center}
\begin{tcolorbox}[]
\small While the \texttt{pxelinux.0} and \texttt{lpxelinux.0} files that ship
with Warewulf to enable network boot support a wide range of hardware, some
hosts may boot more reliably or faster using the BOS versions provided via the
\texttt{syslinux-tftpboot} package. If you encounter PXE issues, consider
replacing the \texttt{pxelinux.0} and \texttt{lpxelinux.0} files supplied with
\texttt{warewulf-provision-ohpc} with versions from \texttt{syslinux-tftpboot}.
\end{tcolorbox}
\end{center}

\vspace*{-0.50cm}
\section{Install \OHPC{} Development Components}
The install procedure outlined in \S\ref{sec:basic_install} highlighted the
steps necessary to install a {\em master} host,\nottoggle{isstateful}{ assemble and customize a{\em
  compute} image, and}{} provision several compute hosts from bare-metal\iftoggle{isstateful}{ and customize their install}{}. With
these steps completed, additional \OHPC{}-provided packages can now be added to
support a flexible HPC development environment including development tools,
C/C++/Fortran compilers, MPI stacks, and a variety of 3rd party libraries. The
following subsections highlight the additional software installation
procedures.



\vspace*{-0.15cm}
\subsection{Development Tools} \label{sec:install_dev_tools}
\input{../../common/dev_tools}

\vspace*{-0.15cm}
\subsection{Compilers} \label{sec:install_compilers}
\OHPC{} presently packages the \GNU{} compiler toolchain integrated with the 
underlying modules-environment system in a hierarchical fashion. The modules
system will conditionally present compiler-dependent software based on the
toolchain currently loaded. 

% begin_ohpc_run
% ohpc_comment_header Install Compilers \ref{sec:install_compilers}
\begin{lstlisting}[language=bash]
[sms](*\#*) (*\install*) gnu7-compilers-ohpc
\end{lstlisting}
% end_ohpc_run

The llvm compiler toolchains are also provided as a standalone additional
compiler family (ie. users can easily switch between gcc/clang environments),
but we do not provide the full complement of downstream library builds.

% begin_ohpc_run
% ohpc_comment_header Install llvm Compilers
\begin{lstlisting}[language=bash]
[sms](*\#*) (*\install*) llvm-compilers-ohpc
\end{lstlisting}
% end_ohpc_run


%\clearpage
\subsection{MPI Stacks} \label{sec:mpi}
For MPI development and runtime support, \OHPC{} provides pre-packaged builds
for a variety of MPI families and transport layers. Currently available options
and their applicability to various network transports are summarized in
Table~\ref{table:mpi}.  The command that follows installs a starting set of MPI
families that are compatible with both ethernet and high-speed fabrics. 

\iftoggleverb{isx86}
% x86_64

\begin{table}[h]
\caption{Available MPI variants} \label{table:mpi}
\centering
\begin{tabular}{@{\hspace*{0.2cm}} *5l @{}}    \toprule
                                  & Ethernet (TCP)                 & \InfiniBand{}                  & \IntelR{} Omni-Path            \\ \midrule
\rowcolor{black!10} MPICH (ofi) & \multicolumn{1}{c}{\checkmark} & \multicolumn{1}{c}{\checkmark} & \multicolumn{1}{c}{\checkmark} \\
 MPICH (ucx)       & \multicolumn{1}{c}{\checkmark} & \multicolumn{1}{c}{\checkmark} & \multicolumn{1}{c}{\checkmark} \\
\rowcolor{black!10} MVAPICH2                          &                                & \multicolumn{1}{c}{\checkmark} &                                \\
MVAPICH2 (psm2) &                              &                                & \multicolumn{1}{c}{\checkmark} \\
\rowcolor{black!10} OpenMPI (ofi/ucx)            & \multicolumn{1}{c}{\checkmark} & \multicolumn{1}{c}{\checkmark} & \multicolumn{1}{c}{\checkmark} \\
%\rowcolor{black!10} OpenMPI (PMIx) & \multicolumn{1}{c}{\checkmark} & \multicolumn{1}{c}{\checkmark} & \multicolumn{1}{c}{\checkmark} \\ \bottomrule
\end{tabular}
\end{table}

\else
% aarch64

\begin{table}[h]
\caption{Available MPI builds} \label{table:mpi}
\centering
\begin{tabular}{@{\hspace*{0.2cm}} *5l @{}}    \toprule
                                  & Ethernet (TCP)                 & \InfiniBand{}                              \\ \midrule
\rowcolor{black!10} MPICH         & \multicolumn{1}{c}{\checkmark} &                                            \\
\rowcolor{black!10} OpenMPI                           & \multicolumn{1}{c}{\checkmark} & \multicolumn{1}{c}{\checkmark} \\
\end{tabular}
\end{table}

\fi

% begin_ohpc_run
% ohpc_comment_header Install MPI Stacks \ref{sec:mpi}
% ohpc_command if [[ ${enable_mpi_defaults} -eq 1 && ${enable_pmix} -eq 0 ]];then
% ohpc_indent 5
\begin{lstlisting}[language=bash]
[sms](*\#*) (*\install*) openmpi4-gnu9-ohpc mpich-ofi-gnu9-ohpc
\end{lstlisting}
% ohpc_indent 0
% ohpc_command elif [[ ${enable_mpi_defaults} -eq 1 && ${enable_pmix} -eq 1 ]];then
% ohpc_indent 5
% ohpc_command (*\install*) openmpi4-pmix-slurm-gnu9-ohpc mpich-ofi-gnu9-ohpc
% ohpc_indent 0
% ohpc_command fi
% end_ohpc_run

Note that OpenHPC 2.x introduces the use of two related transport layers for
the MPICH and OpenMPI builds that support a variety of underlying
fabrics: \href{https://www.openucx.org}{UCX} (Unified Communication X)
and \href{https://ofiwg.github.io/libfabric/}{OFI} (OpenFabrics interfaces).
In the case of OpenMPI, a monolithic build is provided which supports both
transports and end-users can customize their runtime preferences with
environment variables. For MPICH, two separate builds are provided and the
example above highlighted installing the {\texttt ofi} variant.  However, the
packaging is designed such that both versions can be installed simultaneously
and users can switch between the two via normal module command
semantics. Alternatively, a site can choose to install the {\texttt ucx} variant
instead as a drop-in MPICH replacement:

\begin{lstlisting}[language=bash]
[sms](*\#*) (*\install*) mpich-ucx-gnu9-ohpc
\end{lstlisting}

In the case where both MPICH variants are installed, two modules will be
visible in the end-user environment and an example of this configuration is
highlighted is below. 

\begin{lstlisting}[language=bash]
[sms](*\#*) module avail mpich

-------------------- /opt/ohpc/pub/moduledeps/gnu9 ---------------------
   mpich/3.3.2-ofi    mpich/3.3.2-ucx (D)
\end{lstlisting}

If your system includes \InfiniBand{} and you enabled underlying support in
\S\ref{sec:add_ofed} and \S\ref{sec:addl_customizations}, an additional
MVAPICH2 family is available for use:

% begin_ohpc_run
% ohpc_validation_newline
% ohpc_command if [[ ${enable_ib} -eq 1 ]];then
% ohpc_indent 5
\begin{lstlisting}[language=bash]
[sms](*\#*) (*\install*) mvapich2-gnu9-ohpc
\end{lstlisting}
% ohpc_indent 0
% ohpc_command fi
% end_ohpc_run

Alternatively, if your system includes \IntelR{} \OmniPath{}, use the (\texttt{psm2})
variant of MVAPICH2 instead:

% begin_ohpc_run
% ohpc_command if [[ ${enable_opa} -eq 1 ]];then
% ohpc_indent 5
\begin{lstlisting}[language=bash]
[sms](*\#*) (*\install*) mvapich2-psm2-gnu9-ohpc
\end{lstlisting}
% ohpc_indent 0
% ohpc_command fi
% end_ohpc_run

An additional OpenMPI build variant is listed in Table~\ref{table:mpi} which
enables \href{https://pmix.github.io/pmix/}{\color{blue}{PMIx}} job launch
support for use with \SLURM{}. This optional variant is
available as \texttt{openmpi4-pmix-slurm-gnu9-ohpc}.


\subsection{Performance Tools} \label{sec:install_perf_tools}
\OHPC{} provides a variety of open-source tools to aid in application
performance analysis (refer to Appendix~\ref{appendix:manifest} for a listing
of available packages). This group of tools can be installed as follows:

% begin_ohpc_run
% ohpc_comment_header Install Performance Tools \ref{sec:install_perf_tools}
\begin{lstlisting}[language=bash,keywords={},literate={-}{-}1]
# Install perf-tools meta-package
[sms](*\#*) (*\install*) ohpc-gnu13-perf-tools
\end{lstlisting}
% end_ohpc_run


\subsection{Setup default development environment}
System users often find it convenient to have a default development environment
in place so that compilation can be performed directly for parallel programs
requiring MPI. This setup can be conveniently enabled via modules and the \OHPC{}
modules environment is pre-configured to load an \texttt{ohpc} module on login
(if present). The following package install provides a default
environment that enables autotools, the \GNU{} compiler toolchain, and the
OpenMPI stack.

% begin_ohpc_run
\begin{lstlisting}[language=bash]
[sms](*\#*) (*\install*) lmod-defaults-gnu9-openmpi4-ohpc
\end{lstlisting}
% end_ohpc_run

\begin{center}
\begin{tcolorbox}[]
\small
\iftoggleverb{isx86}
If you want to change the default environment from the suggestion above, \OHPC{}
also provides the \GNU{} compiler toolchain with the MPICH and MVAPICH2 stacks:
\fi

\iftoggleverb{isaarch}
If you want to change the default environment from the suggestion above, \OHPC{}
also provides additional default options using the \GNU{} compiler toolchain
with multiple MPICH variants or MVAPICH2. Relevant lmod-default packages names
are as follows:
\fi

\begin{itemize*}
\item lmod-defaults-gnu9-mpich-ofi-ohpc
\item lmod-defaults-gnu9-mpich-ucx-ohpc
\iftoggleverb{isx86}
\item lmod-defaults-gnu9-mvapich2-ohpc
\fi
\end{itemize*}
\end{tcolorbox}
\end{center}


%\vspace*{0.2cm}
\subsection{3rd Party Libraries and Tools} \label{sec:3rdparty}
\input{../../common/third_party_libs_intro}

%----------------
% CentOS specific
%----------------

\begin{lstlisting}[language=bash,keywords={}]
[sms](*\#*) yum search petsc-gnu ohpc
Loaded plugins: fastestmirror
Loading mirror speeds from cached hostfile
=========================== N/S matched: petsc-gnu, ohpc ===========================
petsc-gnu-mvapich2-ohpc.x86_64 : Portable Extensible Toolkit for Scientific Computation
petsc-gnu-openmpi-ohpc.x86_64 : Portable Extensible Toolkit for Scientific Computation
\end{lstlisting}

% begin_ohpc_run
% ohpc_comment_header Install 3rd Party Libraries and Tools \ref{sec:3rdparty}
\begin{lstlisting}[language=bash,keywords={},upquote=true,keepspaces]
[master]$ (*\groupinstall*) fsp-boost        # adds available Boost packages
[master]$ (*\groupinstall*) fsp-fftw         # adds available FFTW packages
[master]$ (*\groupinstall*) fsp-gsl          # adds available GSL packages
[master]$ (*\groupinstall*) fsp-hdf5         # adds available HDF5 packages
[master]$ (*\groupinstall*) fsp-hypre        # adds available Hypre packages
[master]$ (*\groupinstall*) fsp-metis        # adds available METIS packages
[master]$ (*\groupinstall*) fsp-mpiP         # adds available mpiP packages
[master]$ (*\groupinstall*) fsp-mumps        # adds available MUMPS packages
[master]$ (*\groupinstall*) fsp-netcdf       # adds available NetCDF packages
[master]$ (*\groupinstall*) fsp-numpy        # adds available numerical Python packages
[master]$ (*\groupinstall*) fsp-petsc        # adds available PETSC packages
[master]$ (*\groupinstall*) fsp-phdf5        # adds available (parallel) HDF5 packages
[master]$ (*\groupinstall*) fsp-scalapack    # adds available ScaLAPACK packages
[master]$ (*\groupinstall*) fsp-scipy        # adds available scientific Python packages
[master]$ (*\groupinstall*) fsp-trilinos     # adds available Trilinos packages
[master]$ (*\groupinstall*) fsp-adios        # adds available Adios packages
[master]$ (*\install*) R_base-fsp            # adds R
\end{lstlisting}
% end_ohpc_run


\subsection{Optional Development Tool Builds} \label{sec:3rdparty_intel}
In addition to the 3rd party development libraries built using the open source
toolchains mentioned in \S\ref{sec:3rdparty}, \OHPC{} also provides {\em
  optional} compatible builds for use with the compilers and MPI stack included
in newer versions of the \IntelR{}~Parallel Studio XE software suite.  These
packages provide a similar hierarchical user
environment experience as other compiler and MPI families present in \OHPC{}.

To take advantage of the available builds, the Parallel Studio software suite
must be obtained and installed separately. Once installed locally, the \OHPC{}
compatible packages can be installed using standard package manager semantics.
Note that licenses are provided free of charge for many categories of use. In
particular, licenses for compilers and developments tools are provided at no
cost to academic researchers or developers contributing to open-source software
projects. More information on this program can be found at:

\begin{center}
  \href{https://software.intel.com/en-us/qualify-for-free-software}
       {\color{blue}{https://software.intel.com/en-us/qualify-for-free-software}}
\end{center}

\begin{center}
\begin{tcolorbox}[]
As noted in \S\ref{sec:master_customization}, the default installation path for
\OHPC{} (\texttt{/opt/ohpc/pub}) is exported over NFS from the {\em master} to the 
compute nodes, but the Parallel Studio installer defaults to a path of 
\texttt{/opt/intel}. To make the \IntelR{} compilers available to the compute 
nodes one must either customize the Parallel Studio installation path to be 
within \texttt{/opt/ohpc/pub}, or alternatively, add an additional NFS export
for \texttt{/opt/intel} that is mounted on desired compute nodes.
\end{tcolorbox}
\end{center}

\noindent To enable all 3rd party builds available in \OHPC{} that are compatible with
\IntelR{}~Parallel Studio, issue the following:

% begin_ohpc_run
% ohpc_comment_header Install Optional Development Tools for use with Intel Parallel Studio \ref{sec:3rdparty_intel}
% ohpc_command if [[ ${enable_intel_packages} -eq 1 ]];then
% ohpc_indent 5
\begin{lstlisting}[language=bash,keywords={},upquote=true,keepspaces]
# Install OpenHPC compatibility packages (requires prior installation of Parallel Studio)
[sms](*\#*) (*\install*) intel-compilers-devel-ohpc
[sms](*\#*) (*\install*) intel-mpi-devel-ohpc
\end{lstlisting}

% ohpc_command if [[ ${enable_opa} -eq 1 ]];then
% ohpc_indent 10
\begin{lstlisting}[language=bash,keywords={},upquote=true,keepspaces]
<<<<<<< .merge_file_CHumzE
# Optionally, choose Omni-Path enabled MPI builds. Otherwise, skip to retain default MPI stacks
[sms](*\#*) (*\install*) openmpi-psm2-intel-ohpc mvapich2-psm2-intel-ohpc
=======
# Optionally, choose the Omni-Path enabled build for MVAPICH2. Otherwise, skip to retain IB variant
[sms](*\#*) (*\install*) mvapich2-psm2-intel-ohpc
>>>>>>> .merge_file_juf65L
\end{lstlisting}
% ohpc_indent 5
% ohpc_command fi

\begin{lstlisting}[language=bash,keywords={},upquote=true,keepspaces]
# Install 3rd party libraries/tools meta-packages built with Intel toolchain
[sms](*\#*) (*\install*) ohpc-intel-serial-libs
[sms](*\#*) (*\install*) ohpc-intel-io-libs
[sms](*\#*) (*\install*) ohpc-intel-perf-tools
[sms](*\#*) (*\install*) ohpc-intel-python-libs
[sms](*\#*) (*\install*) ohpc-intel-runtimes
[sms](*\#*) (*\install*) ohpc-intel-mpich-parallel-libs
[sms](*\#*) (*\install*) ohpc-intel-mvapich2-parallel-libs
[sms](*\#*) (*\install*) ohpc-intel-openmpi3-parallel-libs
[sms](*\#*) (*\install*) ohpc-intel-impi-parallel-libs
\end{lstlisting}
% ohpc_indent 0
% ohpc_command fi
% end_ohpc_run



\section{Resource Manager Startup} \label{sec:rms_startup}
In section \S\ref{sec:basic_install}, the \SLURM{} resource manager was installed
and configured for use on both the {\em master} host and {\em compute} node
instances. With the cluster nodes up and functional, we can now startup the
resource manager services in preparation for running user jobs. Generally, this
is a two-step process that requires starting up the controller daemons on the {\em
 master} host and the client daemons on each of the {\em compute} hosts.
%Since the {\em compute} hosts were booted into an image that included the SLURM client
%components, the daemons should already be running on the {\em compute}
%hosts.
Note that \SLURM{} leverages the use of the {\em munge} library to provide
authentication services and this daemon also needs to be running on all hosts
within the resource management pool.
%The munge daemons should already
%be running on the {\em compute} subsystem at this point,
The following commands can be used to startup the necessary services to support
resource management under \SLURM{}.

%\iftoggle{isCentOS}{\clearpage}

% Allow for optional sleep to wait for nodes to provision when using install
% script


% begin_ohpc_run
% ohpc_comment_header Allow for optional sleep to wait for provisioning to complete
% ohpc_command sleep ${provision_wait}
% end_ohpc_run

% begin_ohpc_run
% ohpc_comment_header Resource Manager Startup \ref{sec:rms_startup}
\begin{lstlisting}[language=bash,keywords={}]
# Start munge and slurm controller on master host
[sms](*\#*) systemctl enable munge
[sms](*\#*) systemctl enable slurmctld
[sms](*\#*) systemctl start munge
[sms](*\#*) systemctl start slurmctld

# Start slurm clients on compute hosts
[sms](*\#*) pdsh -w ${compute_prefix}[1-${num_computes}] systemctl start munge
[sms](*\#*) pdsh -w ${compute_prefix}[1-${num_computes}] systemctl start slurmd
\end{lstlisting}
% end_ohpc_run

%%% In the default configuration, the {\em compute} hosts will be initialized in an
%%% {\em unknown} state. To place the hosts into production such that they are
%%% eligible to schedule user jobs, issue the following:

%%% % begin_ohpc_run
%%% \begin{lstlisting}[language=bash]
%%% [sms](*\#*) scontrol update partition=normal state=idle
%%% \end{lstlisting}
%%% % end_ohpc_run



\section{Run a Test Job} \label{sec:test_job}
With the resource manager enabled for production usage, users should now be
able to run jobs. To demonstrate this, we will add a ``test'' user on the {\em master}
host that can be used to run an example job.

% begin_ohpc_run
\begin{lstlisting}[language=bash,keywords={}]
[sms](*\#*) useradd -m test
\end{lstlisting}
% end_ohpc_run

\Warewulf{} installs a utility on the compute nodes to automatically
synchronize known files from the provisioning server at five minute intervals. In this
recipe, recall that we previously registered credential files with Warewulf (e.g. passwd,
group, and shadow) so that these files would be propagated during compute node
imaging. However, with the addition of a new ``test'' user above, the files
have been outdated and we need to update the Warewulf database to incorporate
the additions. This re-sync process can be accomplished as follows:

% begin_ohpc_run
\begin{lstlisting}[language=bash,keywords={}]
[sms](*\#*) wwsh file resync passwd shadow group
\end{lstlisting}
% end_ohpc_run

% begin_ohpc_run
% ohpc_command sleep 2
% end_ohpc_run

\begin{center}
\begin{tcolorbox}[]
\small
After re-syncing to notify Warewulf of file modifications made on the {\em
master} host, it should take approximately 5 minutes for the changes to
propagate. However, you can also manually pull the changes from compute nodes
via the following:
% begin_ohpc_run
\begin{lstlisting}[language=bash,keywords={}]
[sms](*\#*) pdsh -w ${compute_prefix}[1-${num_computes}] /warewulf/bin/wwgetfiles
\end{lstlisting}
% end_ohpc_run
\end{tcolorbox}
\end{center}

\input{common/prun}

%\iftoggle{isSLES_ww_slurm_x86}{\clearpage}
%\iftoggle{isCentOS_ww_slurm_x86}{\clearpage}


\subsection{Interactive execution}
To use the newly created ``test'' account to compile and execute the
application {\em interactively} through the resource manager, execute the
following (note the use of \texttt{prun} for parallel job launch which summarizes
the underlying native job launch mechanism being used):

\begin{lstlisting}[language=bash,keywords={}]
# Switch to "test" user
[sms](*\#*) su - test

# Compile MPI "hello world" example
[test@sms ~]$ mpicc -O3 /opt/ohpc/pub/examples/mpi/hello.c

# Submit interactive job request and use prun to launch executable
[test@sms ~]$ salloc -n 8 -N 2

[test@c1 ~]$ prun ./a.out

[prun] Master compute host = c1
[prun] Resource manager = slurm
[prun] Launch cmd = mpiexec.hydra -bootstrap slurm ./a.out

 Hello, world (8 procs total)
    --> Process #   0 of   8 is alive. -> c1
    --> Process #   4 of   8 is alive. -> c2
    --> Process #   1 of   8 is alive. -> c1
    --> Process #   5 of   8 is alive. -> c2
    --> Process #   2 of   8 is alive. -> c1
    --> Process #   6 of   8 is alive. -> c2
    --> Process #   3 of   8 is alive. -> c1
    --> Process #   7 of   8 is alive. -> c2
\end{lstlisting}

\begin{center}
\begin{tcolorbox}[]
The following table provides approximate command equivalences between SLURM and
OpenPBS:

\vspace*{0.15cm}
\input common/rms_equivalence_table
\end{tcolorbox}
\end{center}
\nottoggle{isCentOS}{\clearpage}

\iftoggle{isCentOS}{\clearpage}

\subsection{Batch execution}

For batch execution, \OHPC{} provides a simple job script for reference (also
housed in the \path{/opt/ohpc/pub/examples} directory. This example script can
be used as a starting point for submitting batch jobs to the resource manager
and the example below illustrates use of the script to submit a batch job for
execution using the same executable referenced in the previous interactive example.

\begin{lstlisting}[language=bash,keywords={}]
# Copy example job script
[test@sms ~]$ cp /opt/ohpc/pub/examples/slurm/job.mpi .

# Examine contents (and edit to set desired job sizing characteristics)
[test@sms ~]$ cat job.mpi
#!/bin/bash

#SBATCH -J test               # Job name
#SBATCH -o job.%j.out         # Name of stdout output file (%j expands to %jobId)
#SBATCH -N 2                  # Total number of nodes requested
#SBATCH -n 16                 # Total number of mpi tasks #requested
#SBATCH -t 01:30:00           # Run time (hh:mm:ss) - 1.5 hours

# Launch MPI-based executable

prun ./a.out

# Submit job for batch execution
[test@sms ~]$ sbatch job.mpi
Submitted batch job 339
\end{lstlisting}

\begin{center}
\begin{tcolorbox}[]
\small
The use of the \texttt{\%j} option in the example batch job script shown is a convenient
way to track application output on an individual job basis. The \texttt{\%j} token
is replaced with the \SLURM{} job allocation number once assigned (job~\#339 in
this example).
\end{tcolorbox}
\end{center}




\clearpage
\appendix
%\section*{Appendices}
{\bf \LARGE \centerline{Appendices}} \vspace*{0.2cm}

\addcontentsline{toc}{section}{Appendices}
\renewcommand{\thesubsection}{\Alph{subsection}}

\subsection{Installation Template}  \label{appendix:template_script}

This appendix highlights the availability of a companion installation script
that is included with \OHPC{} documentation. This script, when combined with
local site inputs, can be used to implement a starting recipe for
bare-metal system installation and configuration. This template script is used
during validation efforts to test cluster installations and is provided as a
convenience for administrators as a starting point for potential site
customization.

\vspace*{0.1cm}

\begin{center}
\begin{tcolorbox}[]
\small Note that the template script provided is intended for use during
initial installation and is not designed for repeated execution.  If
modifications are required after using the script initially, we recommend
running the relevant subset of commands interactively.
\end{tcolorbox}
\end{center}

The template script relies on the use of a simple text file to define local
site variables that were outlined in \S\ref{sec:inputs}. By default, the
template installation script attempts to use local variable settings sourced from
the \path{/opt/ohpc/pub/doc/recipes/vanilla/input.local} file, however, this
choice can be overridden by the use of the \texttt{\$\{OHPC\_INPUT\_LOCAL\}}
environment variable. The template install script is intended for execution on
the SMS {\em master} host and is installed as part of the \texttt{docs-ohpc}
package into \path{/opt/ohpc/pub/doc/recipes/vanilla/recipe.sh}. After
enabling the \OHPC{} repository and reviewing the guide for additional
information on the intent of the commands, the general starting approach for
using this template is as follows:

\begin{enumerate}
\item Install the \texttt{docs-ohpc} package

\begin{lstlisting}[language=bash,keywords={}]
[sms](*\#*) (*\install*) docs-ohpc
\end{lstlisting}

\item Copy the provided template input file to use as a starting point to
  define local site settings:
\begin{lstlisting}[language=bash,keywords={},literate={OSVER}{\baseosshort{}}1]
[sms](*\#*) cp /opt/ohpc/pub/doc/recipes/OSVER/input.local input.local
\end{lstlisting}

\item Update \path{input.local} with desired settings

\item Copy the template installation script which contains command-line
  instructions culled from this guide.

\begin{lstlisting}[language=bash,keywords={},basicstyle=\fontencoding{T1}\footnotesize\ttfamily,literate={OSVER}{\baseosshort{}}1
    {ARCH}{\arch{}}1 {PROV}{\MakeLowercase{\provisioner{}}}1
    {RMS}{\rmsshort{}}1 {-}{-}1]
[sms](*\#*) cp -p /opt/ohpc/pub/doc/recipes/OSVER/ARCH/PROV/RMS/recipe.sh .
\end{lstlisting}

\item Review and edit \path{recipe.sh} to suite.

\item Use environment variable to define local input file and execute
  \path{recipe.sh} to perform a local installation.

\begin{lstlisting}[language=bash,keywords={}]
[sms](*\#*) export OHPC_INPUT_LOCAL=./input.local
[sms](*\#*) ./recipe.sh
\end{lstlisting}
\end{enumerate}

\input{../../common/rpmbuild_appendix}
\clearpage

\definecolor{Gray}{gray}{0.5}
\newcommand{\captionSpace}{-0.15cm}
\newcommand{\tabSpaceBot}{1.0cm}
\captionsetup{justification=raggedright,singlelinecheck=false}

\subsection{Package Manifest} \label {appendix:manifest}

\vspace*{0.25cm}
This appendix provides a summary of available meta-package groupings and all of
the individual RPM packages that are available as part of this \OHPC{}
release. The meta-packages provide a mechanism to group related collections of
RPMs by functionality and provide a convenience mechanism for installation.  A
list of the available meta-packages and a brief description is presented in
Table~\ref{table:groups}.

\vspace*{1.25cm}
\begin{table}[h!]
\caption{\bf Available \OHPC{} Meta-packages} \vspace*{\captionSpace{}}
\label{table:groups}
\input manifest/patterns
\end{table}

%% % meta-packages (2)
%% \begin{table}[h!]
%% \caption*{Table~\ref{table:groups} (cont): {\bf Available \OHPC{} Meta-packages} \vspace*{\captionSpace{}} }
%% \input manifest/patterns2
%% \end{table}

\iftoggleverb{isx86}
% meta-packages (3)
\begin{table}[h!]
\caption*{Table~\ref{table:groups} (cont): {\bf Available \OHPC{} Meta-packages} \vspace*{\captionSpace{}} }
\input manifest/patterns3
\end{table}

\fi

\clearpage
What follows next in this Appendix is a series of tables that summarize the
underlying RPM packages available in this \OHPC{} release. These packages are
organized by groupings based on their general functionality and each table
provides information for the specific RPM name, version, brief summary, and the
web URL where additional information can be obtained for the component. Note
that many of the 3rd party community libraries that are pre-packaged
with \OHPC{} are built using multiple compiler and MPI families. In these cases,
the RPM package name includes delimiters identifying the development
environment for which each package build is targeted.  Additional information
on the \OHPC{} package naming scheme is presented in \S\ref{sec:3rdparty}.
The relevant package groupings and associated Table references are as follows:

\vspace*{0.1cm}

\begin{itemize*}
\item Administrative tools (Table~\ref{table:admin})
\iftoggleverb{isWarewulf}
\item Provisioning (Table~\ref{table:provisioning})
\fi
\item Resource management (Table~\ref{table:rms})
\item Compiler families (Table~\ref{table:compiler-families})
\item MPI families (Table~\ref{table:mpi-families})
\item Development tools (Table~\ref{table:dev-tools})
\item Performance analysis tools (Table~\ref{table:perf-tools})
\iftoggleverb{isCentOS_x86}
\item Lustre (Table~\ref{table:lustre})
\fi

%\item Distro support packages and dependencies (Table~\ref{table:distro-packages})
\item IO Libraries (Table~\ref{table:io-libs})
\item Runtimes (Table~\ref{table:runtimes})
\item Serial/Threaded Libraries (Table~\ref{table:serial-libs})
\item Parallel Libraries (Table~\ref{table:parallel-libs})
\end{itemize*}


\newcommand{\firstColWidth}{5.0cm}
\newcommand{\secondColWidth}{1.25cm}

\vspace*{1.0cm}
\urlstyle{same}

% Administration Tools 
\begin{table}[h]
\caption{\bf Administrative Tools} \vspace*{\captionSpace{}} \label{table:admin}
\input manifest/admin
\end{table}
\vspace*{0.5cm}

\renewcommand{\firstColWidth}{6.25cm}
\renewcommand{\secondColWidth}{0.95cm}

% Provisioning
\begin{table}[h!]
\caption{\bf Provisioning} \vspace*{\captionSpace{}} \label{table:provisioning}
\input manifest/provisioning
\vspace*{\tabSpaceBot{}}
\end{table}

\renewcommand{\firstColWidth}{4.1cm}
\renewcommand{\secondColWidth}{1.8cm}

% Resource Management
\begin{table}[h!]
\caption{\bf Resource Management} \vspace*{\captionSpace{}} \label{table:rms}
\input manifest/rms
\vspace*{\tabSpaceBot{}}
\end{table}

\renewcommand{\firstColWidth}{4.8cm}

% Compiler Families
\begin{table}[h!]
\caption{\bf Compiler Families} \vspace*{\captionSpace{}} \label{table:compiler-families}
\input manifest/compiler-families
\vspace*{\tabSpaceBot{}}
\end{table}

\renewcommand{\firstColWidth}{4.8cm}
%\renewcommand{\secondColWidth}{1.5cm}

% MPI Families
\begin{table}[h!]
\caption{\bf MPI Families} \vspace*{\captionSpace{}} \label{table:mpi-families}
\input manifest/mpi-families
\vspace*{\tabSpaceBot{}}
\end{table}

\renewcommand{\firstColWidth}{5.6cm}
\renewcommand{\secondColWidth}{1.5cm}

% Development Tools
\begin{table}[h!]
\caption{\bf Development Tools} \vspace*{\captionSpace{}} \label{table:dev-tools}
\input manifest/dev-tools
\vspace*{\tabSpaceBot{}}
\end{table}

\renewcommand{\firstColWidth}{4.4cm}
\renewcommand{\secondColWidth}{1.72cm}

% Perf Tools
\begin{table}[h!]
\caption{\bf Performance Analysis Tools} \vspace*{\captionSpace{}} \label{table:perf-tools}
\input manifest/perf-tools
\vspace*{\tabSpaceBot{}}
\end{table}

% Distro Packages
\begin{table}[h!]
\caption{\bf Distro Support Packages/Dependencies} \vspace*{\captionSpace{}} \label{table:distro-packages}
\input manifest/distro-packages
\vspace*{\tabSpaceBot{}}
\end{table}

\renewcommand{\firstColWidth}{5.1cm}
\renewcommand{\secondColWidth}{1.4cm}

%%% % Lustre
%%% \begin{table}[h!]
%%% \caption{\bf Lustre} \vspace*{\captionSpace{}} \label{table:lustre}
%%% \input manifest/lustre
%%% \vspace*{\tabSpaceBot{}}
%%% \end{table}

% IO Libs
\begin{table}[h!]
\caption{\bf IO Libraries} \vspace*{\captionSpace{}} \label{table:io-libs}
\input manifest/io-libs
\vspace*{\tabSpaceBot{}}
\end{table}

%\renewcommand{\firstColWidth}{4.5cm}
%\renewcommand{\secondColWidth}{1.5cm}

% Runtimes
\clearpage
\begin{table}[h!]
\caption{\bf Runtimes} \vspace*{\captionSpace{}} \label{table:runtimes}
\input manifest/runtimes
\vspace*{\tabSpaceBot{}}
\end{table}

% Serial libs
\begin{table}[h!]
\caption{\bf Serial/Threaded Libraries} \vspace*{\captionSpace{}} \label{table:serial-libs}
\input manifest/serial-libs
\vspace*{\tabSpaceBot{}}
\end{table}

% Parallel libs
\begin{table}[h!]
\caption{\bf Parallel Libraries} \vspace*{\captionSpace{}} \label{table:parallel-libs}
\input manifest/parallel-libs
\vspace*{\tabSpaceBot{}}
\end{table}

% Parallel libs (2)
\begin{table}[h!]
\caption*{Table~\ref{table:parallel-libs} (cont): {\bf Parallel Libraries} \vspace*{\captionSpace{}} }
\input manifest/parallel-libs2
\vspace*{\tabSpaceBot{}}
\end{table}

\clearpage
\subsection{Package Signatures}
%\addcontentsline{toc}{section}{Appendix B - Package Signatures}

All of the RPMs provided via the \OHPC{} repository are signed with a GPG
signature. By default, the underlying package managers will verify these signatures during
installation to ensure that packages have not been altered. The RPMs can also
be manually verified and the public signing key fingerprint for the latest
repository is shown below: \\

\texttt{Fingerprint: 5392 744D 3C54 3ED5 7847  65E6 8A30 6019 {\bf DA565C6C}} \\

\noindent The following command can be used to verify an RPM once it
has been downloaded locally by confirming if the package is signed, and if so,
indicating which key was used to sign it. The example below highlights usage
for a local copy of the \texttt{docs-ohpc} package and illustrates how the {\em
key ID} matches the fingerprint shown above.

\begin{lstlisting}[language=bash,keywords={}]
[sms](*\#*) rpm --checksig -v docs-ohpc-*.rpm
docs-ohpc-2.0.0-72.1.ohpc.2.0.x86_64.rpm:
    Header V3 RSA/SHA1 Signature, key ID da565c6c: OK
    Header SHA256 digest: OK
    Header SHA1 digest: OK
    Payload SHA256 digest: OK
    V3 RSA/SHA1 Signature, key ID da565c6c: OK
    MD5 digest: OK

\end{lstlisting}






\end{document}

