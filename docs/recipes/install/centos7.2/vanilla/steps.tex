\documentclass[letterpaper]{article}
\usepackage{../../common/ohpc-doc}
\setcounter{secnumdepth}{5}
\setcounter{tocdepth}{5}

% Include git variables
%%% This file has been generated by the vc bundle for TeX.
%%% Do not edit this file!
%%%
%%% Define Git specific macros.
\gdef\GITHash{ab0141b6f1cd0193472944a1642275fb792a8f77}%
\gdef\GITAbrHash{ab0141b}%
\gdef\GITParentHashes{765539311c92eca445116d4e451fa2464e75246f}%
\gdef\GITAbrParentHashes{7655393}%
\gdef\GITAuthorName{Karl W. Schulz}%
\gdef\GITAuthorEmail{karl.w.schulz@intel.com}%
\gdef\GITAuthorDate{2015-04-17 16:00:11 -0500}%
\gdef\GITCommitterName{Karl W. Schulz}%
\gdef\GITCommitterEmail{karl.w.schulz@intel.com}%
\gdef\GITCommitterDate{2015-04-17 16:00:11 -0500}%
%%% Define generic version control macros.
\gdef\VCRevision{\GITAbrHash}%
\gdef\VCAuthor{\GITAuthorName}%
\gdef\VCDateRAW{2015-04-17}%
\gdef\VCDateISO{2015-04-17}%
\gdef\VCDateTEX{2015/04/17}%
\gdef\VCTime{16:00:11 -0500}%
\gdef\VCModifiedText{\textcolor{red}{with local modifications!}}%
%%% Assume clean working copy.
\gdef\VCModified{0}%
\gdef\VCRevisionMod{\VCRevision}%


% Define Base OS and other local macros
\newcommand{\baseOS}{CentOS7.2}
\newcommand{\OSRepo}{CentOS\_7.2}
\newcommand{\baseos}{centos7.2}

\newcommand{\install}{yum -y install}
\newcommand{\chrootinstall}{yum -y --installroot=\$CHROOT install}
\newcommand{\groupinstall}{yum -y groupinstall}
\newcommand{\groupchrootinstall}{yum -y --installroot=\$CHROOT groupinstall}

% boolean for os-specific formatting
\toggletrue{isCentOS}

\begin{document}
\graphicspath{{../../common/figures/}}
\thispagestyle{empty}

% Title Page

\lhead{ \small {\color{logodarkgrey}\fontfamily{phv}\selectfont { Install Guide} - {\baseOS{} Version} (v\OHPCVersion{}) } \vspace*{0pt} }

{\hspace*{4in} \includegraphics[width=1.7in]{ohpc_logo_blue.pdf}}

\vspace*{2cm}
\noindent {\LARGE \color{logodarkgrey} \fontfamily{phv}\selectfont OpenHPC (v\OHPCVersion{})} \vspace*{0.1cm} \\
\noindent {\LARGE \color{logodarkgrey} \fontfamily{phv}\selectfont Cluster Building Recipes} \\ 

{\color{logoblue}\noindent\rule{6.15in}{1.2pt}} \\ \vspace{0.2cm}

\noindent {\Large \color{logodarkgrey} \fontfamily{phv}\selectfont \baseOS{} Base OS} \vspace{0.2cm}

\noindent {\large \color{logodarkgrey} \fontfamily{phv}\selectfont {\em Base Linux* Edition }}

\vspace*{4in}


%\noindent{\normalsize \color{black} Intel Cluster Makers} \vspace*{0.1cm} \\
%{\normalsize \color{logodarkgrey} Copyright~{\small\copyright}~2014-2015 Intel Corporation} \vspace*{0.1cm} \\ 
\noindent{\small \color{black} Document Last Update: \VCDateISO} \vspace*{0.1cm} \\ 
{\small \color{black} Document Revision: \VCRevision} \\ \vspace*{0.1cm}

% Disclaimer  ----------------------------------------------------
\newpage

\vspace*{5.0cm}
\noindent {\Large \color{RoyalBlue} \fontfamily{phv}\selectfont Disclaimer and Legal Information} \\ 

{\footnotesize

\noindent NO LICENSE (EXPRESS OR IMPLIED, BY ESTOPPEL OR OTHERWISE) TO ANY INTELLECTUAL
PROPERTY RIGHTS IS GRANTED BY THIS DOCUMENT. THIS DOCUMENT MAY CONTAIN
INFORMATION ON PRODUCTS, SERVICES AND/OR PROCESSES IN DEVELOPMENT. ALL
INFORMATION PROVIDED HERE IS SUBJECT TO CHANGE WITHOUT NOTICE. CONTACT YOUR
INTEL REPRESENTATIVE TO OBTAIN THE LATEST FORECAST, SCHEDULE, SPECIFICATIONS
AND ROADMAPS. \\

\noindent Intel technologies features and benefits depend on system
configuration and may require enabled hardware, software or service
activation. \\

\noindent No computer system can be absolutely secure. Intel does not control
or audit third-party benchmark data or the web sites referenced in this
document. You should visit the referenced web site and confirm whether
referenced data are accurate.  \\

\noindent Learn more at intel.com, or from the OEM or retailer. \\

\noindent The products described may contain design defects or errors known as
errata which may cause the product to deviate from published
specifications. Current characterized errata are available on request. \\

\noindent Intel, the Intel logo, and others are trademarks of Intel Corporation
in the U.S. and/or other countries. \\

\noindent *Other names and brands may be claimed as the property of others. \\

\noindent {\small\copyright} 2015 Intel Corporation

}
 

\newpage
\tableofcontents
\newpage

% Introduction  --------------------------------------------------

\section{Introduction} \label{sec:introduction}
% begin_ohpc_run
% ohpc_validation_comment -----------------------------------------------------------------------------------------
% ohpc_validation_comment  Example Installation Script Template
% ohpc_validation_comment  
% ohpc_validation_comment  This convenience script encapsulates command-line instructions highlighted in
% ohpc_validation_comment  an OpenHPC Install Guide that can be used as a starting point to perform a local
% ohpc_validation_comment  cluster install beginning with bare-metal. Necessary inputs that describe local
% ohpc_validation_comment  hardware characteristics, desired network settings, and other customizations
% ohpc_validation_comment  are controlled via a companion input file that is used to initialize variables 
% ohpc_validation_comment  within this script.
% ohpc_validation_comment   
% ohpc_validation_comment  Please see the OpenHPC Install Guide(s) for more information regarding the
% ohpc_validation_comment  procedure. Note that the section numbering included in this script refers to
% ohpc_validation_comment  corresponding sections from the companion install guide.
% ohpc_validation_comment -----------------------------------------------------------------------------------------
% ohpc_validation_newline

% ohpc_command inputFile=${OHPC_INPUT_LOCAL:-/opt/ohpc/pub/doc/recipes/BOSSHORT/input.local}
% ohpc_validation_newline
% ohpc_command if [ ! -e ${inputFile} ];then
% ohpc_command    echo "Error: Unable to access local input file -> ${inputFile}"
% ohpc_command    exit 1
% ohpc_command else
% ohpc_command    . ${inputFile} || { echo "Error sourcing ${inputFile}"; exit 1; }
% ohpc_command fi

% ohpc_validation_newline
% ohpc_validation_comment ---------------------------- Begin OpenHPC Recipe ---------------------------------------
% ohpc_validation_comment Commands below are extracted from an OpenHPC install guide recipe and are intended for 
% ohpc_validation_comment execution on the master SMS host.
% ohpc_validation_comment -----------------------------------------------------------------------------------------

% end_ohpc_run

This guide presents a simple cluster installation procedure using components
from the Forest Peak (\FSP{}) software stack. \FSP{} represents an aggregation of a number of
common ingredients required to deploy and manage an HPC Linux* cluster including
provisioning tools, resource management, I/O clients, development tools, and a variety of
scientific libraries. These packages have been pre-built with HPC integration
in mind and represent a mix of open-source components combined with \Intel{}
development and analysis tools (e.g. \Intel{} Parallel Studio XE Cluster Edition).
%This install guide assumes availablility of a {\em master} host
%that will have access to common OS repositories and the \FSP{} repository in order
%to facilitate software installs and dependency resolution.  
The documentation herein is intended to be reasonably generic, but uses the
underlying motivation of a small, 4-node cluster install to define a step-by-step
process. Several optional customizations are included and the intent is that
these collective instructions can be modified as needed for local site
customizations. This guide is targeted at veteran Linux* system administrators. Knowledge of 
software package management, system networking, and PXE booting is assumed. Shell 
examples in the text are givin in BASH, though the equivilent commands in other
shells should work fine. Command-line inputs in this guide are written as follows:

\begin{lstlisting}[language=bash,literate={-}{-}1,keywords={},upquote=true]
[master]$ rm -rf /.
\end{lstlisting}
 \\

\noindent {\bf Base Linux Edition}: this edition of the guide highlights
installation without the use of a companion configuration management system and
directly uses distro-provided package management tools for component
selection. The steps that follow also highlight specific changes to system
configuration files that are required as part of the cluster install
process.
%Other editions of this guide provide similar install steps when using
%specific configuration management systems that can simplify the installation
%and configuration process.

\subsection{Target Audience}

This guide is targeted at experienced \Linux{} system administrators for HPC
environments. Knowledge of software package management, system networking, and
PXE booting is assumed.  Command-line input examples are highlighted throughout
this guide via the following syntax:

\begin{lstlisting}[language=bash,literate={-}{-}1,keywords={},upquote=true]
[master](*\#*) echo "OpenHPC hello world"
\end{lstlisting}

Unless specified otherwise, the examples presented are executed with
elevated (root) privileges. The examples also presume use of the BASH login
shell, though the equivalent commands in other shells can be substituted.

Usefull tips and common pitfalls are highlighted in this guide with the
following syntax:

\begin{center}
\begin{tcolorbox}[]
\small
If you don't know where you are going, you might wind up someplace else.
--Yogi Berra
\end{tcolorbox}
\end{center}


%\noindent {\bf Requirements/Assumptions}: 
\subsection{Requirements/Assumptions}
This installation recipe assumes the availability of a single head node {\em
master}, and four {\em compute} nodes. The {\em master} node serves as the
overall system management server (SMS) and is provisioned with \baseOS{} and is
subsequently configured to provision the remaining {\em compute} nodes with
\Warewulf{} in a stateless configuration. For power management, we assume that the
compute node BMCs are available via IPMI from the chosen master host. For file
systems, we assume that the chosen master server will host an \NFS{} file
system that is made available to the compute nodes. Installation information is
also discussed to optionally include a \Lustre{} file system mount and in this
case, the \Lustre{} file system is assumed to exist previously. 

\begin{figure}[hbt]
\center
\includegraphics[width=0.8\linewidth]{fsp-arch-small.pdf}
\vspace*{-0.2cm}
\caption{Overview of physical cluster architecture.} \label{fig:physical_arch}
\end{figure}
\mbox{}

An outline of the physical architecture discussed is shown in
Figure~\ref{fig:physical_arch} and it highlights the high-level networking
configuration. The master hosts requires at least two ethernet interfaces with
{\em eth0} connected to the local data center network and {\em eth1} used to
provision and manage the cluster backend.  Two logical tcp interfaces are
expected to each compute node: the first is the standard ethernet interface
that will be used for provisioning and resource management. The second is
used to connect to the hosts BMC and is used for power management and
remote console access.  In addition to the tcp networking, there is a
high-speed network (\InfiniBand{} in this recipe) that is also connected to
each of the hosts. This high speed network is used for application message
passing and optionally for \Lustre{} connectivity as well.


\subsection{Bring your own license} \label{sec:byol}
\OHPC{} provides a variety of integrated, pre-packaged elements from the
\IntelR{}~Parallel Studio XE software suite. Portions of the runtimes provided
by the included compilers and MPI components are freely usable. However, in
order to compile new binaries using these tools (or access other analysis tools
like \IntelR{}~Trace Analyzer and Collector, \IntelR{} Inspector, etc.) you
will need to provide your own valid license, and \OHPC{} adopts a {\em
 bring-your-own-license} model. Note that licenses are provided free of
charge for many categories of use. In particular, licenses for compilers and
developments tools are provided at no cost to academic researchers or
developers contributing to open-source software projects. More information on
this program can be found at: 

\begin{center}
\href{https://software.intel.com/en-us/qualify-for-free-software}
     {\color{blue}{https://software.intel.com/en-us/qualify-for-free-software}}
\end{center}


\noindent {\bf Inputs}: since this recipe details installing a cluster
starting from bare-metal, there is a requirement to define IP addresses and gather
hardware MAC addresses in order to support controlled provisioning. These values
are unique to the hardware being used, and this document uses \texttt{<variable>}
names in the command-line examples that follow to highlight where local site
inputs are required. A summary of the required variables used in this recipe
are as follows: \\

\vspace*{0.2cm}
\begin{tabular}{@{}>{\textbullet}cll@{}}
& \texttt{<master\_name>}  & Hostname for master server \\
& \texttt{<master\_ip>} & IP address on master server for provisioning interface \\
& \texttt{<internal\_netmask>} & Subnet netmask for internal cluster network \\
& \texttt{<c1\_ip>},\texttt{<c2\_ip>},\texttt{<c3\_ip>},\texttt{<c4\_ip>}
& Desired compute node addresses \\
& \texttt{<c1\_bmc>},\texttt{<c2\_bmc>},\texttt{<c3\_bmc>},\texttt{<c4\_bmc>}
& BMC addresses for computes \\
& \texttt{<c1\_mac>},\texttt{<c2\_mac>},\texttt{<c3\_mac>},\texttt{<c4\_mac>}
& MAC addresses for computes \\
\end{tabular}


% Base Operating System --------------------------------------------

\section{Install Base Operating System (BOS)}
In an external setting, installing the desired BOS on a
{\em master} SMS host typically involves booting from a DVD ISO image on a new
server. With this approach, insert the \baseOS{} DVD, power cycle the host, and
follow the distro provided directions to install the BOS on your chosen {\em
master} host.  Alternatively, if choosing to use a pre-installed server, please
verify that it is provisioned with the required \baseOS{} distribution. \\

\ifnottoggleverb{isWarewulf4}
Prior to beginning the installation process of \OHPC{} components, several additional
considerations are noted here for the SMS host configuration. First,
the installation recipe herein assumes that
the SMS host name is resolvable locally. Depending on the manner in which you
installed the BOS, there may be an adequate entry already defined
in \path{/etc/hosts}. If not, the following addition can be used to identify
your SMS host.
\begin{lstlisting}[language=bash,keywords={}]
[sms](*\#*) echo ${sms_ip} ${sms_name} >> /etc/hosts
\end{lstlisting}
\fi

While it is theoretically possible to enable SELinux on a cluster provisioned
with \provisioner{},
doing so is beyond the scope of this document. Even the use of
permissive mode can be problematic and we therefore recommend disabling SELinux on the {\em
master} SMS host. If SELinux components are installed locally,
the \texttt{selinuxenabled} command can be used to determine if SELinux is
currently enabled. If enabled, consult the distro documentation for information
on how to disable. \\

Finally, provisioning services rely on DHCP, TFTP, and HTTP network protocols.
Depending on the local BOS configuration on the SMS host, default firewall
rules may prohibit these services. Consequently, this recipe assumes that the
local firewall running on the SMS host is disabled (it is still recommended to
have additional security boundaries like a firewall to protect the cluster from
the Internet). If installed, the default firewall service can be disabled as
follows:


% ------------------------------------------------------------------

\section{Install \OHPC{} Components} \label{sec:basic_install}
With the BOS installed and booted, the next step is to add desired \OHPC{} packages
onto the {\em master} server in order to provide provisioning and resource
management services for the rest of the cluster. The following subsections
highlight this process.


\subsection{Enable \OHPC{} repository for local use} \label{sec:enable_repo}
To begin, enable use of the \OHPC{} repository by adding it to the local list
of available package repositories. Note that this requires network access from
your {\em master} server to the \OHPC{} repository, or alternatively, that
the \OHPC{} repository be mirrored locally.  In cases where network external
connectivity is available, \OHPC{} provides an \texttt{ohpc-release} package
that includes GPG keys for package signing and repository enablement.  The
example which follows illustrates installation of the \texttt{ohpc-release}
package directly from the \OHPC{} build server.

\iftoggleverb{isCentOS}
% CentOS
\begin{lstlisting}[language=bash,keywords={},basicstyle=\fontencoding{T1}\fontsize{7.75}{10}\ttfamily,
	literate={VER}{\OHPCVerTree{}}1 {OSREPO}{\OSRepo{}}1 {ARCH}{\arch{}}1 {-}{-}1]
[sms](*\#*) yum install http://build.openhpc.community/OpenHPC:/VER/OSREPO/ARCH/ohpc-release-VER-1.ARCH.rpm
\end{lstlisting}
\else
% non-CentOS
\begin{lstlisting}[language=bash,keywords={},basicstyle=\fontencoding{T1}\footnotesize\ttfamily,
	literate={VER}{\OHPCVerTree{}}1 {OSREPO}{\OSRepo{}}1 {ARCH}{\arch{}}1 {-}{-}1]
[sms](*\#*) rpm -ivh http://build.openhpc.community/OpenHPC:/VER/OSREPO/ARCH/ohpc-release-VER-1.ARCH.rpm
\end{lstlisting}
\fi

\begin{center}
\begin{tcolorbox}[]
\small Many sites may find it useful or necessary to maintain a local copy of the
OpenHPC repositories. To facilitate this need, we provide downloadable tar
archives -- one containing a repository of binary packages as well as any
available updates, and one containing a repository of source RPMS. The tar file
also contains a small bash script to configure the package manager to use the
local repository after download. To use, simply unpack the tarball where you
would like to host the local repository and execute the make\_repo.sh script.
Tar files for this release can be found at \href{http://build.openhpc.community/dist/\OHPCVersion{}}
        {\color{blue}http://build.openhpc.community/dist/\OHPCVersion{}}
\end{tcolorbox}
\end{center}


% begin_ohpc_run
% ohpc_validation_newline
% ohpc_validation_comment Verify OpenHPC repository has been enabled before proceeding
% ohpc_validation_newline
% ohpc_command yum repolist | grep -q OpenHPC
% ohpc_command if [ $? -ne 0 ];then
% ohpc_command    echo "Error: OpenHPC repository must be enabled locally"
% ohpc_command    exit 1
% ohpc_command fi
% end_ohpc_run

In addition to the \OHPC{} package repository, the {\em master} host also
requires access to the standard base OS distro repositories in order to resolve
necessary dependencies. For \baseOS{}, the requirements are to have access to
both the base OS and EPEL repositories for which mirrors are freely available online:

\begin{itemize*}
\item CentOS-7 - Base 7.2.1511
  (e.g. \href{http://mirror.centos.org/centos-7/7.2.1511/os/x86\_64}
             {\color{blue}{http://mirror.centos.org/centos-7/7.2.1511/os/x86\_64}} )
\item EPEL 7 (e.g. \href{http://download.fedoraproject.org/pub/epel/7/x86\_64}
                        {\color{blue}{http://download.fedoraproject.org/pub/epel/7/x86\_64}} )
\end{itemize*}

\noindent The public EPEL repository will be enabled automatically upon installation of the 
\texttt{ohpc-release} package. Note that this requires the CentOS Extras
repository, which is shipped with CentOS and is enabled by default.

\subsection{Installation template}
The collection of command-line instructions that follow in this guide, when
combined with local site inputs, can be used to implement a 
bare-metal system installation and configuration. The format of these commands
is intended to be usable via direct cut and paste (with variable substitution
for site-specific settings). Alternatively, the \OHPC{} documentation package
(\texttt{docs-ohpc}) includes a template script which includes a summary of all 
of the commands used herein. This script can be used in conjunction with a 
simple text file to define the local site variables defined in the previous 
section (\S~\ref{sec:inputs}) and is provided as a convenience for 
administrators. For additional information on accessing this script, please see
Appendix~\ref{appendix:template_script}.




\subsection{Add provisioning services on {\em master} node} \label{sec:add_provisioning}
With the \OHPC{} repository enabled, we can now begin adding desired components onto the
{\em master} server. This repository provides a number of aliases that group
logical components together in order to help aid in this process. For
reference, a complete list of available group aliases and RPM packages available
via \OHPC{} are provided in Appendix~\ref{appendix:manifest}. To add
support for provisioning services, the following commands illustrate addition
of a common base package followed by the Warewulf provisioning system.

% begin_ohpc_run
% ohpc_comment_header Add baseline OpenHPC and provisioning services \ref{sec:add_provisioning}
\begin{lstlisting}[language=bash,keywords={}]
[sms](*\#*) (*\groupinstall*) ohpc-base
[sms](*\#*) (*\groupinstall*) ohpc-warewulf
\end{lstlisting}
% end_ohpc_run



Provisioning services with Warewulf rely on TFTP and HTTP transport protocols.
Depending on the local Base OS configuration on the SMS host, default firewall
rules may prohibit these services. Consequently, this recipe assumes that the local
firewall running on the SMS host is disabled. If installed, the default
firewall service can be disabled as follows:

% begin_fsp_run
% fsp_validation_comment Disable firewall 
\begin{lstlisting}[language=bash,keywords={}]
[master]$ systemctl disable firewalld
[master]$ systemctl stop firewalld
\end{lstlisting}
% end_fsp_run


% begin_ohpc_run
% ohpc_validation_comment Disable firewall 
\begin{lstlisting}[language=bash,keywords={}]
[sms](*\#*) systemctl disable firewalld
[sms](*\#*) systemctl stop firewalld
\end{lstlisting}
% end_ohpc_run

\begin{center}
\begin{tcolorbox}[]
\small Many server BIOS configurations have PXE network booting configured
as the primary option in the boot order by default. If your compute nodes have
a different device as the first in the sequence, the \texttt{ipmitool} utility
can be used to enable PXE.
\begin{lstlisting}[language=bash]
[sms](*\#*) ipmitool -E -I lanplus -H ${bmc_ipaddr} -U root chassis bootdev pxe options=persistent
\end{lstlisting}
\end{tcolorbox}
\end{center}


HPC systems typically rely on synchronized clocks throughout the system and the
NTP protocol can be used to facilitate this synchronization. To enable NTP
services on the SMS host with a specific server \texttt{\$\{ntp\_server\}},
issue the following:

% begin_ohpc_run
% ohpc_validation_comment Enable NTP services on SMS host
\begin{lstlisting}[language=bash,literate={-}{-}1,keywords={},upquote=true,keepspaces]
[sms](*\#*) systemctl enable ntpd.service
[sms](*\#*) echo "server ${ntp_server}" >> /etc/ntp.conf
[sms](*\#*) systemctl restart ntpd
\end{lstlisting}
% end_ohpc_run


\subsection{Add resource management services on {\em master} node} \label{sec:add_rm}
\OHPC{} provides multiple options for distributed resource management. 
The following command adds the \SLURM{} workload manager server components to the
chosen {\em master} host. Note that client-side components will be added to
the corresponding compute image in a subsequent step.

% begin_ohpc_run
% ohpc_comment_header Add resource management services on master node \ref{sec:add_rm}
\begin{lstlisting}[language=bash,keywords={}]
# Install slurm server meta-package
[sms](*\#*) (*\install*) ohpc-slurm-server

# Identify resource manager hostname on master host
[sms](*\#*) perl -pi -e "s/ControlMachine=\S+/ControlMachine=${sms_name}/" /etc/slurm/slurm.conf
\end{lstlisting}
% end_ohpc_run

\begin{center}
\begin{tcolorbox}[]
  \small SLURM requires enumeration of the physical hardware characteristics
  for compute nodes under its control. In particular, three configuration
  parameters combine to define consumable compute resources: {\em Sockets},
  {\em CoresPerSocket}, and {\em ThreadsPerCore}. The default configuration
  file provided via \OHPC{} assumes dual-socket, 8 cores per socket, and two
  threads per core for this 4-node example. If this does not reflect your local
  hardware, please update the configuration file at
  \path{/etc/slurm/slurm.conf} accordingly to match your particular hardware.
  Note that the SLURM project provides an easy-to-use online configuration tool that
  can be accessed
 \href{https://slurm.schedmd.com/configurator.html}{\color{blue} here}. 
\end{tcolorbox}
\end{center}

% begin_ohpc_run
% ohpc_comment_header Update node configuration for slurm.conf
% ohpc_command if [[ ${update_slurm_nodeconfig} -eq 1 ]];then
% ohpc_indent 5
% ohpc_command perl -pi -e "s/^NodeName=.+$/#/" /etc/slurm/slurm.conf
% ohpc_command perl -pi -e "s/ Nodes=c\S+ / Nodes=c[1-$num_computes] /" /etc/slurm/slurm.conf
% ohpc_command echo -e ${slurm_node_config} >> /etc/slurm/slurm.conf
% ohpc_indent 0
% ohpc_command fi
% end_ohpc_run

Other versions of this guide are available that describe installation of alternate
resource management systems, and they can be found in the \texttt{docs-ohpc}
package.



\subsection{Add \InfiniBand{} support services on {\em master} node} \label{sec:add_ofed}

The following command adds OFED and PSM support using base distro-provided drivers
to the chosen {\em master} host.

% begin_ohpc_run
% ohpc_comment_header Add InfiniBand support services on master node \ref{sec:add_ofed}
\begin{lstlisting}[language=bash,keywords={}]
[sms](*\#*) (*\groupinstall*) "InfiniBand Support"
[sms](*\#*) (*\install*) infinipath-psm

# Load IB drivers
[sms](*\#*) systemctl start rdma
\end{lstlisting}
% end_ohpc_run

With the \InfiniBand{} drivers included, you can also enable (optional) IPoIB functionality
which provides a mechanism to send IP packets over the IB network. If you plan
to mount a \Lustre{} file system over \InfiniBand{} (see \S\ref{sec:lustre_client}
for additional details), then having IPoIB enabled is a requirement for the
\Lustre{} client. \OHPC{} provides a template configuration file to aid in setting up
an {\em ib0} interface on the {\em master} host. To use, copy the template
provided and update the \texttt{\$\{sms\_ipoib\}} and
\texttt{\$\{ipoib\_netmask\}} entries to match local desired settings (alter ib0
naming as appropriate if system contains dual-ported or multiple HCAs). 

% begin_ohpc_run
% ohpc_validation_newline
% ohpc_command if [[ ${enable_ipoib} -eq 1 ]];then
% ohpc_indent 5
% ohpc_validation_comment Enable ib0
\begin{lstlisting}[language=bash,literate={-}{-}1,keywords={},upquote=true]
[sms](*\#*) cp /opt/ohpc/pub/examples/network/centos/ifcfg-ib0 /etc/sysconfig/network-scripts

# Define local IPoIB address and netmask
[sms](*\#*) perl -pi -e "s/master_ipoib/${sms_ipoib}/" /etc/sysconfig/network-scripts/ifcfg-ib0
[sms](*\#*) perl -pi -e "s/ipoib_netmask/${ipoib_netmask}/" /etc/sysconfig/network-scripts/ifcfg-ib0

# Initiate ib0
[sms](*\#*) ifup ib0
\end{lstlisting}
% ohpc_indent 0
% ohpc_command fi
% end_ohpc_run

\vspace*{-0.15cm}
\subsection{Complete basic Warewulf setup for {\em master} node} \label{sec:setup_ww}
At this point, all of the packages necessary to use \Warewulf{} on the {\em
master} host should be installed. Next, we need to update several
configuration files in order to allow \Warewulf{} to work with \baseOS{} and to
support local provisioning using a second private interface (refer to
Figure~\ref{fig:physical_arch}).
%\vspace*{-0.05cm}
\begin{center}
\begin{tcolorbox}[]
\small
By default, \Warewulf{} is configured to
provision over the \texttt{eth1} interface and the steps below include updating
this setting to override with a potentially alternatively-named interface specified by
\texttt{\$\{sms\_eth\_internal\}}.
\end{tcolorbox}
\end{center}


% begin_ohpc_run
% ohpc_comment_header Complete basic Warewulf setup for master node \ref{sec:setup_ww}
%\begin{verbatim}

\begin{lstlisting}[language=bash,literate={-}{-}1,keywords={},upquote=true,keepspaces]
# Configure Warewulf provisioning to use desired internal interface
[sms](*\#*) perl -pi -e "s/device = eth1/device = ${sms_eth_internal}/" /etc/warewulf/provision.conf

# Enable tftp service for compute node image distribution
[sms](*\#*) perl -pi -e "s/^\s+disable\s+= yes/ disable = no/" /etc/xinetd.d/tftp

# Enable http access for Warewulf cgi-bin directory to support newer apache syntax
[sms](*\#*) export MODFILE=/etc/httpd/conf.d/warewulf-httpd.conf
[sms](*\#*) perl -pi -e "s/cgi-bin>\$/cgi-bin>\n Require all granted/" $MODFILE

[sms](*\#*) perl -pi -e "s/Allow from all/Require all granted/" $MODFILE
[sms](*\#*) perl -ni -e "print unless /^\s+Order allow,deny/" $MODFILE

# Enable internal interface for provisioning
[sms](*\#*) ifconfig ${sms_eth_internal} ${sms_ip} netmask ${internal_netmask} up

# Restart/enable relevant services to support provisioning
[sms](*\#*) systemctl restart xinetd
[sms](*\#*) systemctl enable mariadb.service
[sms](*\#*) systemctl restart mariadb
[sms](*\#*) systemctl enable httpd.service
[sms](*\#*) systemctl restart httpd
\end{lstlisting}
%\end{verbatim}
% end_ohpc_run


\subsection{Define {\em compute} image for provisioning}

With the provisioning services enabled, the next step is to define and
customize a system image that can subsequently be used to provision one or more
{\em compute} nodes. The following subsections highlight this process.

\subsubsection{Build initial BOS image} \label{sec:assemble_bos}

The \OHPC{} build of \Warewulf{} includes specific enhancements enabling support for
\baseOS{}. The following steps illustrate the process to build a minimal, default
image for use with \Warewulf{}. We begin by defining a directory structure on the 
{\em master} host that will represent the root filesystem of the compute node. The 
default location for this example is in
\texttt{/opt/ohpc/admin/images/\baseos{}}.

\begin{center}
  \begin{tcolorbox}[]
    \small Note that \Warewulf{} is configured by default to access an external repository
    (vault.centos.org) during the \texttt{wwmkchroot} process.
    If the master host cannot reach the public CentOS mirrors, or if you
    prefer to access a locally cached mirror, set the
    \texttt{\$\{BOS\_MIRROR\}} environment variable to your desired repo
    location and update the template file 
    {\em prior} to running the \texttt{wwmkchroot} command below. For
    example:

% begin_ohpc_run
% ohpc_command if [ ! -z ${BOS_MIRROR+x} ]; then
% ohpc_indent 5
\begin{lstlisting}[language=bash,keywords={}]
# Override default OS repository (optional) - set BOS_MIRROR variable to desired repo location
[sms](*\#*) perl -pi -e "s#^YUM_MIRROR=(\S+)#YUM_MIRROR=${BOS_MIRROR}#" \
   /usr/libexec/warewulf/wwmkchroot/centos-7.tmpl
\end{lstlisting}
% ohpc_indent 0
% ohpc_command fi
% end_ohpc_run

\end{tcolorbox}
\end{center}

% begin_ohpc_run
% ohpc_comment_header Create compute image for Warewulf \ref{sec:assemble_bos}
\begin{lstlisting}[language=bash,literate={-}{-}1,keywords={},upquote=true,keepspaces]
# Define chroot location 
[sms](*\#*) export CHROOT=/opt/ohpc/admin/images/centos7.2

# Build initial chroot image
[sms](*\#*) wwmkchroot centos-7 $CHROOT
\end{lstlisting}
% end_ohpc_run

\subsubsection{Add \OHPC{} components} \label{sec:add_components}

\input{../../common/add_to_compute_chroot_intro}

\noindent Now, we can include additional components to the compute instance using
\texttt{yum} to install into the chroot location defined previously:

% begin_ohpc_run
% ohpc_validation_comment Add OpenHPC components to compute instance
\begin{lstlisting}[language=bash,literate={-}{-}1,keywords={},upquote=true]
# Add Slurm client support
[sms](*\#*) (*\groupchrootinstall*) ohpc-slurm-client

# Add IB support and enable
[sms](*\#*) (*\groupchrootinstall*) "InfiniBand Support"
[sms](*\#*) (*\chrootinstall*) infinipath-psm
[sms](*\#*) chroot $CHROOT systemctl enable rdma

# Add Network Time Protocol (NTP) support
[sms](*\#*) (*\chrootinstall*) ntp

# Add kernel drivers
[sms](*\#*) (*\chrootinstall*) kernel

# Include modules user environment
[sms](*\#*) (*\chrootinstall*) lmod-ohpc
\end{lstlisting}
% end_ohpc_run

\subsubsection{Customize system configuration} \label{sec:master_customization}

Prior to assembling the image, it is advantageous to perform any additional
customizations within the chroot environment created for the desired {\em
 compute} instance. The following steps document the process to add a local
{\em ssh} key created by \Warewulf{} to support remote access, identify the
resource manager server, configure NTP for compute resources, and enable \NFS{}
mounting of a \$HOME file system and the public \OHPC{} install path
(\texttt{/opt/ohpc/pub}) that will be hosted by the {\em master} host in this
example configuration.
%The \NFS{} exporting options use an address/netmask
%combination to limit the export scope to the defined compute nodes.

% begin_ohpc_run
% ohpc_comment_header Customize system configuration \ref{sec:master_customization}
\begin{lstlisting}[language=bash,literate={-}{-}1,keywords={},upquote=true]
# add new cluster key to base image
[sms](*\#*) wwinit ssh_keys
[sms](*\#*) cat ~/.ssh/cluster.pub >> $CHROOT/root/.ssh/authorized_keys

# add NFS client mounts of /home and /opt/ohpc/pub to base image
[sms](*\#*) echo "${sms_ip}:/home /home nfs nfsvers=3,rsize=1024,wsize=1024,cto 0 0" >> $CHROOT/etc/fstab
[sms](*\#*) echo "${sms_ip}:/opt/ohpc/pub /opt/ohpc/pub nfs nfsvers=3 0 0" >> $CHROOT/etc/fstab

# Identify resource manager hostname on master host
[sms](*\#*) perl -pi -e "s/ControlMachine=\S+/ControlMachine=${sms_name}/" /etc/slurm/slurm.conf
\end{lstlisting}
% end_ohpc_run

% begin_ohpc_run
\begin{lstlisting}[language=bash,literate={-}{-}1,keywords={},upquote=true]
# Export /home and OpenHPC public packages from master server to cluster compute nodes
[sms](*\#*) echo "/home *(rw,no_subtree_check,fsid=10,no_root_squash)" >> /etc/exports
[sms](*\#*) echo "/opt/ohpc/pub *(ro,no_subtree_check,fsid=11)" >> /etc/exports
[sms](*\#*) exportfs -a
[sms](*\#*) systemctl restart nfs

# Enable NTP time service on computes and identify master host as local NTP server
[sms](*\#*) chroot $CHROOT systemctl enable ntpd
[sms](*\#*) echo "server ${sms_ip}" >> $CHROOT/etc/ntp.conf
\end{lstlisting}
% end_ohpc_run


\begin{center}
\begin{tcolorbox}[]
  \small Slurm requires enumeration of the physical hardware characteristics
  for compute nodes under its control. In particular, three configuration
  parameters combine to define consumable compute resources: {\em Sockets},
  {\em CoresPerSocket}, and {\em ThreadsPerCore}. The default configuration
  file provided via \OHPC{} assumes dual-socket, 8 cores per socket, and two
  threads per core for this 4-node example. If this does not reflect your local
  hardware, please update the configuration file at
  \path{/etc/slurm/slurm.conf} accordingly to match your particular hardware.
\end{tcolorbox}
\end{center}

% Additional commands when additional computes are requested

% begin_ohpc_run
% ohpc_validation_newline
% ohpc_validation_comment Update basic slurm configuration if additional computes defined
% ohpc_command if [ ${num_computes} -gt 4 ];then
% ohpc_command    perl -pi -e "s/^NodeName=(\S+)/NodeName=c[1-${num_computes}]/" /etc/slurm/slurm.conf
% ohpc_command    perl -pi -e "s/^PartitionName=normal Nodes=(\S+)/PartitionName=normal Nodes=c[1-${num_computes}]/" /etc/slurm/slurm.conf

% ohpc_command    perl -pi -e "s/^NodeName=(\S+)/NodeName=c[1-${num_computes}]/" $CHROOT/etc/slurm/slurm.conf
% ohpc_command    perl -pi -e "s/^PartitionName=normal Nodes=(\S+)/PartitionName=normal Nodes=c[1-${num_computes}]/" $CHROOT/etc/slurm/slurm.conf
% ohpc_command fi
% end_ohpc_run

\clearpage
\subsubsection{Additional Customizations ({\em optional})} \label{sec:addl_customizations}
This section highlights common additional customizations that can {\em
optionally} be applied to the local cluster environment. These customizations
include:

\begin{multicols}{2}
\begin{itemize*}
\item Increase memlock limits

\nottoggle{ispbs}{\item Restrict ssh access to compute resources}

\item Add \beegfs{} client
\item Add \Lustre{} client

\iftoggle{isWarewulf}{\item Enable syslog forwarding}

\item Add \Nagios{} Core monitoring
\item Add \Ganglia{} monitoring
\item Add \clustershell{}
\item Add \mrsh{}
\item Add \genders{}
%%\item Add \powerman{}
\item Add \conman{}  
\end{itemize*}
\end{multicols}

\noindent Details on the steps required for each of these customizations are
discussed further in the following sections.


\paragraph{Increase locked memory limits}
In order to utilize \InfiniBand{} as the underlying high speed interconnect, it is
generally necessary to increase the locked memory settings for system
users. This can be accomplished by updating
the \texttt{/etc/security/limits.conf} file and this should be performed within
the {{\em compute} image and on all job submission hosts. In this recipe, jobs
are submitted from the {\em master} host, and the following commands can be
used to update the maximum locked memory settings on both the master host and
the compute image:

% begin_ohpc_run
% ohpc_comment_header Additional customizations \ref{sec:addl_customizations}
\begin{lstlisting}[language=bash,keywords={},upquote=true]
# Update memlock settings on master
[sms](*\#*) echo "* soft memlock unlimited" >> /etc/security/limits.conf
[sms](*\#*) echo "* hard memlock unlimited" >> /etc/security/limits.conf

# Update memlock settings within compute image
[sms](*\#*) echo "* soft memlock unlimited" >> $CHROOT/etc/security/limits.conf
[sms](*\#*) echo "* hard memlock unlimited" >> $CHROOT/etc/security/limits.conf
\end{lstlisting}
% end_ohpc_run




\paragraph{Enable ssh control via resource manager} 
An additional optional customization that is recommended is to
restrict \texttt{ssh} access on compute nodes to only allow access by users who
have an active job associated with the node. This can be enabled via the use of
a pluggable authentication module (PAM) provided as part of the \SLURM{} package
installs. To enable this feature within the {\em compute} image, issue the
following:

% begin_ohpc_run
% ohpc_validation_newline
% ohpc_validation_comment Enable slurm pam module
\begin{lstlisting}[language=bash,keywords={},upquote=true]
[sms](*\#*) echo "account    required     pam_slurm.so" >> $CHROOT/etc/pam.d/sshd
\end{lstlisting}
% end_ohpc_run




\paragraph{Add Cluster Checker} \label{sec:add_clck}
The \Intel{} Cluster Checker provides a convenient suite of diagnostics that
can be used to aid in isolating hardware and software problems on an installed
cluster. This package can be optionally added to the SMS and compute image
using the commands below. Note that a valid license file will also need to be
installed in order to use this software.

% begin_ohpc_run
% fsp_comment_header Add Cluster Checker to SMS and computes \ref{sec:add_clck}
\begin{lstlisting}[language=bash,keywords={}]
[master]$ (*\install*) intel-clck-ohpc
[master]$ (*\chrootinstall*) intel-clck-ohpc
\end{lstlisting}
% end_ohpc_run


\paragraph{Add \Lustre{} client} \label{sec:lustre_client}
To add \Lustre{} client support on the cluster, it necessary to install the client
and associated modules on each host needing to access a \Lustre{} file system. In
this recipe, it is assumed that the \Lustre{} file system is hosted by servers
that are pre-existing and are not part of the install process. Outlining the
variety of \Lustre{} client mounting options is beyond the scope of this document,
%(please consult \Lustre{} documentation for more details on failover configuration
%support and networking options), 
but the general requirement is to add a mount entry for the desired file system
that defines the management server (MGS) and underlying network transport
protocol. To add client mounts on both the {\em master} server and {\em
compute} image, the following commands can be used. Note that the \Lustre{} file
system to be mounted is identified by the \texttt{\$\{mgs\_fs\_name\}} variable. 
In this example, the file system is configured to be mounted locally
as \path{/mnt/lustre}.


% begin_ohpc_run
% ohpc_validation_newline 
% ohpc_validation_comment Enable Optional packages
% ohpc_validation_newline
% ohpc_command if [[ ${enable_lustre_client} -eq 1 ]];then
% ohpc_indent 5

% ohpc_validation_comment Install Lustre client on master
\begin{lstlisting}[language=bash,keywords={},upquote=true]
# Add Lustre client software to master host
[sms](*\#*) (*\install*) lustre-client-ohpc lustre-client-ohpc-modules
\end{lstlisting}
% end_ohpc_run

% begin_ohpc_run
% ohpc_validation_comment Enable lustre in WW compute image
\begin{lstlisting}[language=bash,keywords={},upquote=true]
# Include Lustre client software in compute image
[sms](*\#*) (*\chrootinstall*) lustre-client-ohpc lustre-client-ohpc-modules

# Include mount point and file system mount in compute image
[sms](*\#*) mkdir $CHROOT/mnt/lustre
[sms](*\#*) echo "${mgs_fs_name} /mnt/lustre lustre defaults,_netdev,localflock 0 0" >> $CHROOT/etc/fstab
\end{lstlisting}
% end_ohpc_run

The default underlying network type used by \Lustre{} is {\em tcp}. If your
external \Lustre{} file system is to be mounted using a network type other than
{\em tcp}, additional configuration files are necessary to identify the desired
network type. The example below illustrates creation of modprobe configuration files
instructing \Lustre{} to use an \InfiniBand{} network with the \textbf{o2ib} LNET driver
attached to \texttt{ib0}. Note that these modifications are made to both the
{\em master} host and {\em compute} image.

% begin_ohpc_run
% ohpc_validation_comment Enable o2ib for Lustre
\begin{lstlisting}[language=bash,keywords={},upquote=true]
[sms](*\#*) echo "options lnet networks=o2ib(ib0)" >> /etc/modprobe.d/lustre.conf
[sms](*\#*) echo "options lnet networks=o2ib(ib0)" >> $CHROOT/etc/modprobe.d/lustre.conf
\end{lstlisting}
% end_ohpc_run

With the \Lustre{} configuration complete, the client can be mounted on the {\em master}
host as follows:
% begin_ohpc_run
% ohpc_validation_comment mount Lustre client on master
\begin{lstlisting}[language=bash,keywords={},upquote=true]
[sms](*\#*) mkdir /mnt/lustre
[sms](*\#*) mount -t lustre -o localflock ${mgs_fs_name} /mnt/lustre
\end{lstlisting}
% ohpc_indent 0
% ohpc_command fi
% ohpc_validation_newline
% end_ohpc_run

\clearpage
\paragraph{Add \Nagios{} monitoring}
\Nagios{} is an open source infrastructure network monitoring package designed
to watch servers, switches, and various services and offers user-defined
alerting facilities for monitoring various aspects of an HPC
cluster. The core \Nagios{} daemon and a variety of monitoring plugins
are provided by the underlying OS distro and the following commands can
be used to install and configure a \Nagios{} server on the {\em
master} node, and add the facility to run tests and gather metrics
from provisioned {\em compute} nodes. This simple configuration
example is intended to be illustrative to walk through defining a
compute host group and enabling an ssh check for the computes. Users
are encouraged to
consult \Nagios{} \href{https://assets.nagios.com/downloads/nagioscore/docs/nagioscore/4/en/}{\color{blue}
documentation} for more information and can install additional plugins
as desired on login nodes, service nodes, or compute hosts.

% begin_ohpc_run
% ohpc_command if [[ ${enable_nagios} -eq 1 ]];then
% ohpc_indent 5
% ohpc_validation_comment Install Nagios on master and vnfs image
\begin{lstlisting}[language=bash,keywords={},upquote=true]
# Install nagios, nrep, and all available plugins on master host
[sms](*\#*) (*\install*) nagios nrpe nagios-plugins-*

# Install nrpe and an example plugin into compute node image
[sms](*\#*) (*\chrootinstall*) nrpe nagios-plugins-ssh

# Enable and configure Nagios NRPE daemon in compute image 
[sms](*\#*) chroot $CHROOT systemctl enable nrpe
[sms](*\#*) perl -pi -e "s/^allowed_hosts=/# allowed_hosts=/" $CHROOT/etc/nagios/nrpe.cfg
[sms](*\#*) echo "nrpe : ${sms_ip}  : ALLOW"    >> $CHROOT/etc/hosts.allow
[sms](*\#*) echo "nrpe : ALL : DENY"            >> $CHROOT/etc/hosts.allow

# Copy example Nagios config file to define a compute group and ssh check
# (note: edit as desired to add all desired compute hosts)
[sms](*\#*) cp /opt/ohpc/pub/examples/compute.cfg /etc/nagios/objects
# Register the config file with nagios
[sms](*\#*) echo "cfg_file=/etc/nagios/objects/compute.cfg" >> /etc/nagios/nagios.cfg

# Update location of mail binary for alert commands
[sms](*\#*) perl -pi -e "s/ \/bin\/mail/ \/usr\/bin\/mailx/g" /etc/nagios/objects/commands.cfg

# Update email address of contact for alerts
[sms](*\#*) perl -pi -e "s/nagios\@localhost/root\@${sms_name}/" /etc/nagios/objects/contacts.cfg

# Add check_ssh command for remote hosts
[sms](*\#*) echo command[check_ssh]=/usr/lib64/nagios/plugins/check_ssh localhost $CHROOT/etc/nagios/nrpe.cfg

# define password for nagiosadmin to be able to connect to web interface
[sms](*\#*) htpasswd -bc /etc/nagios/passwd nagiosadmin ${nagios_web_password}

# Enable Nagios on master, and configure
[sms](*\#*) systemctl enable nagios
[sms](*\#*) systemctl start nagios
# Update permissions on ping command to allow nagios user to execute
[sms](*\#*) chmod u+s `which ping`
\end{lstlisting}
% ohpc_indent 0
% ohpc_command fi
% end_ohpc_run



\clearpage
\paragraph{Add \Ganglia{} monitoring}
\Ganglia{} is a scalable distributed system monitor tool for high-performance computing systems such as clusters and grids. It allows the user to remotely view live or historical statistics (such as CPU load averages or network utilization) for all machines that are being monitored.

The following commands will setup \Ganglia{} on the master node.


\paragraph{Add \clustershell{}}
\clustershell{} is an event-based Python library to execute commands in parallel
across cluster nodes. 

% begin_ohpc_run
% ohpc_validation_newline
% ohpc_indent 5
% ohpc_validation_comment Install clustershell
\begin{lstlisting}[language=bash,keywords={},upquote=true]
# Install ClusterShell
[sms](*\#*) (*\install*) clustershell-ohpc

# Setup node definitions
[sms](*\#*) cd /etc/clustershell/groups.d
[sms](*\#*) mv local.cfg local.cfg.orig
[sms](*\#*) if true ; then
    cat << EOF > local.cfg
    adm: ${sms_name}
    compute: c[1-${num_computes}]
    all: @adm,@compute
    EOF
fi
\end{lstlisting}
% ohpc_indent 0
% end_ohpc_run



\paragraph{Add \mrsh{}}
\mrsh{} is a secure remote shell utility, like {\em ssh}, which uses
\MUNGE{} for authentication and encryption. By using the \MUNGE{}
installation used by \SLURM{}, \mrsh{} provides passwordless shell
access on systems using the same \MUNGE{} key without having to track
{\em ssh} keys. Like {\em ssh}, \mrsh{} provides a remote copy command,
{\em mrcp}, and can be used as a {\em rcmd} by {\em pdsh}.

% begin_ohpc_run
% ohpc_validation_newline
% ohpc_command if [ ${enable_mrsh} -eq 1 ];then
% ohpc_indent 5
% ohpc_validation_comment Install mrsh
\begin{lstlisting}[language=bash,keywords={},upquote=true]
# Install mrsh
[sms](*\#*) (*\install*) mrsh-ohpc mrsh-rsh-compat-ohpc
[sms](*\#*) (*\chrootinstall*) mrsh-ohpc mrsh-rsh-compat-ohpc mrsh-server-ohpc
\end{lstlisting}
% ohpc_indent 0
% ohpc_command fi
% end_ohpc_run



\paragraph{Add \genders{}}
\genders{} is a static cluster configuration database or node typing database
used for cluster configuration management. Other tools and users can access the
\genders{} database in order to make decisions about where an action, or even
what action, is appropriate based on associated types or "\genders{}".

Values may also be assigned to and retrieved from a {\em gender} to provide
further granularity.

% begin_ohpc_run
% ohpc_validation_newline
% ohpc_command if [[ ${enable_genders} -eq 1 ]];then
% ohpc_indent 5
% ohpc_validation_comment Install genders
\begin{lstlisting}[language=bash,keywords={},upquote=true]
# Install genders
[sms](*\#*) (*\install*) genders-ohpc

# Generate a sample genders file
[sms](*\#*) echo -e "${sms_name}\tsms" > /etc/genders
[sms](*\#*) for ((i=0; i<$num_computes; i++)) ; do
              echo -e "${c_name[$i]}\tcompute,bmc=${c_bmc[$i]}"
           done >> /etc/genders
\end{lstlisting}
% ohpc_indent 0
% ohpc_command fi
% end_ohpc_run



%% \paragraph{Add \powerman{}}
%% \powerman{} abstracts many different kinds of power control interfaces (IPMI, 
smart PDU, etc) into a single clean interface. \powerman{} accepts node ranges
for controlling sets of nodes, and is the default {\em resetcmd} for \conman{}
as configured by this recipe.

% begin_ohpc_run
% ohpc_validation_newline
% ohpc_command if [[ ${enable_powerman} -eq 1 ]];then
% ohpc_indent 5
% ohpc_validation_comment Optionally, install powerman
\begin{lstlisting}[language=bash,keywords={},upquote=true]
# Install powerman
[sms](*\#*) (*\install*) powerman-ohpc

# Create a basic powerman.conf
[sms](*\#*) cp /etc/powerman/powerman.conf{.example,}

[sms](*\#*) perl -pi -e 's/^\#(tcpwrappers yes)/$1/' /etc/powerman/powerman.conf
[sms](*\#*) perl -pi -e 's/^\#(listen "0.0.0.0:10101")/$1/' /etc/powerman/powerman.conf
[sms](*\#*) perl -pi -e 's/^\#(include "\/etc\/powerman\/ipmipower\.dev")/$1/' \
            /etc/powerman/powerman.conf
[sms](*\#*) for ((i=0; i<$num_computes; i++)) ; do
            perl -pi -e 'print "device \"ipmi'$i'\" \"ipmipower\" \"/usr/sbin/ipmipower -h ".
                "'${c_bmc[$i]}' -u '$bmc_username' -p ".
                "'${IPMI_PASSWORD:-undefined}'|&\"\n" if(/^\#device "ipmi1"/);' /etc/powerman/powerman.conf
        done
[sms](*\#*) for ((i=0; i<$num_computes; i++)) ; do
            perl -pi -e 'print "node \"'${c_name[$i]}'\" \"ipmi'$i'\" \"'${c_bmc[$i]}'\"\n"
                if(/^\#node "t1"/);' /etc/powerman/powerman.conf
        done

# Start powerman
[sms](*\#*) systemctl start powerman

# Check power status
[sms](*\#*) pm -q
\end{lstlisting}
% ohpc_indent 0
% ohpc_command fi
% end_ohpc_run



\paragraph{Add \conman{}} \label{sec:add_conman}
\conman{} is a serial console management program designed to support a large
number of console devices and simultaneous users. It supports logging console
device output and connecting to compute node consoles via IPMI
serial-over-lan. Installation and example configuration is outlined below.

% begin_ohpc_run
% ohpc_validation_newline
% ohpc_validation_comment Optionally, enable conman and configure
% ohpc_command if [[ ${enable_ipmisol} -eq 1 ]];then
% ohpc_indent 5
\begin{lstlisting}[language=bash,keywords={},upquote=true]
# Install conman to provide a front-end to compute consoles and log output
[sms](*\#*) (*\install*) conman-ohpc

# Configure conman for computes (note your IPMI password is required for console access)
[sms](*\#*) for ((i=0; i<$num_computes; i++)) ; do
              echo -n 'CONSOLE name="'${c_name[$i]}'" dev="ipmi:'${c_bmc[$i]}'" '
              echo 'ipmiopts="'U:${bmc_username},P:${IPMI_PASSWORD:-undefined},W:solpayloadsize'"'
        done >> /etc/conman.conf

# Enable and start conman
[sms](*\#*) systemctl enable conman
[sms](*\#*) systemctl start conman
\end{lstlisting}
% ohpc_indent 0
% ohpc_command fi
% end_ohpc_run

\noindent Note that an additional kernel boot option is typically necessary to
enable serial console output. This option is highlighted in \S\ref{sec:optional_kargs} after
compute nodes have been registered with the provisioning system.

% # Define node kernel arguments to support SOL console
% [sms](*\#*) wwsh -y provision set "${compute_regex}" --kargs "${kargs} console=ttyS1,115200"


\clearpage
\paragraph{Enable forwarding of system logs} \label{sec:add_syslog}
It is often desirable to consolidate system logging information for the cluster in a
central location, both to provide easy access to the data, and to reduce the
impact of storing data inside the stateless compute node's memory footprint. The
following commands highlight the steps necessary to configure compute nodes to
forward their logs to the SMS, and to allow the SMS to accept these log requests.


% begin_ohpc_run
% ohpc_comment_header Add Cluster Checker to SMS and computes \ref{sec:add_clck}
\begin{lstlisting}[language=bash,keywords={}]
# Configure SMS to receive messages and reload rsyslog configuration
[master](*\#*) echo "$ModLoad imudp" >> /etc/rsyslog.conf
[master](*\#*) echo "$UDPServerRun 514" >> /etc/rsyslog.conf
[master](*\#*) systemctl kill -s HUP rsyslog

# Define compute node forwarding destination
[master](*\#*) echo "*.* @${sms_eth_internal}:514" >> $CHROOT/etc/rsyslog.conf
\end{lstlisting}
% end_ohpc_run


\subsubsection{Import files} \label{sec:file_import}
The \Warewulf{} system includes functionality to import arbitrary files from
the provisioning server for distribution to managed hosts. This is one way to
distribute user credentials to {\em compute} nodes. To import local file-based
credentials, issue the following:

% begin_ohpc_run
% ohpc_comment_header Import files \ref{sec:file_import}
\begin{lstlisting}[language=bash,literate={-}{-}1,keywords={},upquote=true]
[sms](*\#*) wwsh file import /etc/passwd
[sms](*\#*) wwsh file import /etc/group
[sms](*\#*) wwsh file import /etc/shadow

\end{lstlisting}
% \end_ohpc_run


%----------------
% CentOS specific
%----------------

% begin_ohpc_run
% ohpc_validation_newline
% ohpc_command if [[ ${enable_ipoib} -eq 1 ]];then
% ohpc_indent 5
\begin{lstlisting}[language=bash,literate={-}{-}1,keywords={},upquote=true]
[sms](*\#*) wwsh file import /opt/ohpc/pub/examples/network/centos/ifcfg-ib0.ww
[sms](*\#*) wwsh -y file set ifcfg-ib0.ww --path=/etc/sysconfig/network-scripts/ifcfg-ib0
\end{lstlisting}
% ohpc_indent 0
% ohpc_command fi
% \end_ohpc_run

\subsection{Finalizing provisioning configuration} \label{sec:assemble_bootstrap}

\Warewulf{} employs a two-stage boot process for provisioning nodes via
creation of a bootstrap image that is used to initialize the process, and a virtual node
file system capsule containing the full system image. This section highlights
creation of the necessary provisioning images, followed by the registration of
desired compute nodes.

\subsubsection{Assemble bootstrap image}

The bootstrap image includes the runtime kernel and associated modules, as well
as some simple scripts to complete the provisioning process. The
following commands highlight the inclusion of additional drivers and creation
of the bootstrap image based on the running kernel.

%\iftoggle{isCentOS_ww_slurm_aarch}{\clearpage}

% begin_ohpc_run
% ohpc_comment_header Assemble bootstrap image \ref{sec:assemble_bootstrap}
\begin{lstlisting}[language=bash,literate={-}{-}1,keywords={},upquote=true]
# (Optional) Include drivers from kernel updates;  needed if enabling additional kernel modules on computes
[sms](*\#*) export WW_CONF=/etc/warewulf/bootstrap.conf
[sms](*\#*) echo "drivers += updates/kernel/" >> $WW_CONF

# Build bootstrap image
[sms](*\#*) wwbootstrap `uname -r`
\end{lstlisting}
% end_ohpc_run

\subsubsection{Assemble Virtual Node File System (VNFS) image}

With the local site customizations in place, the following step uses the
\texttt{wwvnfs} command to assemble a VNFS capsule from the chroot environment
defined for the {\em compute} instance.

% begin_ohpc_run
% ohpc_validation_comment Assemble VNFS
\begin{lstlisting}[language=bash,literate={-}{-}1,keywords={},upquote=true]
[sms](*\#*) wwvnfs --chroot $CHROOT
\end{lstlisting}
% end_ohpc_run

\iftoggle{isCentOS_ww_slurm_aarch}{\vspace*{0.4cm}}

\iftoggle{isSLES_ww_slurm_aarch}{\vspace*{-0.1cm}}

\subsubsection{Register nodes for provisioning}

In preparation for provisioning, we can now define the desired network settings
for four example compute nodes with the underlying provisioning system and
restart the \texttt{dhcp} service. Note the use of variable names for the
desired compute hostnames, node IPs, and MAC addresses which should be modified
to accommodate local settings and hardware.  By default, \Warewulf{} uses
network interface names of the \texttt{eth\#} variety and adds kernel boot
arguments to maintain this scheme on newer kernels. Consequently, when specifying
the desired provisioning interface via the \texttt{\$eth\_provision} variable,
it should follow this convention. Alternatively, if you prefer to use the
predictable network interface naming scheme (e.g. names like \texttt{en4s0f0}),
additional steps are included to alter the default kernel boot arguments and take
the \texttt{eth\#} named interface down after bootstrapping so the normal init
process can bring it up again using the desired name.

\iftoggleverb{isx86}
Also included in these steps are commands
to enable \Warewulf{} to manage IPoIB settings and corresponding definitions of
IPoIB addresses for the compute nodes. This is typically optional unless you
are planning to include a \Lustre{} client mount over \InfiniBand{}.
\fi
The final step
in this process associates the VNFS image assembled in previous steps with the
newly defined compute nodes, utilizing the user credential files and munge key
that were imported in \S\ref{sec:file_import}.




%\clearpage
% begin_ohpc_run
% ohpc_validation_comment Add hosts to cluster

\begin{lstlisting}[language=bash,keywords={},upquote=true,basicstyle=\footnotesize\ttfamily,]
# Set provisioning interface as the default networking device
[sms](*\#*) echo "GATEWAYDEV=${eth_provision}" > /tmp/network.$$
[sms](*\#*) wwsh -y file import /tmp/network.$$ --name network
[sms](*\#*) wwsh -y file set network --path /etc/sysconfig/network --mode=0644 --uid=0

# Add nodes to Warewulf datastore
[sms](*\#*) for ((i=0; i<$num_computes; i++)) ; do
                wwsh -y node new ${c_name[i]} --ipaddr=${c_ip[i]} --hwaddr=${c_mac[i]} -D ${eth_provision}
        done
\end{lstlisting}
% end_ohpc_run

% begin_ohpc_run
% ohpc_validation_comment Add hosts to cluster (Cont.)
\begin{lstlisting}[language=bash,keywords={},upquote=true,basicstyle=\footnotesize\ttfamily,literate={BOSVER}{\baseos{}}1]
# Define provisioning image for hosts
[sms](*\#*) wwsh -y provision set "${compute_regex}" --vnfs=centos7.2 --bootstrap=`uname -r` \
    --files=dynamic_hosts,passwd,group,shadow,slurm.conf,munge.key,network 
\end{lstlisting}

% ohpc_validation_newline
% ohpc_validation_comment Optionally, add arguments to bootstrap kernel
% ohpc_command if [[ ${enable_kargs} ]]; then
% ohpc_command    wwsh provision set "${compute_regex}" --kargs=${kargs}
% ohpc_command fi

% ohpc_validation_newline
% ohpc_validation_comment Restart ganglia services to pick up hostfile changes
% ohpc_command if [[ ${enable_ganglia} -eq 1 ]];then
% ohpc_command   systemctl restart gmond
% ohpc_command   systemctl restart gmetad
% ohpc_command fi

% ohpc_validation_newline
% ohpc_validation_comment Optionally, define IPoIB network settings (required if planning to mount Lustre over IB)
% ohpc_command if [[ ${enable_ipoib} -eq 1 ]];then
% ohpc_indent 5
\begin{lstlisting}[language=bash,keywords={},upquote=true,basicstyle=\footnotesize\ttfamily]
# Optionally define IPoIB network settings (required if planning to mount Lustre over IB)
[sms](*\#*) for ((i=0; i<$num_computes; i++)) ; do
              wwsh -y node set ${c_name[$i]} -D ib0 --ipaddr=${c_ipoib[$i]} --netmask=${ipoib_netmask}
        done
[sms](*\#*) wwsh -y provision set "${compute_regex}" --fileadd=ifcfg-ib0.ww
\end{lstlisting}
% ohpc_indent 0
% ohpc_command fi
% ohpc_validation_newline
% end_ohpc_run

\begin{center}
\begin{tcolorbox}[]
\small \Warewulf{} includes a utility named \texttt{wwnodescan} 
to automatically register new compute nodes versus the outlined node-addition approach
which requires hardware MAC addresses to be gathered in advance.  With
\texttt{wwnodescan}, nodes will be added to the \Warewulf{} database in the
order in which their DHCP requests are received by the master, so care must be
taken to boot nodes in the order one wishes to see preserved in the Warewulf
database. The IP address provided will be incremented after each node is found,
and the utility will exit after all specified nodes have been found. Example
usage is highlighted below:
\begin{lstlisting}[language=bash,keywords={},upquote=true,basicstyle=\footnotesize\ttfamily,literate={BOSVER}{\baseos{}}1]
[sms](*\#*) wwnodescan --netdev=${eth_provision} --ipaddr=${c_ip[0]} --netmask=${internal_netmask} \
    --vnfs=BOSVER --bootstrap=`uname -r` ${c_name[0]}-${c_name[3]}
\end{lstlisting}
\end{tcolorbox}
\end{center}


% begin_ohpc_run
\begin{lstlisting}[language=bash,keywords={},upquote=true,basicstyle=\footnotesize\ttfamily]
# Restart dhcp / update PXE
[sms](*\#*) systemctl restart dhcpd
[sms](*\#*) wwsh pxe update
\end{lstlisting}
% end_ohpc_run

\subsubsection{Optional kernel arguments} \label{sec:optional_kargs}
If you chose to enable \conman{} in \S\ref{sec:add_conman}, additional
boot-time kernel arguments are needed to enable serial console
redirection. An example provisioning setting which adds to any other kernel arguments defined
in \texttt{\$\{kargs\}} is as follows:

% begin_ohpc_run
% ohpc_validation_newline
% ohpc_validation_comment Optionally, enable console redirection 
% ohpc_command if [[ ${enable_ipmisol} -eq 1 ]];then
% ohpc_indent 5
\begin{lstlisting}[language=bash,keywords={},upquote=true]
# Define node kernel arguments to support SOL console
[sms](*\#*) wwsh -y provision set "${compute_regex}" --kargs "${kargs} console=ttyS1,115200"
\end{lstlisting}
% ohpc_indent 0
% ohpc_command fi
% end_ohpc_run


\subsubsection{Optionally configure stateful provisioning}
Warewulf normally defaults to running the assembled VNFS image out of system
memory in a {\em stateless} configuration. Alternatively, Warewulf can also be
used to partition and format persistent storage such that the VNFS image can be
installed locally to disk in a {\em stateful} manner.  This does, however,
require that a boot loader (GRUB) be added to the image as follows:

\begin{lstlisting}[language=bash,literate={-}{-}1,keywords={},upquote=true]
# Add GRUB2 bootloader and re-assemble VNFS image
[sms](*\#*) (*\chrootinstall*) grub2
[sms](*\#*) wwvnfs  --chroot $CHROOT
\end{lstlisting}

\noindent Enabling stateful nodes also requires additional site-specific, disk-related
parameters in the Warewulf configuration, and several example partitioning scripts are 
provided in the distribution. 

\begin{lstlisting}[language=bash,literate={-}{-}1,keywords={},upquote=true]
# Select (and customize) appropriate parted layout example
[sms](*\#*) cp /etc/warewulf/filesystem/examples/gpt_example.cmds /etc/warewulf/filesystem/gpt.cmds
[sms](*\#*) wwsh provision set --filesystem=gpt "${compute_regex}" 
[sms](*\#*) wwsh provision set --bootloader=sda "${compute_regex}" 
\end{lstlisting}

\begin{center}
\begin{tcolorbox}[]
\small
Those provisioning compute nodes in UEFI mode will install a slightly different
set of packages in to the VNFS. Warewulf also provides an example EFI filesystem
layout.
\begin{lstlisting}[language=bash,literate={-}{-}1,keywords={},upquote=true]
# Add GRUB2 bootloader and re-assemble VNFS image
[sms](*\#*) (*\chrootinstall*) grub2-efi grub2-efi-modules
[sms](*\#*) wwvnfs  --chroot $CHROOT
[sms](*\#*) cp /etc/warewulf/filesystem/examples/efi_example.cmds /etc/warewulf/filesystem/efi.cmds
[sms](*\#*) wwsh provision set --filesystem=efi "${compute_regex}" 
[sms](*\#*) wwsh provision set --bootloader=sda "${compute_regex}" 
\end{lstlisting}
\end{tcolorbox}
\end{center}

\noindent Upon subsequent reboot of the modified nodes, Warewulf will partition
and format the disk to host the desired VNFS image.  Once the image is installed 
to disk, warewulf can be configured to use the nodes' local storage as the boot 
device.

\begin{lstlisting}[language=bash,literate={-}{-}1,keywords={},upquote=true]
# Configure local boot (after successful provisioning)
[sms](*\#*) wwsh provision set --bootlocal=normal "${compute_regex}"
\end{lstlisting}


\vspace*{-0.1cm}
\subsection{Boot compute nodes} \label{sec:boot_computes}
At this point, the {\em master} server should be able to boot the newly defined
compute nodes. Assuming that the compute hardware BIOS settings are configured
to boot over PXE, all that is required to initiate the provisioning process is to power
cycle each of the desired hosts.  On the Zeus cluster, each of the service
processors for the compute hosts are available via a separate network.
You can point a web browser to their respective IPs to reboot, or
you can issue ipmi commands directly from the {\em master} cluster node.  
 

%----------------
% CentOS specific
%----------------

\begin{center}
\begin{tcolorbox}[]
\small While the \texttt{pxelinux.0} and \texttt{lpxelinux.0} files that ship
with Warewulf to enable network boot support a wide range of hardware, some
hosts may boot more reliably or faster using the BOS versions provided via the
\texttt{syslinux-tftpboot} package. If you encounter PXE issues, consider
replacing the \texttt{pxelinux.0} and \texttt{lpxelinux.0} files supplied with
\texttt{warewulf-provision-ohpc} with versions from \texttt{syslinux-tftpboot}.
\end{tcolorbox}
\end{center}

\vspace*{-0.50cm}
\section{Install \OHPC{} Development Components}
The install procedure outlined in \S\ref{sec:basic_install} highlighted the
steps necessary to install a {\em master} host, assemble and customize a {\em
  compute} image, and provision several compute hosts from bare-metal. With
these steps completed, additional \OHPC{}-provided packages can now be added to
support a flexible HPC development environment including development tools,
C/C++/Fortran compilers, MPI stacks, and a variety of 3rd party libraries. The
following subsections highlight the additional software installation
procedures, including the addition of Intel licensed software (e.g. Composer
compiler suite, \Intel{} MPI). It is assumed that the end-site administrator
will procure and install the necessary licenses in order to use the Intel
proprietary software.


\vspace*{-0.15cm}
\subsection{Development Tools} \label{sec:install_dev_tools}
To aid in general development efforts, \OHPC{} provides recent versions of the \GNU{}
autotools collection and the Valgrind memory debugger. These can be installed as follows:

% begin_ohpc_run
% ohpc_comment_header Install Development Tools \ref{sec:install_dev_tools}
\begin{lstlisting}[language=bash,keywords={},literate={-}{-}1]
[master]$ (*\groupinstall*) ohpc-autotools
[master]$ (*\install*) valgrind-ohpc
\end{lstlisting}
% end_ohpc_run


\vspace*{-0.15cm}
\subsection{Compilers} \label{sec:install_compilers}
\FSP{} presently packages two compiler families ({\GNU{}} and {\Intel{}
  Parallel Studio}) that are integrated within the underlying
modules-environment system in a hierarchical fashion. End users of a \FSP{}
system can choose to access one compiler at a time and will be presented with
additional compiler-dependent software as a function of which compiler
toolchain is currently loaded. Each compiler toolchain can be installed
separately and the following commands illustrate the installation of both along
with any necessary dependencies:

% begin_fsp_run
% fsp_comment_header Install Compilers \ref{sec:install_compilers}
\begin{lstlisting}[language=bash]
[master]$ (*\install*) gnu-compilers-fsp intel-compilers-devel-fsp
\end{lstlisting}
% end_fsp_run


%\clearpage
\subsection{MPI Stacks} \label{sec:mpi}
For MPI development support, \OHPC{} presently provides pre-packaged builds for
the following MPI families and transport layers: 

\iftoggleverb{isx86}
% x86_64

\begin{table}[h]
\centering
\begin{tabular}{@{\hspace*{0.2cm}} *5l @{}}    \toprule
                                  & Ethernet (TCP)                 & \InfiniBand{}                  & \IntelR{} Omni-Path            \\ \midrule
\rowcolor{black!10} MPICH         & \multicolumn{1}{c}{\checkmark} &                                &                                \\
MVAPICH2                          &                                & \multicolumn{1}{c}{\checkmark} &                                \\
\rowcolor{black!10} MVAPICH2-psm2 &                                &                                & \multicolumn{1}{c}{\checkmark} \\
OpenMPI                           & \multicolumn{1}{c}{\checkmark} & \multicolumn{1}{c}{\checkmark} &                                \\
\rowcolor{black!10} OpenMPI-psm2  & \multicolumn{1}{c}{\checkmark} & \multicolumn{1}{c}{\checkmark} & \multicolumn{1}{c}{\checkmark} \\ \bottomrule
\end{tabular}
\end{table}

\else
% aarch64

\begin{table}[h]
\centering
\begin{tabular}{@{\hspace*{0.2cm}} *5l @{}}    \toprule
                                  & Ethernet (TCP)                 & \InfiniBand{}                              \\ \midrule
\rowcolor{black!10} MPICH         & \multicolumn{1}{c}{\checkmark} &                                            \\
\rowcolor{black!10} OpenMPI                           & \multicolumn{1}{c}{\checkmark} & \multicolumn{1}{c}{\checkmark} \\
\end{tabular}
\end{table}
\fi


% begin_ohpc_run
% ohpc_comment_header Install MPI Stacks \ref{sec:mpi}
% ohpc_command if [[ ${enable_mpi_defaults} -eq 1 ]];then
% ohpc_indent 5
\begin{lstlisting}[language=bash]
[sms](*\#*) (*\install*) openmpi-gnu-ohpc mvapich2-gnu-ohpc mpich-gnu-ohpc
\end{lstlisting}
% ohpc_indent 0
% ohpc_command elif [[ ${enable_mpi_opa} -eq 1 ]];then
% ohpc_indent 5
% ohpc_command (*\install*) openmpi-psm2-gnu-ohpc mvapich2-psm2-gnu-ohpc
% ohpc_indent 0
% ohpc_command fi
% end_ohpc_run

% TODO add tooltip about existence of OPA stacks




\subsection{Performance Tools} \label{sec:install_perf_tools}
To aid in application performance analysis, \OHPC{} provides a variety of
open-source and Intel licensed software (see \S\ref{sec:byol} for more details
regarding Intel software). These can be installed as follows:

% begin_ohpc_run
% ohpc_comment_header Install Performance Tools \ref{sec:install_perf_tools}
\begin{lstlisting}[language=bash,keywords={},literate={-}{-}1]
[sms](*\#*) (*\install*) papi-ohpc
[sms](*\#*) (*\install*) intel-itac-ohpc
[sms](*\#*) (*\install*) intel-vtune-ohpc
[sms](*\#*) (*\install*) intel-advisor-ohpc
[sms](*\#*) (*\install*) intel-inspector-ohpc
[sms](*\#*) (*\groupinstall*) ohpc-mpiP
[sms](*\#*) (*\groupinstall*) ohpc-tau
\end{lstlisting}
% end_ohpc_run




\subsection{Setup default development environment}
System users often find it convenient to have a default development environment
in place so that compilation can be performed directly for parallel programs
requiring MPI. This setup can be conveniently enabled via modules and the \OHPC{}
modules environment is pre-configured to load an \texttt{ohpc} module on login
(if present). The following package install provides a default
environment that enables autotools, the gnu compiler toolchain, and the
MVAPICH2 MPI stack.

% begin_ohpc_run
\begin{lstlisting}[language=bash]
[sms](*\#*) (*\install*) lmod-defaults-gnu-mvapich2-ohpc
\end{lstlisting}
% end_ohpc_run

\begin{center}
\begin{tcolorbox}[]
\small
If you want to change the default environment from the suggestion above, \OHPC{}
provides additional choices for each of the available compiler/MPI
combinations. In particular, the following packages can be substituted instead:
\begin{multicols}{2}
\begin{itemize*}
\item lmod-defaults-gnu-openmpi-ohpc
\item lmod-defaults-gnu-impi-ohpc
\item lmod-defaults-intel-mvapich2-ohpc
\item lmod-defaults-intel-openmpi-ohpc
\item lmod-defaults-intel-impi-ohpc
\end{itemize*}
\end{multicols}
\end{tcolorbox}
\end{center}


%\vspace*{0.2cm}
\subsection{3rd Party Libraries and Tools} \label{sec:3rdparty}
\OHPC{} provides pre-packaged builds for a number of popular open-source
tools and libraries used by HPC applications and developers. For
example, \OHPC{} provides builds for \FFTW{} and \hdffive{} (including serial and parallel
I/O support), and the \GNU{} Scientific Library (GSL). Again, multiple builds of
each package are available in the \OHPC{} repository to support multiple compiler
and MPI family combinations where appropriate. Note, however, that not all
combinatorial permutations may be available for components where there are known
license incompatibilities. The general naming convention
for builds provided by \OHPC{} is to append the compiler and MPI family name that
the library was built against directly into the package name. For example,
libraries that do not require MPI as part of the build process adopt the
following RPM name: \\

\noindent
\texttt{package-<compiler\_family>-ohpc-<package\_version>-<release>.rpm} \\

\noindent Packages that do require MPI as part of the build expand upon this convention to
additionally include the MPI family name as follows: \\

\noindent
\texttt{package-<compiler\_family>-<mpi\_family>-ohpc-<package\_version>-<release>.rpm} \\

To illustrate this further, the command below queries the locally configured
repositories to identify all of the available PETSc packages that were built
with the \GNU{} toolchain. The resulting output that is included shows that
pre-built versions are available for each of the supported MPI families
presented in \S\ref{sec:mpi}.

\iftoggleverb{isCentOS}
\begin{lstlisting}[language=bash,keywords={}]
[sms](*\#*) yum search petsc-gnu ohpc
Loaded plugins: fastestmirror
Loading mirror speeds from cached hostfile
=========================== N/S matched: petsc-gnu, ohpc ===========================
petsc-gnu-impi-ohpc.x86_64 : Portable Extensible Toolkit for Scientific Computation
petsc-gnu-mpich-ohpc.x86_64 : Portable Extensible Toolkit for Scientific Computation
petsc-gnu-mvapich2-ohpc.x86_64 : Portable Extensible Toolkit for Scientific Computation
petsc-gnu-openmpi-ohpc.x86_64 : Portable Extensible Toolkit for Scientific Computation
\end{lstlisting}
\else
\begin{lstlisting}[language=bash,keepspaces=true,keywords={}]
[sms](*\#*) zypper search -t package petsc-gnu-*-ohpc
Loading repository data...
Reading installed packages...

S | Name                    | Summary
--+-------------------------+--------------------------------------------------------+--------
i | petsc-gnu-mvapich2-ohpc | Portable Extensible Toolkit for Scientific Computation | package
i | petsc-gnu-openmpi-ohpc  | Portable Extensible Toolkit for Scientific Computation | package
\end{lstlisting}
\fi


%----------------
% CentOS specific
%----------------

\begin{lstlisting}[language=bash,keywords={}]
[sms](*\#*) yum search petsc-gnu ohpc
Loaded plugins: fastestmirror
Loading mirror speeds from cached hostfile
=========================== N/S matched: petsc-gnu, ohpc ===========================
petsc-gnu-impi-ohpc.x86_64 : Portable Extensible Toolkit for Scientific Computation
petsc-gnu-mvapich2-ohpc.x86_64 : Portable Extensible Toolkit for Scientific Computation
petsc-gnu-openmpi-ohpc.x86_64 : Portable Extensible Toolkit for Scientific Computation
\end{lstlisting}

% begin_ohpc_run
% ohpc_comment_header Install 3rd Party Libraries and Tools \ref{sec:3rdparty}
\begin{lstlisting}[language=bash,keywords={},upquote=true,keepspaces]
[sms](*\#*) (*\groupinstall*) ohpc-adios        # adds available Adios packages
[sms](*\#*) (*\groupinstall*) ohpc-boost        # adds available Boost packages
[sms](*\#*) (*\groupinstall*) ohpc-fftw         # adds available FFTW packages
[sms](*\#*) (*\groupinstall*) ohpc-gsl          # adds available GSL packages
[sms](*\#*) (*\groupinstall*) ohpc-hdf5         # adds available HDF5 packages
[sms](*\#*) (*\groupinstall*) ohpc-hypre        # adds available Hypre packages
[sms](*\#*) (*\groupinstall*) ohpc-metis        # adds available METIS packages
[sms](*\#*) (*\groupinstall*) ohpc-mpiP         # adds available mpiP packages
[sms](*\#*) (*\groupinstall*) ohpc-mumps        # adds available MUMPS packages
[sms](*\#*) (*\groupinstall*) ohpc-netcdf       # adds available NetCDF packages
[sms](*\#*) (*\groupinstall*) ohpc-numpy        # adds available numerical Python packages
[sms](*\#*) (*\groupinstall*) ohpc-openblas     # adds available OpenBLAS packages
[sms](*\#*) (*\groupinstall*) ohpc-petsc        # adds available PETSC packages
[sms](*\#*) (*\groupinstall*) ohpc-phdf5        # adds available (parallel) HDF5 packages
[sms](*\#*) (*\groupinstall*) ohpc-scalapack    # adds available ScaLAPACK packages
[sms](*\#*) (*\groupinstall*) ohpc-scipy        # adds available scientific Python packages
[sms](*\#*) (*\groupinstall*) ohpc-trilinos     # adds available Trilinos packages
[sms](*\#*) (*\install*) EasyBuild-ohpc         # adds EasyBuild
[sms](*\#*) (*\install*) R_base-ohpc            # adds R
\end{lstlisting}
% end_ohpc_run


\section{Resource Manager Startup} \label{sec:rms_startup}
In section \S\ref{sec:basic_install}, the \SLURM{} resource manager was installed
and configured for use on both the {\em master} host and {\em compute} node
instances. With the cluster nodes up and functional, we can now startup the
resource manager services in preparation for running user jobs. Generally, this
is a two-step process that requires starting up the controller daemons on the {\em
 master} host and the client daemons on each of the {\em compute} hosts.
%Since the {\em compute} hosts were booted into an image that included the SLURM client
%components, the daemons should already be running on the {\em compute}
%hosts. 
Note that \SLURM{} leverages the use of the {\em munge} library to provide
authentication services and this daemon also needs to be running on all hosts
within the resource management pool. 
%The munge daemons should already
%be running on the {\em compute} subsystem at this point, 
The following commands can be used to startup the necessary services to support
resource management under \SLURM{}.

\iftoggle{isCentOS}{\clearpage}

% begin_ohpc_run
% ohpc_comment_header Resource Manager Startup \ref{sec:rms_startup}
\begin{lstlisting}[language=bash,keywords={}]
# start munge and slurm controller on master host
[sms](*\#*) systemctl enable munge
[sms](*\#*) systemctl enable slurmctld
[sms](*\#*) systemctl start munge
[sms](*\#*) systemctl start slurmctld

# start slurm clients on compute hosts
[sms](*\#*) pdsh -w c[1-4] systemctl start slurmd
\end{lstlisting}
% end_ohpc_run

In the default configuration, the {\em compute} hosts will be initialized in an
{\em unknown} state. To place the hosts into production such that they are
eligible to schedule user jobs, issue the following:

% begin_ohpc_run
\begin{lstlisting}[language=bash]
[sms](*\#*) scontrol update nodename=c[1-4] state=idle
\end{lstlisting}
% end_ohpc_run


% Additional recipe commands for additional computes.

% begin_ohpc_run
% ohpc_validation_newline
% ohpc_validation_comment Startup slurm on additional computes if defined
% ohpc_command if [ ${num_computes} -gt 4 ];then
% ohpc_command    pdsh -w c[5-$num_computes] systemctl start slurmd
% ohpc_command    scontrol update nodename=c[5-$num_computes] state=idle
% ohpc_command fi
% end_ohpc_run


\section{Run a Test Job} \label{sec:test_job}
With the resource manager enabled for production usage, users should now be
able to run jobs.  Recall that we added a ``test'' user on the {\em master}
host in \S\ref{sec:add_rm} that can now be used to run an example test job.
\OHPC{} includes a simple ``hello-world'' MPI application in the
\texttt{/opt/ohpc/pub/examples} directory that can be used for this quick
compilation and execution.  To use the test account to compile and execute the
application interactively through the resource manager, execute the following:

\begin{lstlisting}[language=bash,keywords={}]
# switch to "test" user
[master]# su - test

# Compile MPI "hello world" example
[test@master ~]$ mpicc -o hello -O3 /opt/ohpc/pub/examples/mpi/hello.c

# Submit interactive job request and use prun to launch executable
[test@master ~]$ srun -n 8 -N 2 --pty /bin/bash

[test@c1 ~]$ prun ./hello

[prun] Master compute host = c1
[prun] Launch cmd = mpiexec.hydra -bootstrap slurm ./hello

 Hello, world (8 procs total)
    --> Process #   0 of   8 is alive. -> c1
    --> Process #   4 of   8 is alive. -> c2
    --> Process #   1 of   8 is alive. -> c1
    --> Process #   5 of   8 is alive. -> c2
    --> Process #   2 of   8 is alive. -> c1
    --> Process #   6 of   8 is alive. -> c2
    --> Process #   3 of   8 is alive. -> c1
    --> Process #   7 of   8 is alive. -> c2
\end{lstlisting}


\clearpage
\appendix
%\section*{Appendices}
{\bf \LARGE \centerline{Appendices}} \vspace*{0.2cm}

\addcontentsline{toc}{section}{Appendices}
\renewcommand{\thesubsection}{\Alph{subsection}}

\subsection{Installation Template}  \label{appendix:template_script}

This appendix highlights the availability of a companion installation script
that is included with \OHPC{} documentation.  This script, when combined with
local site inputs, can be used to implement a starting recipe for
bare-metal system installation and configuration. This template script is used
during validation efforts to test cluster installations and is provided as a
convenience for administrators as a starting point for potential site
customization. 

The template script relies on the use of a simple text file to
define local site variables that were outlined in \S\ref{sec:inputs}.
% The collection of command-line instructions that are in this guide, when
% combined with local site inputs,  To aid in direct usage of the
% commands called out in this particular recipe, and to also allow for potential
% site customization, the \OHPC{} documentation package includes a template script
% summarizing the commands used herein. This script can be used in conjunction
% with a simple text file to define the local site variables defined in the
% previous section (
By default, the template install script attempts to use local variable settings
sourced from the \path{/opt/ohpc/pub/doc/recipes/vanilla/input.local} file,
however, this choice can be overridden by the use of the
\texttt{OHPC\_INPUT\_LOCAL} environment variable. The template install script is
intended for execution on the SMS {\em master} host and is installed as part of
the \texttt{docs-ohpc} package into \path{/opt/ohpc/pub/doc/recipes/vanilla/recipe.sh}.
After enabling the \OHPC{} repository and reviewing the guide for additional information on the intent of the
commands, the general starting approach for using this template is as follows:

\begin{enumerate}
\item Install the \texttt{docs-ohpc} package

\begin{lstlisting}[language=bash,keywords={}]
[master]# (*\install*) docs-ohpc
\end{lstlisting}

\item Copy the provided template input file to use as a starting point to
  define local site settings:
\begin{lstlisting}
[master]# cp  /opt/ohpc/pub/doc/recipes/vanilla/input.local input.local
\end{lstlisting}

\item Update \path{input.local} with desired settings

\item Copy the template installation script which contains command-line
  instructions culled from this guide.

\begin{lstlisting}[language=bash,keywords={}]
[master]# cp -p /opt/ohpc/pub/doc/recipes/vanilla/recipe.sh .
\end{lstlisting}

\item Review and edit \path{recipe.sh} to suite.

\item Use environment variable to define local input file and execute
  \path{recipe.sh} to perform a local installation.

\begin{lstlisting}[language=bash,keywords={}]
[master]# export OHPC_INPUT_LOCAL=./input.local
[master]# ./recipe.sh
\end{lstlisting}
\end{enumerate}




\subsection{RPM Rebuild}  \label{appendix:rpmbuild}

Users of OpenHPC may find it necessary to rebuild one of the packages to satisfy
local requirements. A simple way to accomplish this is to install the provided
source RPM, modify the specfile, and rebuild the binary RPM. A brief example
follows.

\begin{lstlisting}[language=bash,keywords={},basicstyle=\fontencoding{T1}\footnotesize\ttfamily,
    literate={VER}{\OHPCVersion{}}1 {OSREPO}{\OSRepo{}}1 {-}{-}1]
# Install rpm-build package from base OS distro
[test@sms ~]$ (*\install*) rpm-build

# Download SRPM from OpenHPC repository and install locally
[test@sms ~]$ wget \
  http://build.openhpc.community/OpenHPC:/VER:/Factory/OSREPO/src/fftw-gnu-openmpi-ohpc-3.3.4-4.1.src.rpm
[test@sms ~]$ rpm -i fftw-gnu-openmpi-ohpc-3.3.4-4.1.src.rpm

# Modify spec file as desired
[test@sms ~]$ cd ~/rpmbuild/SPECS
[test@sms ~rpmbuild/SPECS]$ perl -pi -e "s/enable-static=no/enable-static=yes/" fftw.spec

# Rebuild binary RPM. Note that additional directives can be specified to modify build
[test@sms ~rpmbuild/SPECS]$ rpmbuild -bb --define "compiler_family intel" fftw.spec

# As privileged user, install the new package
[sms](*\#*) (*\install*) ~test/rpmbuild/RPMS/x86_64/fftw-intel-openmpi-ohpc-3.3.4-4.1.rpm
\end{lstlisting}

\clearpage

\appendix
\section*{Appendix - Package Manifest}
\addcontentsline{toc}{section}{Appendix - Package Manifest}
\renewcommand{\thesubsection}{\Alph{subsection}}

%\subsection{Package Manifest}

This appendix provides a summary the underlying RPM packages that are available
as part of this \FSP{} release. These packages are presented in groupings
based on their general functionality that are organized as follows:

\begin{itemize*}
\item Administrative tools
\item Provisioning
\item Resource management
\item Compiler families
\item MPI families
\item Development tools
\item Performance analysis tools
\item Distro support packages and dependencies
\item IO Libraries
\item Serial Libraries
\item Parallel Libraries
\end{itemize*}

What follows in this Appendix are tables that summarize the packages in each
group including information on the RPM name, version, brief summary, and the web
URL where additional information can be contained for the component. Many of the 3rd
party community libraries that are pre-packaged with \FSP{} are built using
multiple compiler and MPI families. In these cases, the RPM package name
includes delimiters identifying the development environment for which each
package build is targeted.  Additional information on the \FSP{} package
naming scheme is presented in \S\ref{sec:3rdparty}.


\definecolor{Gray}{gray}{0.5}

\newcommand{\firstColWidth}{3.5cm}
\newcommand{\secondColWidth}{1.5cm}

% Administration Tools 

\captionsetup{justification=raggedright,singlelinecheck=false}

\vspace*{1.0cm}

\newcommand{\captionSpace}{-0.15cm}
\newcommand{\tabSpaceBot}{1.0cm}

\begin{table}[h]
\caption{\bf Administrative Tools} \vspace*{\captionSpace{}}
\input data/manifest/admin
\end{table}
\vspace*{0.5cm}

\renewcommand{\firstColWidth}{4.5cm}
\renewcommand{\secondColWidth}{2.0cm}

% Provisioning

\begin{table}[h!]
\caption{\bf Provisioning} \vspace*{\captionSpace{}}
\input data/manifest/provisioning
\vspace*{\tabSpaceBot{}}
\end{table} 

% Resource Management
\begin{table}[h!]
\caption{\bf Resource Management} \vspace*{\captionSpace{}}
\input data/manifest/rms
\vspace*{\tabSpaceBot{}}
\end{table}

% Compiler Families
\begin{table}[h!]
\caption{\bf Compiler Families} \vspace*{\captionSpace{}}
\input data/manifest/compiler-families
\vspace*{\tabSpaceBot{}}
\end{table}

% MPI Families
\begin{table}[h!]
\caption{\bf MPI Families} \vspace*{\captionSpace{}}
\input data/manifest/mpi-families
\vspace*{\tabSpaceBot{}}
\end{table}

\renewcommand{\firstColWidth}{4.5cm}

% Development Tools
\begin{table}[h!]
\caption{\bf Development Tools} \vspace*{\captionSpace{}}
\input data/manifest/dev-tools
\vspace*{\tabSpaceBot{}}
\end{table}

% Perf Tools
\begin{table}[h!]
\caption{\bf Performance Analysis Tools} \vspace*{\captionSpace{}}
\input data/manifest/perf-tools
\vspace*{\tabSpaceBot{}}
\end{table}

% Distro Packages
\begin{table}[h!]
\caption{\bf Distro Support Packages/Dependencies} \vspace*{\captionSpace{}}
\input data/manifest/distro-packages
\vspace*{\tabSpaceBot{}}
\end{table}

\renewcommand{\firstColWidth}{4.75cm}
\renewcommand{\secondColWidth}{1.75cm}

% Lustre
\begin{table}[h!]
\caption{\bf Lustre} \vspace*{\captionSpace{}}
\input data/manifest/lustre
\vspace*{\tabSpaceBot{}}
\end{table}

% IO Libs
\begin{table}[h!]
\caption{\bf IO Libraries} \vspace*{\captionSpace{}}
\input data/manifest/io-libs
\vspace*{\tabSpaceBot{}}
\end{table}

% Serial libs
\begin{table}[h!]
\caption{\bf Serial Libraries} \vspace*{\captionSpace{}} 
\input data/manifest/serial-libs
\vspace*{\tabSpaceBot{}}
\end{table}

% Parallel libs
\begin{table}[h!]
\caption{\bf Parallel Libraries} \vspace*{\captionSpace{}} 
\input data/manifest/parallel-libs
\vspace*{\tabSpaceBot{}}
\end{table}






\clearpage
\subsection{Package Signatures}
%\addcontentsline{toc}{section}{Appendix B - Package Signatures}

All of the RPMs provided via the \OHPC{} repository are signed with a GPG
signature. By default, the underlying package managers will verify these signatures during
installation to ensure that packages have not been altered. The RPMs can also
be manually verified and the public signing key fingerprint for the latest
repository is shown below: \\

\texttt{Fingerprint: 5392 744D 3C54 3ED5 7847  65E6 8A30 6019 {\bf DA565C6C}} \\

\noindent The following command can be used to verify an RPM once it
has been downloaded locally by confirming if the package is signed, and if so,
indicating which key was used to sign it. The example below highlights usage
for a local copy of the \texttt{docs-ohpc} package and illustrates how the {\em
key ID} matches the fingerprint shown above.

\begin{lstlisting}[language=bash,keywords={}]
[sms](*\#*) rpm --checksig -v docs-ohpc-*.rpm
docs-ohpc-2.0.0-72.1.ohpc.2.0.x86_64.rpm:
    Header V3 RSA/SHA1 Signature, key ID da565c6c: OK
    Header SHA256 digest: OK
    Header SHA1 digest: OK
    Payload SHA256 digest: OK
    V3 RSA/SHA1 Signature, key ID da565c6c: OK
    MD5 digest: OK

\end{lstlisting}






\end{document}

