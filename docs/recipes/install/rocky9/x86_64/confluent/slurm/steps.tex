\documentclass[letterpaper]{article}
\usepackage{common/ohpc-doc}
\setcounter{secnumdepth}{5}
\setcounter{tocdepth}{5}

% Include git variables
%%% This file has been generated by the vc bundle for TeX.
%%% Do not edit this file!
%%%
%%% Define Git specific macros.
\gdef\GITHash{ab0141b6f1cd0193472944a1642275fb792a8f77}%
\gdef\GITAbrHash{ab0141b}%
\gdef\GITParentHashes{765539311c92eca445116d4e451fa2464e75246f}%
\gdef\GITAbrParentHashes{7655393}%
\gdef\GITAuthorName{Karl W. Schulz}%
\gdef\GITAuthorEmail{karl.w.schulz@intel.com}%
\gdef\GITAuthorDate{2015-04-17 16:00:11 -0500}%
\gdef\GITCommitterName{Karl W. Schulz}%
\gdef\GITCommitterEmail{karl.w.schulz@intel.com}%
\gdef\GITCommitterDate{2015-04-17 16:00:11 -0500}%
%%% Define generic version control macros.
\gdef\VCRevision{\GITAbrHash}%
\gdef\VCAuthor{\GITAuthorName}%
\gdef\VCDateRAW{2015-04-17}%
\gdef\VCDateISO{2015-04-17}%
\gdef\VCDateTEX{2015/04/17}%
\gdef\VCTime{16:00:11 -0500}%
\gdef\VCModifiedText{\textcolor{red}{with local modifications!}}%
%%% Assume clean working copy.
\gdef\VCModified{0}%
\gdef\VCRevisionMod{\VCRevision}%


% Define Base OS and other local macros
\newcommand{\baseOS}{Rocky 9.4}
\newcommand{\OSRepo}{Rocky\_9.4}
\newcommand{\OSTree}{EL\_9}
\newcommand{\OSTag}{el9}
\newcommand{\baseos}{rocky9.4}
\newcommand{\baseosshort}{rocky9}
\newcommand{\provisioner}{confluent}
\newcommand{\provheader}{\provisioner{}}
\newcommand{\rms}{SLURM}
\newcommand{\rmsshort}{slurm}
\newcommand{\arch}{x86\_64}

% Define package manager commands
\newcommand{\pkgmgr}{dnf}
\newcommand{\addrepo}{wget -P /etc/yum.repos.d}
\newcommand{\chrootaddrepo}{wget -P \$CHROOT/etc/yum.repos.d}
\newcommand{\clean}{dnf clean expire-cache}
\newcommand{\chrootclean}{dnf --installroot=\$CHROOT clean expire-cache}
\newcommand{\install}{dnf -y install}
\newcommand{\chrootinstall}{nodeshell compute dnf -y install}
\newcommand{\groupinstall}{dnf -y groupinstall}
\newcommand{\groupchrootinstall}{nodeshell compute dnf -y groupinstall}
\newcommand{\remove}{dnf -y remove}
\newcommand{\upgrade}{dnf -y upgrade}
\newcommand{\chrootupgrade}{dnf -y --installroot=\$CHROOT upgrade}
\newcommand{\tftppkg}{syslinux-tftpboot}
\newcommand{\beegfsrepo}{https://www.beegfs.io/release/beegfs\_7.2.1/dists/beegfs-rhel8.repo}

% boolean for os-specific formatting
\toggletrue{isCentOS}
\toggletrue{isCentOS_ww_slurm_x86}
\toggletrue{isSLURM}
\toggletrue{isx86}
\toggletrue{isCentOS_x86}

\begin{document}
\graphicspath{{common/figures/}}
\thispagestyle{empty}

% Title Page
% Title page and running header definition

\lhead{ \small {\color{logodarkgrey}\fontfamily{phv}\selectfont { Install Guide
    (v\OHPCVersion{})}:  {\baseOS{}/\arch{} + \provheader{} + \rms{}} } \vspace*{0pt} }

{\hspace*{4in} \includegraphics[width=1.7in]{ohpc_logo_blue.pdf}}

\vspace*{2cm}
\noindent {\LARGE \color{logodarkgrey} \fontfamily{phv}\selectfont OpenHPC (v\OHPCVersion{})} \vspace*{0.1cm} \\
\noindent {\LARGE \color{logodarkgrey} \fontfamily{phv}\selectfont Cluster Building Recipes} \\ 

{\color{logoblue}\noindent\rule{6.15in}{1.2pt}} \\ 

\noindent {\Large \color{logodarkgrey} \fontfamily{phv}\selectfont \baseOS{} Base OS} \\ 

\noindent{\Large\color{logodarkgrey}\fontfamily{phv}\selectfont{\provisioner{}/\rms{}
Edition for Linux*} (\arch{})} \\

{\color{logoblue}\noindent\rule{6.15in}{1.2pt}} \\ \vspace{0.2cm}  

\iftoggleverb{isxCATstateful}
\vspace*{0.25cm}
\noindent{\large\color{logodarkgrey}\fontfamily{phv}\selectfont{Stateful
Provisioning}}
\fi

\vspace*{2in}

\noindent{\small \color{black} Document Last Update: \VCDateISO} \vspace*{0.1cm} \\ 
{\small \color{black} Document Revision: \VCRevision} \\ \vspace*{0.1cm}

% Disclaimer
\newpage

\vspace*{5.0cm}
\noindent {\Large \color{RoyalBlue} \fontfamily{phv}\selectfont Disclaimer and Legal Information} \\ 

{\footnotesize

\noindent NO LICENSE (EXPRESS OR IMPLIED, BY ESTOPPEL OR OTHERWISE) TO ANY INTELLECTUAL
PROPERTY RIGHTS IS GRANTED BY THIS DOCUMENT. THIS DOCUMENT MAY CONTAIN
INFORMATION ON PRODUCTS, SERVICES AND/OR PROCESSES IN DEVELOPMENT. ALL
INFORMATION PROVIDED HERE IS SUBJECT TO CHANGE WITHOUT NOTICE. CONTACT YOUR
INTEL REPRESENTATIVE TO OBTAIN THE LATEST FORECAST, SCHEDULE, SPECIFICATIONS
AND ROADMAPS. \\

\noindent Intel technologies features and benefits depend on system
configuration and may require enabled hardware, software or service
activation. \\

\noindent No computer system can be absolutely secure. Intel does not control
or audit third-party benchmark data or the web sites referenced in this
document. You should visit the referenced web site and confirm whether
referenced data are accurate.  \\

\noindent Learn more at intel.com, or from the OEM or retailer. \\

\noindent The products described may contain design defects or errors known as
errata which may cause the product to deviate from published
specifications. Current characterized errata are available on request. \\

\noindent Intel, the Intel logo, and others are trademarks of Intel Corporation
in the U.S. and/or other countries. \\

\noindent *Other names and brands may be claimed as the property of others. \\

\noindent {\small\copyright} 2015 Intel Corporation

}


\newpage
\tableofcontents
\newpage

% Introduction  --------------------------------------------------

\section{Introduction} \label{sec:introduction}
% begin_ohpc_run
% ohpc_validation_comment -----------------------------------------------------------------------------------------
% ohpc_validation_comment  Example Installation Script Template
% ohpc_validation_comment  
% ohpc_validation_comment  This convenience script encapsulates command-line instructions highlighted in
% ohpc_validation_comment  an OpenHPC Install Guide that can be used as a starting point to perform a local
% ohpc_validation_comment  cluster install beginning with bare-metal. Necessary inputs that describe local
% ohpc_validation_comment  hardware characteristics, desired network settings, and other customizations
% ohpc_validation_comment  are controlled via a companion input file that is used to initialize variables 
% ohpc_validation_comment  within this script.
% ohpc_validation_comment   
% ohpc_validation_comment  Please see the OpenHPC Install Guide(s) for more information regarding the
% ohpc_validation_comment  procedure. Note that the section numbering included in this script refers to
% ohpc_validation_comment  corresponding sections from the companion install guide.
% ohpc_validation_comment -----------------------------------------------------------------------------------------
% ohpc_validation_newline

% ohpc_command inputFile=${OHPC_INPUT_LOCAL:-/opt/ohpc/pub/doc/recipes/BOSSHORT/input.local}
% ohpc_validation_newline
% ohpc_command if [ ! -e ${inputFile} ];then
% ohpc_command    echo "Error: Unable to access local input file -> ${inputFile}"
% ohpc_command    exit 1
% ohpc_command else
% ohpc_command    . ${inputFile} || { echo "Error sourcing ${inputFile}"; exit 1; }
% ohpc_command fi

% ohpc_validation_newline
% ohpc_validation_comment ---------------------------- Begin OpenHPC Recipe ---------------------------------------
% ohpc_validation_comment Commands below are extracted from an OpenHPC install guide recipe and are intended for 
% ohpc_validation_comment execution on the master SMS host.
% ohpc_validation_comment -----------------------------------------------------------------------------------------

% end_ohpc_run

This guide presents a simple cluster installation procedure using components
from the Forest Peak (\FSP{}) software stack. \FSP{} represents an aggregation of a number of
common ingredients required to deploy and manage an HPC Linux* cluster including
provisioning tools, resource management, I/O clients, development tools, and a variety of
scientific libraries. These packages have been pre-built with HPC integration
in mind and represent a mix of open-source components combined with \Intel{}
development and analysis tools (e.g. \Intel{} Parallel Studio XE Cluster Edition).
%This install guide assumes availablility of a {\em master} host
%that will have access to common OS repositories and the \FSP{} repository in order
%to facilitate software installs and dependency resolution.  
The documentation herein is intended to be reasonably generic, but uses the
underlying motivation of a small, 4-node cluster install to define a step-by-step
process. Several optional customizations are included and the intent is that
these collective instructions can be modified as needed for local site
customizations. This guide is targeted at veteran Linux* system administrators. Knowledge of 
software package management, system networking, and PXE booting is assumed. Shell 
examples in the text are givin in BASH, though the equivilent commands in other
shells should work fine. Command-line inputs in this guide are written as follows:

\begin{lstlisting}[language=bash,literate={-}{-}1,keywords={},upquote=true]
[master]$ rm -rf /.
\end{lstlisting}
 \\

\noindent {\bf Base Linux Edition}: this edition of the guide highlights
installation without the use of a companion configuration management system and
directly uses distro-provided package management tools for component
selection. The steps that follow also highlight specific changes to system
configuration files that are required as part of the cluster install
process.
%Other editions of this guide provide similar install steps when using
%specific configuration management systems that can simplify the installation
%and configuration process.

\subsection{Target Audience}

This guide is targeted at experienced \Linux{} system administrators for HPC
environments. Knowledge of software package management, system networking, and
PXE booting is assumed.  Command-line input examples are highlighted throughout
this guide via the following syntax:

\begin{lstlisting}[language=bash,literate={-}{-}1,keywords={},upquote=true]
[master](*\#*) echo "OpenHPC hello world"
\end{lstlisting}

Unless specified otherwise, the examples presented are executed with
elevated (root) privileges. The examples also presume use of the BASH login
shell, though the equivalent commands in other shells can be substituted.

Usefull tips and common pitfalls are highlighted in this guide with the
following syntax:

\begin{center}
\begin{tcolorbox}[]
\small
If you don't know where you are going, you might wind up someplace else.
--Yogi Berra
\end{tcolorbox}
\end{center}


%\noindent {\bf Requirements/Assumptions}: 
\subsection{Requirements/Assumptions}
This installation recipe assumes the availability of a single head node {\em
master}, and four {\em compute} nodes. The {\em master} node serves as the
overall system management server (SMS) and is provisioned with \baseOS{} and is
subsequently configured to provision the remaining {\em compute} nodes with
\Warewulf{} in a stateless configuration. For power management, we assume that the
compute node BMCs are available via IPMI from the chosen master host. For file
systems, we assume that the chosen master server will host an \NFS{} file
system that is made available to the compute nodes. Installation information is
also discussed to optionally include a \Lustre{} file system mount and in this
case, the \Lustre{} file system is assumed to exist previously. 

\begin{figure}[hbt]
\center
\includegraphics[width=0.8\linewidth]{fsp-arch-small.pdf}
\vspace*{-0.2cm}
\caption{Overview of physical cluster architecture.} \label{fig:physical_arch}
\end{figure}
\mbox{}

An outline of the physical architecture discussed is shown in
Figure~\ref{fig:physical_arch} and it highlights the high-level networking
configuration. The master hosts requires at least two ethernet interfaces with
{\em eth0} connected to the local data center network and {\em eth1} used to
provision and manage the cluster backend.  Two logical tcp interfaces are
expected to each compute node: the first is the standard ethernet interface
that will be used for provisioning and resource management. The second is
used to connect to the hosts BMC and is used for power management and
remote console access.  In addition to the tcp networking, there is a
high-speed network (\InfiniBand{} in this recipe) that is also connected to
each of the hosts. This high speed network is used for application message
passing and optionally for \Lustre{} connectivity as well.


\noindent {\bf Inputs}: since this recipe details installing a cluster
starting from bare-metal, there is a requirement to define IP addresses and gather
hardware MAC addresses in order to support controlled provisioning. These values
are unique to the hardware being used, and this document uses \texttt{<variable>}
names in the command-line examples that follow to highlight where local site
inputs are required. A summary of the required variables used in this recipe
are as follows: \\

\vspace*{0.2cm}
\begin{tabular}{@{}>{\textbullet}cll@{}}
& \texttt{<master\_name>}  & Hostname for master server \\
& \texttt{<master\_ip>} & IP address on master server for provisioning interface \\
& \texttt{<internal\_netmask>} & Subnet netmask for internal cluster network \\
& \texttt{<c1\_ip>},\texttt{<c2\_ip>},\texttt{<c3\_ip>},\texttt{<c4\_ip>}
& Desired compute node addresses \\
& \texttt{<c1\_bmc>},\texttt{<c2\_bmc>},\texttt{<c3\_bmc>},\texttt{<c4\_bmc>}
& BMC addresses for computes \\
& \texttt{<c1\_mac>},\texttt{<c2\_mac>},\texttt{<c3\_mac>},\texttt{<c4\_mac>}
& MAC addresses for computes \\
\end{tabular}




% Base Operating System --------------------------------------------

\section{Install Base Operating System (BOS)}
In an external setting, installing the desired BOS on a
{\em master} SMS host typically involves booting from a DVD ISO image on a new
server. With this approach, insert the \baseOS{} DVD, power cycle the host, and
follow the distro provided directions to install the BOS on your chosen {\em
master} host.  Alternatively, if choosing to use a pre-installed server, please
verify that it is provisioned with the required \baseOS{} distribution. \\

\ifnottoggleverb{isWarewulf4}
Prior to beginning the installation process of \OHPC{} components, several additional
considerations are noted here for the SMS host configuration. First,
the installation recipe herein assumes that
the SMS host name is resolvable locally. Depending on the manner in which you
installed the BOS, there may be an adequate entry already defined
in \path{/etc/hosts}. If not, the following addition can be used to identify
your SMS host.
\begin{lstlisting}[language=bash,keywords={}]
[sms](*\#*) echo ${sms_ip} ${sms_name} >> /etc/hosts
\end{lstlisting}
\fi

While it is theoretically possible to enable SELinux on a cluster provisioned
with \provisioner{},
doing so is beyond the scope of this document. Even the use of
permissive mode can be problematic and we therefore recommend disabling SELinux on the {\em
master} SMS host. If SELinux components are installed locally,
the \texttt{selinuxenabled} command can be used to determine if SELinux is
currently enabled. If enabled, consult the distro documentation for information
on how to disable. \\

Finally, provisioning services rely on DHCP, TFTP, and HTTP network protocols.
Depending on the local BOS configuration on the SMS host, default firewall
rules may prohibit these services. Consequently, this recipe assumes that the
local firewall running on the SMS host is disabled (it is still recommended to
have additional security boundaries like a firewall to protect the cluster from
the Internet). If installed, the default firewall service can be disabled as
follows:


%\clearpage
% begin_ohpc_run
% ohpc_validation_newline
% ohpc_validation_comment Disable firewall
\begin{lstlisting}[language=bash,keywords={}]
[sms](*\#*) systemctl disable firewalld
[sms](*\#*) systemctl stop firewalld
\end{lstlisting}
% end_ohpc_run

% ------------------------------------------------------------------

\section{Install \Confluent{} and Provision Nodes with BOS} \label{sec:provision_compute_bos}
Installation is accomplished in two steps: First,  a generic OS
image is installed on {\em compute} nodes and then, once the nodes are up
and running, \OHPC{} components are added to both the SMS and the nodes at the
same time.

\subsection{Enable \Confluent{} repository for local use} \label{sec:enable_confluent}
To begin, enable use of the public \Confluent{} repository by adding it to the local list
of available package repositories. This also requires network access from
your {\em master} server to the internet, or alternatively, that
the repository be mirrored locally. In this case, we use the network.

% begin_ohpc_run
% ohpc_validation_newline
% ohpc_comment_header Enable Confluent repositories \ref{sec:enable_confluent}
\begin{lstlisting}[language=bash,keywords={},basicstyle=\fontencoding{T1}\fontsize{8.0}{10}\ttfamily,
	literate={VER}{\OHPCVerTree{}}1 {OSREPO}{\OSTree{}}1 {TAG}{\OSTag{}}1 {ARCH}{\arch{}}1 {-}{-}1]
[sms](*\#*) (*\install*) https://hpc.lenovo.com/yum/latest/el9/x86_64/lenovo-hpc-yum-1-1.x86_64.rpm
\end{lstlisting}
% end_ohpc_run

\subsection{Add provisioning services on {\em master} node} \label{sec:add_provisioning}
% -*- mode: latex; fill-column: 120; -*-
With \Confluent{} repository enabled, issue the following install the provisioning
service on {\em master} node

% begin_ohpc_run
% ohpc_comment_header Add baseline OpenHPC and provisioning services \ref{sec:add_provisioning}
\begin{lstlisting}[language=bash,keywords={}]
[sms](*\#*) (*\install*) lenovo-confluent
[sms](*\#*) (*\install*) tftp-server

# enable Confluent and its tools for use in current shell
[sms](*\#*) systemctl enable confluent --now
[sms](*\#*) systemctl enable httpd --now
[sms](*\#*) systemctl enable tftp.socket --now
[sms](*\#*) source /etc/profile.d/confluent_env.sh
\end{lstlisting}
% ohpc_validation_newline
% end_ohpc_run

\vspace*{-0.15cm}
\subsection{Complete basic \Confluent{} setup for {\em master} node} \label{sec:setup_confluent}
At this point, all of the packages necessary to use \Confluent{} on the {\em master}
host should be installed. Next, we enable support for local provisioning using
a second private interface (refer to Figure~\ref{fig:physical_arch})

% begin_ohpc_run
% ohpc_comment_header Complete basic Confluent setup for master node \ref{sec:setup_confluent}
%\begin{verbatim}
\begin{lstlisting}[language=bash,literate={-}{-}1,keywords={},upquote=true,keepspaces]
# Enable internal interface for provisioning
[sms](*\#*) ip link set dev ${sms_eth_internal} up
[sms](*\#*) ip address add ${sms_ip}/${internal_netmask} broadcast + dev ${sms_eth_internal}

\end{lstlisting}
%\end{verbatim}
% end_ohpc_run


\noindent \Confluent{} requires a network domain name specification for system-wide name
resolution. This value can be set to match your local DNS schema or given a
unique identifier such as `local`. A default group called everything is 
automatically added to every node. It provides a method to indicate global settings.
Attributes may all be specified on the command line, and an example set could be:

% begin_ohpc_run
% ohpc_validation_newline
% ohpc_validation_comment Define local domainname, deployment protocol and dns 
\begin{lstlisting}[language=bash,keywords={},upquote=true,basicstyle=\footnotesize\ttfamily,literate={BOSVER}{\baseos{}}1]
[sms](*\#*) nodegroupattrib everything deployment.useinsecureprotocols=${deployment_protocols} dns.domain=${dns_domain} 
[sms](*\#*) nodegroupattrib everything dns.servers=${dns_servers} net.ipv4_gateway=${ipv4_gateway}
\end{lstlisting}

\noindent We will also define 

\subsection{Define {\em compute} image for provisioning}
% -*- mode: latex; fill-column: 120; -*-

With the provisioning services enabled, the next step is to define
a system image that can subsequently be
used to provision one or more {\em compute} nodes. The following subsections highlight this process.

\subsubsection{Build initial BOS image} \label{sec:assemble_bos}
The following steps illustrate the process to build a minimal, default image for use with \Confluent{}. To begin, you will
first need to have a local copy of the ISO image available for the underlying OS. In this recipe, the relevant ISO image
is \texttt{Rocky-9.4-x86\_64-dvd.iso} (available from the Rocky
\href{https://rockylinux.org/download/}{\color{blue}download} page).
We initialize the image
creation process using the \texttt{osdeploy} command assuming that the necessary ISO image is available locally in
\texttt{\$\{iso\_path\}} as follows:

The \texttt{osdeploy initialize} command is used to prepare a confluent server to deploy deploy operating systems.
For first time setup, run osdeploy initialize interactively to be walked through the various options using: 
\texttt{osdeploy initialize -i}

% begin_ohpc_run
% ohpc_comment_header Initialize OS images for use with Confluent \ref{sec:assemble_bos}
\begin{lstlisting}[language=bash,literate={-}{-}1,keywords={},upquote=true,keepspaces,literate={BOSVER}{\baseos{}}1]
[sms](*\#*) osdeploy initialize -${initialize_options}
[sms](*\#*) osdeploy import ${iso_path}

\end{lstlisting}
% end_ohpc_run

\noindent Once completed, OS image should be available for use within \Confluent{}. These can be queried via:

\begin{lstlisting}[language=bash,literate={-}{-}1,keywords={},upquote=true,keepspaces,literate={BOSVER}{\baseos{}}1]
# Query available images
[sms](*\#*) osdeploy list
Distributions:
  rocky-8.5-x86_64
  rocky-9.4-x86_64
Profiles:
  rhel-9.4-x86_64-default
  rocky-8.5-x86_64-default
\end{lstlisting}

If needing to copy files from the sms node to the compute nodes during deployment, this can be done by
modifying the syncfiles file that is created when \texttt{osdeploy import} command is run. For an environment
that has no DNS server and needs to have /etc/hosts file synced amongst all the nodes, the following command
should be run.   

% begin_ohpc_run
% ohpc_validation_newline
% ohpc_validation_comment Sync the hosts file in cluster
\begin{lstlisting}[language=bash,literate={-}{-}1,keywords={},upquote=true,keepspaces,literate={BOSVER}{\baseos{}}1]
  [sms](*\#*) echo "/etc/hosts -> /etc/hosts" >> /var/lib/confluent/public/os/rocky-9.4-x86_64-default/syncfiles

\end{lstlisting}
% end_ohpc_run
  

%The \texttt{CHROOT} environment variable highlights the path and is used by
%subsequent commands to augment the basic installation.



\vspace*{0.9cm}
\subsection{Add compute nodes into \Confluent{} database} \label{sec:confluent_add_nodes}
%\subsubsection{Register nodes for provisioning}

\noindent Next, we add {\em compute} nodes and define their properties as
attributes in \Confluent{} database.
These hosts are grouped logically into a group named {\em
compute} to facilitate group-level commands used later in the recipe. The compute
group has to be defined first before we can add any nodes to the group using the 
{\texttt nodegroup define} command. Note the
use of variable names for the desired compute hostnames, node IPs, MAC
addresses, and BMC login credentials, which should be modified to accommodate
local settings and hardware. To enable serial console access via  \Confluent{},
{\texttt console.method}
property is also defined. 

% begin_ohpc_run
% ohpc_validation_newline
% ohpc_validation_comment Add hosts to cluster \ref{sec:confluent_add_nodes}
\begin{lstlisting}[language=bash,keywords={},upquote=true,basicstyle=\footnotesize\ttfamily,]
#define the compute group
[sms](*\#*) nodegroupdefine compute

# Define nodes as objects in confluent database
[sms](*\#*) for ((i=0; i<$num_computes; i++)) ; do
nodedefine ${c_name[$i]} groups=everything,compute hardwaremanagement.manager=${c_bmc[$i]} \ 
secret.hardwaremanagementuser=${bmc_username} secret.hardwaremanagementpassword=${bmc_password} \ 
net.hwaddr=${c_mac[$i]} net.ipv4_address=${c_ip[$i]}
        done
\end{lstlisting}
% end_ohpc_run

\begin{center}
  \begin{tcolorbox}[]
    \small
Defining nodes one-by-one, as done above, is only efficient
for a small number of nodes. For larger node counts,
\Confluent{} provides capabilities for automated detection and
configuration.
Consult the
\href{https://hpc.lenovo.com/users/documentation/confluentdisco.html}{\color{blue}\Confluent{}
Hardware Discovery \& Define Node Guide}.
\end{tcolorbox}
\end{center}


%\clearpage
If enabling {\em optional} IPoIB functionality (e.g. to support Lustre over \InfiniBand{}), additional
settings are required to define the IPoIB network with \Confluent{} and specify
desired IP settings for each compute. This can be accomplished as follows for
the {\em ib0} interface:

% begin_ohpc_run
% ohpc_validation_newline
% ohpc_validation_comment Setup IPoIB networking
% ohpc_command if [[ ${enable_ipoib} -eq 1 ]];then
% ohpc_indent 5
\begin{lstlisting}[language=bash,keywords={},upquote=true,basicstyle=\footnotesize\ttfamily]
# Register desired IPoIB IPs per compute
[sms](*\#*) for ((i=0; i<$num_computes; i++)) ; do
		nodeattrib ${c_name[i]} net.ib0.ipv4_address=${c_ipoib[i]}/${ipoib_netmask}
        done
\end{lstlisting}
% ohpc_indent 0
% ohpc_command fi
% end_ohpc_run

%\clearpage
confluent2hosts can be used to help generate /etc/hosts entries for a noderange.
It can read from the confluent db, using -a. In this mode, each net.value.attribute
group is pulled together into hosts lines. ipv4 and ipv6 address
fields are associated with the corresponding hostname attributes.

% begin_ohpc_run
% ohpc_validation_newline
% ohpc_validation_comment Generating lines to append to local /etc/hosts 
\begin{lstlisting}[language=bash,keywords={},upquote=true,basicstyle=\footnotesize\ttfamily]
# Add nodes to /etc/hosts
[sms](*\#*) confluent2hosts -a compute
\end{lstlisting}
% end_ohpc_run

%\clearpage
With the desired compute nodes and domain identified, the remaining steps in the
provisioning configuration process are to define the provisioning mode and
image for the {\em compute} group and use \Confluent{} commands to complete
configuration for network services like DNS and DHCP. These tasks are
accomplished as follows:

%\clearpage
% begin_ohpc_run
% ohpc_validation_newline
% ohpc_validation_comment Complete networking setup, associate provisioning image
\begin{lstlisting}[language=bash,keywords={},upquote=true,basicstyle=\footnotesize\ttfamily,literate={BOSSHORT}{\baseosshort{}}1 {IMAGE}{\installimage{}}1]
# Associate desired provisioning image for computes
[sms](*\#*) nodedeploy -n compute -p rocky-9.4-x86_64-default
\end{lstlisting}

%%% If the Lustre client was enabled for computes in \S\ref{sec:lustre_client}, you
%%% should be able to mount the file system post-boot using the fstab entry
%%% (e.g. via ``\texttt{mount /mnt/lustre}''). Alternatively, if
%%% you prefer to have the file system mounted automatically at boot time, a simple
%%% postscript can be created and registered with \xCAT{} for this purpose as follows.
%%%
%%% % begin_ohpc_run
%%% % ohpc_validation_newline
%%% % ohpc_validation_comment Optionally create xCAT postscript to mount Lustre client
%%% % ohpc_command if [ ${enable_lustre_client} -eq 1 ];then
%%% % ohpc_indent 5
%%% \begin{lstlisting}[language=bash,keywords={},upquote=true,basicstyle=\footnotesize\ttfamily,literate={BOSVER}{\baseos{}}1]
%%% # Optionally create postscript to mount Lustre client at boot
%%% [sms](*\#*) echo '#!/bin/bash' > /install/postscripts/lustre-client
%%% [sms](*\#*) echo 'mount /mnt/lustre' >> /install/postscripts/lustre-client
%%% [sms](*\#*) chmod 755 /install/postscripts/lustre-client
%%% # Register script for computes
%%% [sms](*\#*) chdef compute -p postscripts=lustre-client
%%% \end{lstlisting}
%%% % ohpc_indent 0
%%% % ohpc_command fi
%%% % end_ohpc_run
%%%



%\vspace*{-0.25cm}
\subsection{Boot compute nodes} \label{sec:boot_computes}
Prior to booting the {\em compute} hosts, we configure them to use PXE as their
next boot mode. After the initial PXE, ensuing boots will return to using the default boot device
specified in the BIOS.

% begin_ohpc_run
% ohpc_comment_header Set nodes to netboot \ref{sec:boot_computes}
\begin{lstlisting}[language=bash,keywords={},upquote=true]
[sms](*\#*) nodesetboot compute network
\end{lstlisting}
% end_ohpc_run

At this point, the {\em master} server should be able to boot the newly defined
compute nodes. This is done by using the \texttt{nodepower} \Confluent{} command
leveraging IPMI protocol set up during the the {\em compute} node definition
in \S~\ref{sec:confluent_add_nodes}. The following power cycles each of the
desired hosts.


% begin_ohpc_run
% ohpc_comment_header Boot compute nodes
\begin{lstlisting}[language=bash,keywords={},upquote=true]
[sms](*\#*) nodepower compute boot
\end{lstlisting}
% end_ohpc_run

Once kicked off, the boot process should take about 5-10
minutes (depending on BIOS post times).  You can monitor the
provisioning by using the \texttt{nodeconsole} command, which displays serial console
for a selected node. Note that the escape sequence
is \texttt{CTRL-e c .} typed sequentially.

Successful provisioning can be verified by a parallel command on the compute
nodes. The \Confluent{}-provided
\texttt{nodeshell} command, which uses \Confluent{} node names and groups.  
For example, to run a command on
the newly imaged compute hosts using \texttt{nodeshell}, execute the following:

\begin{lstlisting}[language=bash]
[sms](*\#*) nodeshell compute uptime
c1:  12:56:50 up 14 min,  0 users,  load average: 0.00, 0.01, 0.04
c2:  12:56:50 up 13 min,  0 users,  load average: 0.00, 0.02, 0.05
c3:  12:56:50 up 14 min,  0 users,  load average: 0.00, 0.02, 0.05
c4:  12:56:50 up 14 min,  0 users,  load average: 0.00, 0.01, 0.04
\end{lstlisting}


% begin_ohpc_run
% ohpc_validation_newline
% ohpc_validation_comment Check if provisioning is in progress
% ohpc_validation_newline
% ohpc_command while true; do
% ohpc_command    deployment_pending=false
% ohpc_command    while read line; do
% ohpc_command        dp=$(echo $line | cut -d ":" -f 2)
% ohpc_command          if [ "$dp" != " completed" ];
% ohpc_command          then
% ohpc_command               deployment_pending=true
% ohpc_command          fi
% ohpc_command    done <<< `nodedeploy compute`
% ohpc_command    if $deployment_pending;
% ohpc_command     then
% ohpc_command          echo "deployment still pending"
% ohpc_command          deployment_pending=false
% ohpc_command          sleep 60
% ohpc_command     else
% ohpc_command          break
% ohpc_command     fi
% ohpc_command done
% end_ohpc_run


\section{Install \OHPC{} Components} \label{sec:basic_install}
With the BOS installed and booted, the next step is to add desired \OHPC{} packages
onto the {\em master} server in order to provide provisioning and resource
management services for the rest of the cluster. The following subsections
highlight this process.


\subsection{Enable \OHPC{} repository for local use} \label{sec:enable_repo}
To begin, enable use of the \OHPC{} repository by adding it to the local list
of available package repositories. Note that this requires network access from
your {\em master} server to the \OHPC{} repository, or alternatively, that
the \OHPC{} repository be mirrored locally.  In cases where network external
connectivity is available, \OHPC{} provides an \texttt{ohpc-release} package
that includes GPG keys for package signing and enabling the repository.  The
example which follows illustrates installation of the \texttt{ohpc-release}
package directly from the \OHPC{} build server.

% begin_ohpc_run
% ohpc_validation_comment Add OpenHPC components to compute instance
\begin{lstlisting}[language=bash,literate={-}{-}1,keywords={},upquote=true]
# Add OpenHPC repo 
[sms](*\#*) (*\install*) http://repos.openhpc.community/OpenHPC/3/EL_9/x86_64/ohpc-release-3-1.el9.x86_64.rpm
\end{lstlisting}
% end_ohpc_run

\begin{center}
\begin{tcolorbox}[]
\small Many sites may find it useful or necessary to maintain a local copy of the
\OHPC{} repositories. To facilitate this need, standalone tar
archives are provided -- one containing a repository of binary packages as well as any
available updates, and one containing a repository of source RPMS. The tar files
also contain a simple bash script to configure the package manager to use the
local repository after download. To use, simply unpack the tarball where you
would like to host the local repository and execute the \texttt{make\_repo.sh} script.
Tar files for this release can be found at \href{http://repos.openhpc.community/dist/\OHPCVersion}
        {\color{blue}{http://repos.openhpc.community/dist/\OHPCVersion}}
\end{tcolorbox}
\end{center}


% begin_ohpc_run
% ohpc_validation_newline
% ohpc_validation_comment Verify OpenHPC repository has been enabled before proceeding
% ohpc_validation_newline
% ohpc_command dnf repolist | grep -q OpenHPC
% ohpc_command if [ $? -ne 0 ];then
% ohpc_command    echo "Error: OpenHPC repository must be enabled locally"
% ohpc_command    exit 1
% ohpc_command fi
% end_ohpc_run

In addition to the \OHPC{} 
\iftoggle{isxCAT}{and \xCAT{} package repositories,}{package repository,}
the {\em master} host also requires access to the standard base OS distro
repositories in order to resolve necessary dependencies. For \baseOS{}, the
requirements are to have access to the BaseOS, Appstream, Extras, PowerTools,
and EPEL repositories for which mirrors are freely available online:

\begin{itemize*}
\item Rocky-8
  (e.g. \href{http://download.rockylinux.org/pub/rocky/8/}
             {\color{blue}{http://download.rockylinux.org/pub/rocky/8/}} )
\item EPEL 8 (e.g. \href{http://download.fedoraproject.org/pub/epel/8/}
                        {\color{blue}{http://download.fedoraproject.org/pub/epel/8/}} )
\end{itemize*}

\noindent The public EPEL repository will be enabled automatically upon
installation of the \texttt{ohpc-release} package. Note that this does depend
on the Rocky Extras repository, which is shipped with Rocky and is typically
enabled by default.  In contrast, the PowerTools repository is typically
disabled in a standard install, but can be enabled from EPEL as follows:

\begin{lstlisting}[language=bash,literate={-}{-}1,keywords={},upquote=true]
[sms](*\#*) yum install dnf-plugins-core
[sms](*\#*) yum config-manager --set-enabled powertools
\end{lstlisting}



% begin_ohpc_run
\begin{lstlisting}[language=bash,keywords={},basicstyle=\fontencoding{T1}\fontsize{8.0}{10}\ttfamily,literate={ARCH}{\arch{}}1 {-}{-}1]
[sms](*\#*) (*\install*) epel-release

# Enable crb on sms
[sms](*\#*) /usr/bin/crb enable
\end{lstlisting}
% end_ohpc_run

Now \OHPC{} packages can be installed. To add the base package on the SMS
issue the following
% begin_ohpc_run
\begin{lstlisting}[language=bash,keywords={},basicstyle=\fontencoding{T1}\fontsize{8.0}{10}\ttfamily,literate={ARCH}{\arch{}}1 {-}{-}1]
[sms](*\#*)  (*\install*) ohpc-base
\end{lstlisting}
% end_ohpc_run

\subsection{Installation template}
The collection of command-line instructions that follow in this guide, when
combined with local site inputs, can be used to implement a 
bare-metal system installation and configuration. The format of these commands
is intended to be usable via direct cut and paste (with variable substitution
for site-specific settings). Alternatively, the \OHPC{} documentation package
(\texttt{docs-ohpc}) includes a template script which includes a summary of all 
of the commands used herein. This script can be used in conjunction with a 
simple text file to define the local site variables defined in the previous 
section (\S~\ref{sec:inputs}) and is provided as a convenience for 
administrators. For additional information on accessing this script, please see
Appendix~\ref{appendix:template_script}.



\subsection{Setup time synchronization service on {\em master} node} \label{sec:add_ntp}
HPC systems typically rely on synchronized clocks throughout the system and the
NTP protocol can be used to facilitate this synchronization. To enable NTP
services on the SMS host with a specific server \texttt{\$\{ntp\_server\}},
issue the following:

% begin_ohpc_run
% ohpc_validation_comment Enable NTP services on SMS host
\begin{lstlisting}[language=bash,literate={-}{-}1,keywords={},upquote=true,keepspaces]
[sms](*\#*) systemctl enable ntpd.service
[sms](*\#*) echo "server ${ntp_server}" >> /etc/ntp.conf
[sms](*\#*) systemctl restart ntpd
\end{lstlisting}
% end_ohpc_run


%\begin{center}
\begin{tcolorbox}[]
\small Many server BIOS configurations have PXE network booting configured
as the primary option in the boot order by default. If your compute nodes have
a different device as the first in the sequence, the \texttt{ipmitool} utility
can be used to enable PXE.
\begin{lstlisting}[language=bash]
[sms](*\#*) ipmitool -E -I lanplus -H ${bmc_ipaddr} -U root chassis bootdev pxe options=persistent
\end{lstlisting}
\end{tcolorbox}
\end{center}



\subsection{Add resource management services on {\em master} node} \label{sec:add_rm}
\OHPC{} provides multiple options for distributed resource management. 
The following command adds the \SLURM{} workload manager server components to the
chosen {\em master} host. Note that client-side components will be added to
the corresponding compute image in a subsequent step.

% begin_ohpc_run
% ohpc_comment_header Add resource management services on master node \ref{sec:add_rm}
\begin{lstlisting}[language=bash,keywords={}]
# Install slurm server meta-package
[sms](*\#*) (*\install*) ohpc-slurm-server

# Identify resource manager hostname on master host
[sms](*\#*) perl -pi -e "s/ControlMachine=\S+/ControlMachine=${sms_name}/" /etc/slurm/slurm.conf
\end{lstlisting}
% end_ohpc_run

\begin{center}
\begin{tcolorbox}[]
  \small SLURM requires enumeration of the physical hardware characteristics
  for compute nodes under its control. In particular, three configuration
  parameters combine to define consumable compute resources: {\em Sockets},
  {\em CoresPerSocket}, and {\em ThreadsPerCore}. The default configuration
  file provided via \OHPC{} assumes dual-socket, 8 cores per socket, and two
  threads per core for this 4-node example. If this does not reflect your local
  hardware, please update the configuration file at
  \path{/etc/slurm/slurm.conf} accordingly to match your particular hardware.
  Note that the SLURM project provides an easy-to-use online configuration tool that
  can be accessed
 \href{https://slurm.schedmd.com/configurator.html}{\color{blue} here}. 
\end{tcolorbox}
\end{center}

% begin_ohpc_run
% ohpc_comment_header Update node configuration for slurm.conf
% ohpc_command if [[ ${update_slurm_nodeconfig} -eq 1 ]];then
% ohpc_indent 5
% ohpc_command perl -pi -e "s/^NodeName=.+$/#/" /etc/slurm/slurm.conf
% ohpc_command perl -pi -e "s/ Nodes=c\S+ / Nodes=c[1-$num_computes] /" /etc/slurm/slurm.conf
% ohpc_command echo -e ${slurm_node_config} >> /etc/slurm/slurm.conf
% ohpc_indent 0
% ohpc_command fi
% end_ohpc_run

Other versions of this guide are available that describe installation of alternate
resource management systems, and they can be found in the \texttt{docs-ohpc}
package.



\subsection{Optionally add \InfiniBand{} support services on {\em master} node} \label{sec:add_ofed}
The following command adds OFED and PSM support using base distro-provided drivers
to the chosen {\em master} host.

% begin_ohpc_run
% ohpc_comment_header Optionally add InfiniBand support services on master node \ref{sec:add_ofed}
% ohpc_command if [[ ${enable_ib} -eq 1 ]];then
% ohpc_indent 5
\begin{lstlisting}[language=bash,keywords={}]
[sms](*\#*) (*\groupinstall*) "InfiniBand Support"
\end{lstlisting}
% ohpc_indent 0
% ohpc_command fi
% end_ohpc_run

\begin{center}
  \begin{tcolorbox}[]
InfiniBand networks require a subnet management service that can typically be
run on either an administrative node, or on the switch itself. The optimal placement and
configuration of the
subnet manager is beyond the scope of this document, but \baseOS{} provides the
\texttt{opensm} package should you choose to run it on the {\em master} node.
\end{tcolorbox}
\end{center}

% begin_ohpc_run
% ohpc_validation_newline
% ohpc_validation_comment Optionally enable opensm subnet manager
% ohpc_command if [[ ${enable_opensm} -eq 1 ]];then
% ohpc_indent 5
% ohpc_command (*\install*) opensm
% ohpc_command systemctl enable opensm
% ohpc_command systemctl start opensm
% ohpc_indent 0
% ohpc_command fi


With the \InfiniBand{} drivers included, you can also enable (optional) IPoIB functionality
which provides a mechanism to send IP packets over the IB network. If you plan
to mount a \Lustre{} file system over \InfiniBand{} (see \S\ref{sec:lustre_client}
for additional details), then having IPoIB enabled is a requirement for the
\Lustre{} client. \OHPC{} provides a template configuration file to aid in setting up
an {\em ib0} interface on the {\em master} host. To use, copy the template
provided and update the \texttt{\$\{sms\_ipoib\}} and
\texttt{\$\{ipoib\_netmask\}} entries to match local desired settings (alter ib0
naming as appropriate if system contains dual-ported or multiple HCAs).

% begin_ohpc_run
% ohpc_validation_newline
% ohpc_validation_comment Optionally enable IPoIB interface on SMS
% ohpc_command if [[ ${enable_ipoib} -eq 1 ]];then
% ohpc_indent 5
% ohpc_validation_comment Enable ib0
\begin{lstlisting}[language=bash,literate={-}{-}1,keywords={},upquote=true]
[sms](*\#*) cp /opt/ohpc/pub/examples/network/centos/ifcfg-ib0 /etc/sysconfig/network-scripts

# Define local IPoIB address and netmask
[sms](*\#*) perl -pi -e "s/master_ipoib/${sms_ipoib}/" /etc/sysconfig/network-scripts/ifcfg-ib0
[sms](*\#*) perl -pi -e "s/ipoib_netmask/${ipoib_netmask}/" /etc/sysconfig/network-scripts/ifcfg-ib0

# configure NetworkManager to *not* override local /etc/resolv.conf
[sms](*\#*) echo "[main]"   >  /etc/NetworkManager/conf.d/90-dns-none.conf
[sms](*\#*) echo "dns=none" >> /etc/NetworkManager/conf.d/90-dns-none.conf
# Start up NetworkManager to initiate ib0
[sms](*\#*) systemctl start NetworkManager

\end{lstlisting}
% ohpc_indent 0
% ohpc_command fi
% end_ohpc_run



\subsection{Optionally add \OmniPath{} support services on {\em master} node} \label{sec:add_opa}
The following command adds Omni-Path support using base distro-provided drivers
to the chosen {\em master} host.

% begin_ohpc_run
% ohpc_comment_header Optionally add Omni-Path support services on master node \ref{sec:add_opa}
% ohpc_command if [[ ${enable_opa} -eq 1 ]];then
% ohpc_indent 5
\begin{lstlisting}[language=bash,keywords={}]
[sms](*\#*) (*\install*) opa-basic-tools

# Load RDMA services
[sms](*\#*) systemctl start rdma
\end{lstlisting}
% ohpc_indent 0
% ohpc_command fi
% end_ohpc_run

\begin{center}
  \begin{tcolorbox}[]
\OmniPath{} networks require a subnet management service that can typically be
run on either an administrative node, or on the switch itself. The optimal
placement and configuration of the subnet manager is beyond the scope of this
document, but \baseOS{} provides the \texttt{opa-fm} package should you choose
to run it on the {\em master} node.
\end{tcolorbox}
\end{center}

% begin_ohpc_run
% ohpc_validation_newline
% ohpc_validation_comment Optionally enable OPA fabric manager
% ohpc_command if [[ ${enable_opafm} -eq 1 ]];then
% ohpc_indent 5
% ohpc_command (*\install*) opa-fm
% ohpc_command systemctl enable opafm
% ohpc_command systemctl start opafm
% ohpc_indent 0
% ohpc_command fi




\vspace*{0.2cm}
\subsubsection{Add \OHPC{} components} \label{sec:add_components}
% -*- mode: latex; fill-column: 120; -*-

The next step is adding \OHPC{} components to the {\em compute} nodes that at this
point are running basic OSes.  This process will leverage two \Confluent{}-provided
commands: \texttt{nodeshell} to run \texttt{\pkgmgr{}} installer on all the
nodes in parallel  and \texttt{nodersync} to distribute configuration files from the
SMS to the {\em compute} nodes.

\noindent To do this, repositories on the {\em compute} nodes need to be configured
properly.

\Confluent{} has automatically setup an  OS repository on the SMS and configured the
nodes to use it, but it has  also enabled online OS repositories.


\noindent Next, we alse add the OHPC repo to the compute nodes \S\ref{sec:enable_repo}

% begin_ohpc_run
% ohpc_comment_header Setup nodes repositories and Install OHPC components \ref{sec:add_components}
\begin{lstlisting}[language=bash,literate={-}{-}1,keywords={},upquote=true]
# Add OpenHPC repo 
[sms](*\#*) (*\chrootinstall*) http://repos.openhpc.community/OpenHPC/3/EL_9/x86_64/ohpc-release-3-1.el9.x86_64.rpm
\end{lstlisting}
% end_ohpc_run

The {\em compute} nodes also need access to the EPEL repository, a required
dependency for \OHPC{} packages. 

% begin_ohpc_run
% ohpc_comment_header Configure access to EPEL repo
\begin{lstlisting}[language=bash,literate={-}{-}1,keywords={},upquote=true]
# Add epel repo
[sms](*\#*) (*\chrootinstall*) epel-release

\end{lstlisting}
% end_ohpc_run


\noindent Additionally, a workaround is needed for \OHPC{} documentation files,
which are installed into a read-only NFS share /opt/ohpc/pub. Any package
attempting to write to that directory will fail to install. The following
prevents that by directing \texttt{rpm} not to install documentation files on
the {\em compute} nodes:

% begin_ohpc_run
\begin{lstlisting}[language=bash,literate={-}{-}1,keywords={},upquote=true]
[sms](*\#*) nodeshell compute echo -e %_excludedocs 1 \>\> ~/.rpmmacros
\end{lstlisting}
% end_ohpc_run

\noindent Now \OHPC{} and other cluster-related software components can be
installed on the nodes. The first step is to install a base compute package:
% begin_ohpc_run
% ohpc_comment_header Add OpenHPC base components to compute image
\begin{lstlisting}[language=bash,literate={-}{-}1,keywords={},upquote=true]
# Install compute node base meta-package
[sms](*\#*) (*\chrootinstall*) ohpc-base-compute
\end{lstlisting}
% end_ohpc_run

\noindent Next, we can include additional components:


%\newpage
% begin_ohpc_run
% ohpc_validation_comment Add OpenHPC components to compute instance
\begin{lstlisting}[language=bash,literate={-}{-}1,keywords={},upquote=true]
# Add Slurm client support meta-package
[sms](*\#*) (*\chrootinstall*) ohpc-slurm-client

# Add Network Time Protocol (NTP) support
[sms](*\#*) (*\chrootinstall*) ntp

# Add kernel drivers
[sms](*\#*) (*\chrootinstall*) kernel

# Enable crb 
[sms](*\#*) nodeshell compute /usr/bin/crb enable

# Include nfs-utils  
[sms](*\#*) (*\chrootinstall*) nfs-utils

# Include modules user environment
[sms](*\#*) (*\chrootinstall*) lmod-ohpc
\end{lstlisting}


% end_ohpc_run

% ohpc_comment_header Optionally add InfiniBand support services in compute node image \ref{sec:add_components}
% ohpc_command if [[ ${enable_ib} -eq 1 ]];then
% ohpc_indent 5
\begin{lstlisting}[language=bash,literate={-}{-}1,keywords={},upquote=true]
# Optionally add IB support and enable
[sms](*\#*) (*\groupchrootinstall*) "InfiniBand Support"
\end{lstlisting}
% ohpc_indent 0
% ohpc_command fi
% end_ohpc_run

\vspace*{-0.25cm}
\subsubsection{Customize system configuration} \label{sec:master_customization}
Here we set up \NFS{}  mounting of a
\$HOME file system and the public \OHPC{} install path (\texttt{/opt/ohpc/pub})
that will be hosted by the {\em master} host in this  example configuration.

\vspace*{0.15cm}
% begin_ohpc_run
% ohpc_comment_header Customize system configuration \ref{sec:master_customization}
\begin{lstlisting}[language=bash,literate={-}{-}1,keywords={},upquote=true]
# Disable /tftpboot and /install export entries
[sms](*\#*) perl -pi -e "s|/tftpboot|#/tftpboot|" /etc/exports
[sms](*\#*) perl -pi -e "s|/install|#/install|" /etc/exports

# Export /home and OpenHPC public packages from master server
[sms](*\#*) echo "/home *(rw,no_subtree_check,fsid=10,no_root_squash)" >> /etc/exports
[sms](*\#*) echo "/opt/ohpc/pub *(ro,no_subtree_check,fsid=11)" >> /etc/exports
[sms](*\#*) exportfs -a
[sms](*\#*) systemctl restart nfs-server
[sms](*\#*) systemctl enable nfs-server


# Create NFS client mounts of /home and /opt/ohpc/pub on compute hosts
[sms](*\#*) nodeshell compute echo \
        "\""${sms_ip}:/home /home nfs nfsvers=3,nodev,nosuid 0 0"\"" \>\> /etc/fstab
[sms](*\#*) nodeshell compute echo \
        "\""${sms_ip}:/opt/ohpc/pub /opt/ohpc/pub nfs nfsvers=3,nodev 0 0"\"" \>\> /etc/fstab
[sms](*\#*) nodeshell compute systemctl restart nfs

# Mount NFS shares
[sms](*\#*) nodeshell compute mount /home
[sms](*\#*) nodeshell compute mkdir -p /opt/ohpc/pub
[sms](*\#*) nodeshell compute mount /opt/ohpc/pub

\end{lstlisting}
% end_ohpc_run



% Additional commands when additional computes are requested

% begin_ohpc_run
% ohpc_validation_newline
% ohpc_validation_comment Update basic slurm configuration if additional computes defined
% ohpc_validation_comment This is performed on the SMS, nodes will pick it up config file is copied there later
% ohpc_command if [ ${num_computes} -gt 4 ];then
% ohpc_command    perl -pi -e "s/^NodeName=(\S+)/NodeName=${compute_prefix}[1-${num_computes}]/" /etc/slurm/slurm.conf
% ohpc_command    perl -pi -e "s/^PartitionName=normal Nodes=(\S+)/PartitionName=normal Nodes=${compute_prefix}[1-${num_computes}]/" /etc/slurm/slurm.conf
% ohpc_command fi
% end_ohpc_run

%\clearpage
\subsubsection{Additional Customization ({\em optional})} \label{sec:addl_customizations}
This section highlights common additional customizations that can {\em
optionally} be applied to the local cluster environment. These customizations
include:

\begin{multicols}{2}
\begin{itemize*}
\item Increase memlock limits

\nottoggle{ispbs}{\item Restrict ssh access to compute resources}

\item Add \beegfs{} client
\item Add \Lustre{} client

\iftoggle{isWarewulf}{\item Enable syslog forwarding}

\item Add \Nagios{} Core monitoring
\item Add \Ganglia{} monitoring
\item Add \clustershell{}
\item Add \mrsh{}
\item Add \genders{}
%%\item Add \powerman{}
\item Add \conman{}  
\end{itemize*}
\end{multicols}

\noindent Details on the steps required for each of these customizations are
discussed further in the following sections.


\paragraph{Increase locked memory limits}
In order to utilize \InfiniBand{} or Omni-Path as the underlying high speed interconnect, it is
generally necessary to increase the locked memory settings for system
users. This can be accomplished by updating
the \texttt{/etc/security/limits.conf} file and this should be performed
on all job submission hosts. In this recipe, jobs
are submitted from the {\em master} host, and the following commands can be
used to update the maximum locked memory settings on both the master host and
compute nodes:

% begin_ohpc_run
% ohpc_validation_newline
% ohpc_validation_comment Update memlock settings
\begin{lstlisting}[language=bash,keywords={},upquote=true]
# Update memlock settings on master
[sms](*\#*) perl -pi -e 's/# End of file/\* soft memlock unlimited\n$&/s' /etc/security/limits.conf
[sms](*\#*) perl -pi -e 's/# End of file/\* hard memlock unlimited\n$&/s' /etc/security/limits.conf
# Update memlock settings on compute
[sms](*\#*) nodeshell compute perl -pi -e "'s/# End of file/\* soft memlock unlimited\n$&/s' \
            /etc/security/limits.conf"
[sms](*\#*) nodeshell compute perl -pi -e "'s/# End of file/\* hard memlock unlimited\n$&/s' \
            /etc/security/limits.conf"
\end{lstlisting}
% end_ohpc_run

\paragraph{Enable ssh control via resource manager}
An additional optional customization that is recommended is to
restrict \texttt{ssh} access on compute nodes to only allow access by users who
have an active job associated with the node. This can be enabled via the use of
a pluggable authentication module (PAM) provided as part of the \SLURM{} package
installs. To enable this feature on {\em compute} nodes, issue the
following:

% begin_ohpc_run
% ohpc_validation_newline
% ohpc_validation_comment Enable slurm pam module
\begin{lstlisting}[language=bash,keywords={},upquote=true]
[sms](*\#*) nodeshell compute echo "\""account required pam_slurm.so"\"" \>\> /etc/pam.d/sshd
\end{lstlisting}
% end_ohpc_run

\paragraph{Add \Lustre{} client} \label{sec:lustre_client}
To add \Lustre{} client support on the cluster, it necessary to install the client
and associated modules on each host needing to access a \Lustre{} file system. In
this recipe, it is assumed that the \Lustre{} file system is hosted by servers
that are pre-existing and are not part of the install process. Outlining the
variety of \Lustre{} client mounting options is beyond the scope of this document,
%(please consult \Lustre{} documentation for more details on failover configuration
%support and networking options), 
but the general requirement is to add a mount entry for the desired file system
that defines the management server (MGS) and underlying network transport
protocol. To add client mounts on both the {\em master} server and {\em
compute} image, the following commands can be used. Note that the \Lustre{} file
system to be mounted is identified by the \texttt{\$\{mgs\_fs\_name\}} variable. 
In this example, the file system is configured to be mounted locally
as \path{/mnt/lustre}.

% begin_ohpc_run
% ohpc_validation_newline
% ohpc_validation_comment Enable Optional packages
% ohpc_validation_newline
% ohpc_command if [[ ${enable_lustre_client} -eq 1 ]];then
% ohpc_indent 5

% ohpc_validation_comment Install Lustre client on master
\begin{lstlisting}[language=bash,keywords={},upquote=true]
    # Add Lustre client software to master host
    [sms](*\#*) (*\install*) lustre-client-ohpc
    \end{lstlisting}
    % end_ohpc_run
    
    % begin_ohpc_run
    % ohpc_validation_comment Enable lustre on compute nodes
    \begin{lstlisting}[language=bash,keywords={},upquote=true]
    # Enable lustre on compute nodes
    [sms](*\#*) (*\chrootinstall*) lustre-client-ohpc
    
    # Create mount point and file system mount
    [sms](*\#*) nodeshell compute mkdir /mnt/lustre
    [sms](*\#*) nodeshell compute echo  \
            "\""${mgs_fs_name} /mnt/lustre lustre defaults,_netdev,localflock,retry=2 0 0"\"" \>\> /etc/fstab
    \end{lstlisting}
    % end_ohpc_run
    
\begin{center}
\begin{tcolorbox}[]
\small

The suggested mount options shown for Lustre leverage the
\texttt{localflock} option. This is a
\href{http://wiki.lustre.org/Mounting_a_Lustre_File_System_on_Client_Nodes}{\color{blue}{Lustre-specific}}
setting that enables client-local flock support. It is much faster
than cluster-wide flock, but if you have an application requiring
cluster-wide, coherent file locks, use the standard \texttt{flock}
attribute instead.

\end{tcolorbox}
\end{center}


The default underlying network type used by \Lustre{} is {\em tcp}. If your
external \Lustre{} file system is to be mounted using a network type other than
{\em tcp}, additional configuration files are necessary to identify the desired
network type. The example below illustrates creation of modprobe configuration files
instructing \Lustre{} to use an \InfiniBand{} network with the \textbf{o2ib} LNET driver
attached to \texttt{ib0}. Note that these modifications are made to both the
{\em master} host and {\em compute} nodes.

%x\clearpage
% begin_ohpc_run
% ohpc_validation_comment Enable o2ib for Lustre
\begin{lstlisting}[language=bash,keywords={},upquote=true]
[sms](*\#*) echo "options lnet networks=o2ib(ib0)" >> /etc/modprobe.d/lustre.conf
[sms](*\#*) nodeshell compute echo "\""options lnet networks=o2ib\(ib0\)"\"" \>\> /etc/modprobe.d/lustre.conf
\end{lstlisting}
% end_ohpc_run

With the \Lustre{} configuration complete, the client can be mounted on the {\em master}
and {\em compute} hosts as follows:
% begin_ohpc_run
% ohpc_validation_comment mount Lustre client on master and compute
\begin{lstlisting}[language=bash,keywords={},upquote=true]
[sms](*\#*) mkdir /mnt/lustre
[sms](*\#*) mount -t lustre -o localflock ${mgs_fs_name} /mnt/lustre

# Mount on compute nodes
[sms](*\#*) nodeshell compute mount /mnt/lustre
\end{lstlisting}
% ohpc_indent 0
% ohpc_command fi
% ohpc_validation_newline
% end_ohpc_run

\vspace*{0.4cm}

\paragraph{Add \clustershell{}}
\clustershell{} is an event-based Python library to execute commands in parallel
across cluster nodes. 

% begin_ohpc_run
% ohpc_validation_newline
% ohpc_indent 5
% ohpc_validation_comment Install clustershell
\begin{lstlisting}[language=bash,keywords={},upquote=true]
# Install ClusterShell
[sms](*\#*) (*\install*) clustershell-ohpc

# Setup node definitions
[sms](*\#*) cd /etc/clustershell/groups.d
[sms](*\#*) mv local.cfg local.cfg.orig
[sms](*\#*) if true ; then
    cat << EOF > local.cfg
    adm: ${sms_name}
    compute: c[1-${num_computes}]
    all: @adm,@compute
    EOF
fi
\end{lstlisting}
% ohpc_indent 0
% end_ohpc_run



\paragraph{Add \genders{}}
\genders{} is a static cluster configuration database or node typing database
used for cluster configuration management. Other tools and users can access the
\genders{} database in order to make decisions about where an action, or even
what action, is appropriate based on associated types or "\genders{}".

Values may also be assigned to and retrieved from a {\em gender} to provide
further granularity.

% begin_ohpc_run
% ohpc_validation_newline
% ohpc_command if [[ ${enable_genders} -eq 1 ]];then
% ohpc_indent 5
% ohpc_validation_comment Install genders
\begin{lstlisting}[language=bash,keywords={},upquote=true]
# Install genders
[sms](*\#*) (*\install*) genders-ohpc

# Generate a sample genders file
[sms](*\#*) echo -e "${sms_name}\tsms" > /etc/genders
[sms](*\#*) for ((i=0; i<$num_computes; i++)) ; do
              echo -e "${c_name[$i]}\tcompute,bmc=${c_bmc[$i]}"
           done >> /etc/genders
\end{lstlisting}
% ohpc_indent 0
% ohpc_command fi
% end_ohpc_run



\paragraph{Add Magpie}
Magpie contains a number of scripts to aid in running a variety of big data software
frameworks within HPC queuing environments. Examples include Hadoop, Spark, Hbase, Storm, Pig,
Mahout, Phoenix, Kafka, Zeppelin, and Zookeeper.  Consult the online
\href{https://github.com/LLNL/magpie}{\color{blue}repository} for
more information on using these scripts; basic installation is outlined as follows:

% begin_ohpc_run
% ohpc_validation_newline
% ohpc_command if [[ ${enable_magpie} -eq 1 ]];then
% ohpc_indent 5
% ohpc_validation_comment Install magpie
\begin{lstlisting}[language=bash,keywords={},upquote=true]
# Install magpie
[sms](*\#*) (*\install*) magpie-ohpc
\end{lstlisting}
% ohpc_indent 0
% ohpc_command fi
% end_ohpc_run



\paragraph{Add \conman{}} \label{sec:add_conman}
\conman{} is a serial console management program designed to support a large
number of console devices and simultaneous users. It supports logging console
device output and connecting to compute node consoles via IPMI
serial-over-lan. Installation and example configuration is outlined below.

% begin_ohpc_run
% ohpc_validation_newline
% ohpc_validation_comment Optionally, enable conman and configure
% ohpc_command if [[ ${enable_ipmisol} -eq 1 ]];then
% ohpc_indent 5
\begin{lstlisting}[language=bash,keywords={},upquote=true]
# Install conman to provide a front-end to compute consoles and log output
[sms](*\#*) (*\install*) conman-ohpc

# Configure conman for computes (note your IPMI password is required for console access)
[sms](*\#*) for ((i=0; i<$num_computes; i++)) ; do
              echo -n 'CONSOLE name="'${c_name[$i]}'" dev="ipmi:'${c_bmc[$i]}'" '
              echo 'ipmiopts="'U:${bmc_username},P:${IPMI_PASSWORD:-undefined},W:solpayloadsize'"'
        done >> /etc/conman.conf

# Enable and start conman
[sms](*\#*) systemctl enable conman
[sms](*\#*) systemctl start conman
\end{lstlisting}
% ohpc_indent 0
% ohpc_command fi
% end_ohpc_run

\noindent Note that an additional kernel boot option is typically necessary to
enable serial console output. This option is highlighted in \S\ref{sec:optional_kargs} after
compute nodes have been registered with the provisioning system.

% # Define node kernel arguments to support SOL console
% [sms](*\#*) wwsh -y provision set "${compute_regex}" --kargs "${kargs} console=ttyS1,115200"


\paragraph{Add \nhc{}} \label{sec:add_nhc}
Resource managers often provide for a periodic "node health check" to be
performed on each compute node to verify that the node is working
properly. Nodes which are determined to be "unhealthy" can be marked as down or
offline so as to prevent jobs from being scheduled or run on them. This helps
increase the reliability and throughput of a cluster by reducing preventable
job failures due to misconfiguration, hardware failure, etc. OpenHPC
distributes \nhc{} to fulfill this requirement.

In a typical scenario, the \nhc{} driver script is run periodically on each compute
node by the resource manager client daemon. It loads its
configuration file to determine which checks are to be run on the current node
(based on its hostname). Each matching check is run, and if a failure is
encountered, \nhc{} will exit with an error message describing the problem. It can
also be configured to mark nodes offline so that the scheduler will not assign
jobs to bad nodes, reducing the risk of system-induced job failures. 

% begin_ohpc_run
% ohpc_validation_newline
% ohpc_validation_comment Optionally, enable nhc and configure
\begin{lstlisting}[language=bash,keywords={},upquote=true]
# Install NHC on master and compute nodes
[sms](*\#*) (*\install*) nhc-ohpc
[sms](*\#*) (*\chrootinstall*) nhc-ohpc
\end{lstlisting}
% end_ohpc_run


% begin_ohpc_run
% ohpc_validation_newline
\begin{lstlisting}[language=bash,keywords={},upquote=true]
# Register as SLURM's health check program
[sms](*\#*) echo "HealthCheckProgram=/usr/sbin/nhc" >> /etc/slurm/slurm.conf
[sms](*\#*) echo "HealthCheckInterval=300" >> /etc/slurm/slurm.conf  # execute every five minutes
\end{lstlisting}
% end_ohpc_run




%\subsubsection{Identify files for synchronization} \label{sec:file_import}
%The \Confluent{} system includes functionality to synchronize files located on the
SMS server for distribution to managed hosts. This is one way to
distribute user credentials to {\em compute} nodes (alternatively, you may
prefer to use a central authentication service like LDAP). To import local file-based
credentials, issue the following to enable the {\em synclist} feature and
register user credential files:

% begin_ohpc_run
% ohpc_comment_header Import files \ref{sec:file_import}
\begin{lstlisting}[language=bash,literate={-}{-}1,keywords={},upquote=true,literate={BOSSHORT}{\baseosshort{}}1]
# Add desired credential files to synclist
[sms](*\#*) echo "/etc/passwd -> /etc/passwd" > /var/lib/confluent/public/os/rocky-9.4-x86_64-default
[sms](*\#*) echo "/etc/group -> /etc/group" >> /var/lib/confluent/public/os/rocky-9.4-x86_64-default
[sms](*\#*) echo "/etc/shadow -> /etc/shadow" >> /var/lib/confluent/public/os/rocky-9.4-x86_64-default
\end{lstlisting}
% \end_ohpc_run
%\noindent Similarly, to import the
%global Slurm configuration file and the
cryptographic
key
%and associated file permissions
that is required by the {\em munge}
authentication library to be available on every host in the resource management
pool, issue the following:

% begin_ohpc_run
\begin{lstlisting}[language=bash,literate={-}{-}1,keywords={},upquote=true]
[sms](*\#*) echo "/etc/munge/munge.key -> /etc/munge/munge.key" >> /var/lib/confluent/public/os/rocky-9.4-x86_64-default
\end{lstlisting}
% \end_ohpc_run

\begin{center}
\begin{tcolorbox}[]
\small
The ``\texttt{nodeapply compute -F}'' command can be used to distribute changes made to any
defined synchronization files on the SMS host. Users wishing to automate this process may
want to consider adding a crontab entry to perform this action at defined intervals.
\end{tcolorbox}
\end{center}

%%%\subsubsection{Optional kernel arguments} \label{sec:optional_kargs}
%%%If you chose to enable \conman{} in \S\ref{sec:add_conman}, additional
boot-time kernel arguments are needed to enable serial console
redirection. An example provisioning setting which adds to any other kernel arguments defined
in \texttt{\$\{kargs\}} is as follows:

% begin_ohpc_run
% ohpc_validation_newline
% ohpc_validation_comment Optionally, enable console redirection 
% ohpc_command if [[ ${enable_ipmisol} -eq 1 ]];then
% ohpc_indent 5
\begin{lstlisting}[language=bash,keywords={},upquote=true]
# Define node kernel arguments to support SOL console
[sms](*\#*) wwsh -y provision set "${compute_regex}" --kargs "${kargs} console=ttyS1,115200"
\end{lstlisting}
% ohpc_indent 0
% ohpc_command fi
% end_ohpc_run


\section{Install \OHPC{} Development Components}
The install procedure outlined in \S\ref{sec:basic_install} highlighted the
steps necessary to install a {\em master} host, assemble and customize a {\em
  compute} image, and provision several compute hosts from bare-metal. With
these steps completed, additional \OHPC{}-provided packages can now be added to
support a flexible HPC development environment including development tools,
C/C++/Fortran compilers, MPI stacks, and a variety of 3rd party libraries. The
following subsections highlight the additional software installation
procedures, including the addition of Intel licensed software (e.g. Composer
compiler suite, \Intel{} MPI). It is assumed that the end-site administrator
will procure and install the necessary licenses in order to use the Intel
proprietary software.


%\vspace*{-0.15cm}
%\clearpage
\subsection{Development Tools} \label{sec:install_dev_tools}
To aid in general development efforts, \OHPC{} provides recent versions of the \GNU{}
autotools collection and the Valgrind memory debugger. These can be installed as follows:

% begin_ohpc_run
% ohpc_comment_header Install Development Tools \ref{sec:install_dev_tools}
\begin{lstlisting}[language=bash,keywords={},literate={-}{-}1]
[master]$ (*\groupinstall*) ohpc-autotools
[master]$ (*\install*) valgrind-ohpc
\end{lstlisting}
% end_ohpc_run


\vspace*{-0.15cm}
\subsection{Compilers} \label{sec:install_compilers}
\FSP{} presently packages two compiler families ({\GNU{}} and {\Intel{}
  Parallel Studio}) that are integrated within the underlying
modules-environment system in a hierarchical fashion. End users of a \FSP{}
system can choose to access one compiler at a time and will be presented with
additional compiler-dependent software as a function of which compiler
toolchain is currently loaded. Each compiler toolchain can be installed
separately and the following commands illustrate the installation of both along
with any necessary dependencies:

% begin_fsp_run
% fsp_comment_header Install Compilers \ref{sec:install_compilers}
\begin{lstlisting}[language=bash]
[master]$ (*\install*) gnu-compilers-fsp intel-compilers-devel-fsp
\end{lstlisting}
% end_fsp_run


%\clearpage
\subsection{MPI Stacks} \label{sec:mpi}
\input{common/mpi_slurm}

\subsection{Performance Tools} \label{sec:install_perf_tools}
To aid in application performance analysis, \OHPC{} provides a variety of
open-source and Intel licensed software (see \S\ref{sec:byol} for more details
regarding Intel software). These can be installed as follows:

% begin_ohpc_run
% ohpc_comment_header Install Performance Tools \ref{sec:install_perf_tools}
\begin{lstlisting}[language=bash,keywords={},literate={-}{-}1]
[sms](*\#*) (*\install*) papi-ohpc
[sms](*\#*) (*\install*) intel-itac-ohpc
[sms](*\#*) (*\install*) intel-vtune-ohpc
[sms](*\#*) (*\install*) intel-advisor-ohpc
[sms](*\#*) (*\install*) intel-inspector-ohpc
[sms](*\#*) (*\groupinstall*) ohpc-mpiP
[sms](*\#*) (*\groupinstall*) ohpc-tau
\end{lstlisting}
% end_ohpc_run




\subsection{Setup default development environment}
System users often find it convenient to have a default development environment
in place so that compilation can be performed directly for parallel programs
requiring MPI. This setup can be conveniently enabled via modules and the \OHPC{}
modules environment is pre-configured to load an \texttt{ohpc} module on login
(if present). The following package install provides a default
environment that enables autotools, the gnu compiler toolchain, and the
MVAPICH2 MPI stack.

% begin_ohpc_run
\begin{lstlisting}[language=bash]
[sms](*\#*) (*\install*) lmod-defaults-gnu-mvapich2-ohpc
\end{lstlisting}
% end_ohpc_run

\begin{center}
\begin{tcolorbox}[]
\small
If you want to change the default environment from the suggestion above, \OHPC{}
provides additional choices for each of the available compiler/MPI
combinations. In particular, the following packages can be substituted instead:
\begin{multicols}{2}
\begin{itemize*}
\item lmod-defaults-gnu-openmpi-ohpc
\item lmod-defaults-gnu-impi-ohpc
\item lmod-defaults-intel-mvapich2-ohpc
\item lmod-defaults-intel-openmpi-ohpc
\item lmod-defaults-intel-impi-ohpc
\end{itemize*}
\end{multicols}
\end{tcolorbox}
\end{center}


%\vspace*{0.2cm}
\subsection{3rd Party Libraries and Tools} \label{sec:3rdparty}
\OHPC{} provides pre-packaged builds for a number of popular open-source
tools and libraries used by HPC applications and developers. For
example, \OHPC{} provides builds for \FFTW{} and \hdffive{} (including serial and parallel
I/O support), and the \GNU{} Scientific Library (GSL). Again, multiple builds of
each package are available in the \OHPC{} repository to support multiple compiler
and MPI family combinations where appropriate. Note, however, that not all
combinatorial permutations may be available for components where there are known
license incompatibilities. The general naming convention
for builds provided by \OHPC{} is to append the compiler and MPI family name that
the library was built against directly into the package name. For example,
libraries that do not require MPI as part of the build process adopt the
following RPM name: \\

\noindent
\texttt{package-<compiler\_family>-ohpc-<package\_version>-<release>.rpm} \\

\noindent Packages that do require MPI as part of the build expand upon this convention to
additionally include the MPI family name as follows: \\

\noindent
\texttt{package-<compiler\_family>-<mpi\_family>-ohpc-<package\_version>-<release>.rpm} \\

To illustrate this further, the command below queries the locally configured
repositories to identify all of the available PETSc packages that were built
with the \GNU{} toolchain. The resulting output that is included shows that
pre-built versions are available for each of the supported MPI families
presented in \S\ref{sec:mpi}.

\iftoggleverb{isCentOS}
\begin{lstlisting}[language=bash,keywords={}]
[sms](*\#*) yum search petsc-gnu ohpc
Loaded plugins: fastestmirror
Loading mirror speeds from cached hostfile
=========================== N/S matched: petsc-gnu, ohpc ===========================
petsc-gnu-impi-ohpc.x86_64 : Portable Extensible Toolkit for Scientific Computation
petsc-gnu-mpich-ohpc.x86_64 : Portable Extensible Toolkit for Scientific Computation
petsc-gnu-mvapich2-ohpc.x86_64 : Portable Extensible Toolkit for Scientific Computation
petsc-gnu-openmpi-ohpc.x86_64 : Portable Extensible Toolkit for Scientific Computation
\end{lstlisting}
\else
\begin{lstlisting}[language=bash,keepspaces=true,keywords={}]
[sms](*\#*) zypper search -t package petsc-gnu-*-ohpc
Loading repository data...
Reading installed packages...

S | Name                    | Summary
--+-------------------------+--------------------------------------------------------+--------
i | petsc-gnu-mvapich2-ohpc | Portable Extensible Toolkit for Scientific Computation | package
i | petsc-gnu-openmpi-ohpc  | Portable Extensible Toolkit for Scientific Computation | package
\end{lstlisting}
\fi


% begin_ohpc_run
% ohpc_comment_header Install 3rd Party Libraries and Tools \ref{sec:3rdparty}
\begin{lstlisting}[language=bash,keywords={},upquote=true,keepspaces]
[sms](*\#*) (*\groupinstall*) ohpc-adios        # adds available Adios packages
[sms](*\#*) (*\groupinstall*) ohpc-boost        # adds available Boost packages
[sms](*\#*) (*\groupinstall*) ohpc-fftw         # adds available FFTW packages
[sms](*\#*) (*\groupinstall*) ohpc-gsl          # adds available GSL packages
[sms](*\#*) (*\groupinstall*) ohpc-hdf5         # adds available HDF5 packages
[sms](*\#*) (*\groupinstall*) ohpc-hypre        # adds available Hypre packages
[sms](*\#*) (*\groupinstall*) ohpc-metis        # adds available METIS packages
[sms](*\#*) (*\groupinstall*) ohpc-mpiP         # adds available mpiP packages
[sms](*\#*) (*\groupinstall*) ohpc-mumps        # adds available MUMPS packages
[sms](*\#*) (*\groupinstall*) ohpc-netcdf       # adds available NetCDF packages
[sms](*\#*) (*\groupinstall*) ohpc-numpy        # adds available numerical Python packages
[sms](*\#*) (*\groupinstall*) ohpc-openblas     # adds available OpenBLAS packages
[sms](*\#*) (*\groupinstall*) ohpc-petsc        # adds available PETSC packages
[sms](*\#*) (*\groupinstall*) ohpc-phdf5        # adds available (parallel) HDF5 packages
[sms](*\#*) (*\groupinstall*) ohpc-scalapack    # adds available ScaLAPACK packages
[sms](*\#*) (*\groupinstall*) ohpc-scipy        # adds available scientific Python packages
[sms](*\#*) (*\groupinstall*) ohpc-trilinos     # adds available Trilinos packages
[sms](*\#*) (*\install*) EasyBuild-ohpc         # adds EasyBuild
[sms](*\#*) (*\install*) R_base-ohpc            # adds R
\end{lstlisting}
% end_ohpc_run


% begin_ohpc_run
% ohpc_command if [[ ${enable_mpi_defaults} -eq 1 ]];then
% ohpc_indent 5
\begin{lstlisting}[language=bash,keywords={},upquote=true,keepspaces]
# Install parallel lib meta-packages for all available MPI toolchains
[sms](*\#*) (*\install*) ohpc-gnu7-mpich-parallel-libs
[sms](*\#*) (*\install*) ohpc-gnu7-mvapich2-parallel-libs
[sms](*\#*) (*\install*) ohpc-gnu7-openmpi-parallel-libs
\end{lstlisting}
% ohpc_indent 0
% ohpc_command elif [[ ${enable_mpi_opa} -eq 1 ]];then
% ohpc_indent 5
% ohpc_command (*\install*) ohpc-gnu7-mvapich2-parallel-libs
% ohpc_command (*\install*) ohpc-gnu7-openmpi-parallel-libs
% ohpc_indent 0
% ohpc_command fi
% end_ohpc_run


\vspace*{.6cm}
\subsection{Optional Development Tool Builds} \label{sec:3rdparty_intel}
\input{common/oneapi_enabled_builds_slurm}

\section{Resource Manager Startup} \label{sec:rms_startup}
In section \S\ref{sec:basic_install}, the \SLURM{} resource manager was
installed and configured on the  {\em master} host. \SLURM{} clients were also
installed, but have not been configured yet. To do so, \SLURM{} and
\texttt{munge} configuration files need to be copied to the nodes. This can be
accomplished as follows:

% begin_ohpc_run
% ohpc_comment_header Resource Manager Startup \ref{sec:rms_startup}
\begin{lstlisting}[language=bash,keywords={}]
[sms](*\#*) nodersync /etc/slurm/slurm.conf compute:/etc/slurm/slurm.conf
[sms](*\#*) nodersync /etc/munge/munge.key compute:/etc/munge/munge.key
\end{lstlisting}
% end_ohpc_run


With \SLURM{} configured, we can now startup the
resource manager services in preparation for running user jobs. Generally, this
is a two-step process that requires starting up the controller daemons on the {\em
 master} host and the client daemons on each of the {\em compute} hosts.
%Since the {\em compute} hosts were booted into an image that included the SLURM client
%components, the daemons should already be running on the {\em compute}
%hosts.
Note that \SLURM{} leverages the use of the {\em munge} library to provide
authentication services and this daemon also needs to be running on all hosts
within the resource management pool.
%The munge daemons should already
%be running on the {\em compute} subsystem at this point,
The following commands can be used to startup the necessary services to support
resource management under \SLURM{}.

%\iftoggle{isCentOS}{\clearpage}

% begin_ohpc_run
\begin{lstlisting}[language=bash,keywords={}]
# Start munge and slurm controller on master host
[sms](*\#*) systemctl enable munge
[sms](*\#*) systemctl enable slurmctld
[sms](*\#*) systemctl start munge
[sms](*\#*) systemctl start slurmctld

# Start slurm clients on compute hosts
[sms](*\#*) nodeshell compute systemctl enable munge
[sms](*\#*) nodeshell compute systemctl enable slurmd
[sms](*\#*) nodeshell compute systemctl start munge
[sms](*\#*) nodeshell compute systemctl start slurmd
\end{lstlisting}
% end_ohpc_run

%%% In the default configuration, the {\em compute} hosts will be initialized in an
%%% {\em unknown} state. To place the hosts into production such that they are
%%% eligible to schedule user jobs, issue the following:

%%% % begin_ohpc_run
%%% \begin{lstlisting}[language=bash]
%%% [sms](*\#*) scontrol update partition=normal state=idle
%%% \end{lstlisting}
%%% % end_ohpc_run

After this, check status of the nodes within \SLURM{} by using the
\texttt{sinfo} command. All compute nodes should be in an {\em idle} state
(without asterisk). If the state is reported as {\em unknown}, the following
might help:

%%% % begin_ohpc_run
\begin{lstlisting}[language=bash]
[sms](*\#*) scontrol update partition=normal state=idle
\end{lstlisting}
%%% % end_ohpc_run

In case of additional \SLURM{} issues, ensure that the configuration file fits
your hardware and that it is identical across the nodes. Also, verify
that \SLURM{} user id is the same on the SMS and {\em compute} nodes. You may
also consult
\href{https://slurm.schedmd.com/troubleshoot.html}{\color{blue}\SLURM{}
Troubleshooting Guide}.


\section{Run a Test Job} \label{sec:test_job}
With the resource manager enabled for production usage, users should now be
able to run jobs. To demonstrate this, we will add a `test` user on the {\em master}
host that can be used to run an example job.

% begin_ohpc_run
\begin{lstlisting}[language=bash,keywords={}]
[sms](*\#*) useradd -m test
\end{lstlisting}
% end_ohpc_run

Next, the user's credentials need to be distributed across the cluster.
\Confluent{}'s \texttt{nodeappy} has a merge functionality that adds new entries into
credential files on {\em compute} nodes:

% begin_ohpc_run
\begin{lstlisting}[language=bash,keywords={}]
# Create a sync file for pushing user credentials to the nodes
[sms](*\#*) echo "/etc/passwd -> /etc/passwd" >> /var/lib/confluent/public/os/rocky-9.4-x86_64-default/syncfiles
[sms](*\#*) echo "/etc/group -> /etc/group"   >> /var/lib/confluent/public/os/rocky-9.4-x86_64-default/syncfiles

# Use Confluent to distribute credentials to nodes
[sms](*\#*) nodeapply -F compute 
\end{lstlisting}
% end_ohpc_run


~\\
\OHPC{} includes a simple ``hello-world'' MPI application in the
\path{/opt/ohpc/pub/examples} directory that can be used for this
quick compilation and execution. \OHPC{} also provides a companion
job-launch utility named \texttt{prun} that is installed in concert
with the pre-packaged MPI toolchains. This convenience script provides
a mechanism to abstract job launch across different resource managers
and MPI stacks such that a single launch command can be
used for parallel job launch in a variety of \OHPC{} environments. It
also provides a centralizing mechanism for administrators to customize
desired environment settings for their users.



\iftoggle{isSLES_ww_slurm_x86}{\clearpage}
%\iftoggle{isxCAT}{\clearpage}

\subsection{Interactive execution}
To use the newly created ``test'' account to compile and execute the
application {\em interactively} through the resource manager, execute the
following (note the use of \texttt{prun} for parallel job launch which summarizes
the underlying native job launch mechanism being used):

\begin{lstlisting}[language=bash,keywords={}]
# Switch to "test" user
[sms](*\#*) su - test

# Compile MPI "hello world" example
[test@sms ~]$ mpicc -O3 /opt/ohpc/pub/examples/mpi/hello.c

# Submit interactive job request and use prun to launch executable
[test@sms ~]$ salloc -n 8 -N 2

[test@c1 ~]$ prun ./a.out

[prun] Master compute host = c1
[prun] Resource manager = slurm
[prun] Launch cmd = mpiexec.hydra -bootstrap slurm ./a.out

 Hello, world (8 procs total)
    --> Process #   0 of   8 is alive. -> c1
    --> Process #   4 of   8 is alive. -> c2
    --> Process #   1 of   8 is alive. -> c1
    --> Process #   5 of   8 is alive. -> c2
    --> Process #   2 of   8 is alive. -> c1
    --> Process #   6 of   8 is alive. -> c2
    --> Process #   3 of   8 is alive. -> c1
    --> Process #   7 of   8 is alive. -> c2
\end{lstlisting}

\begin{center}
\begin{tcolorbox}[]
The following table provides approximate command equivalences between SLURM and
OpenPBS:

\vspace*{0.15cm}
\input common/rms_equivalence_table
\end{tcolorbox}
\end{center}
\nottoggle{isCentOS}{\clearpage}

\iftoggle{isCentOS}{\clearpage}

\subsection{Batch execution}

For batch execution, \OHPC{} provides a simple job script for reference (also
housed in the \path{/opt/ohpc/pub/examples} directory. This example script can
be used as a starting point for submitting batch jobs to the resource manager
and the example below illustrates use of the script to submit a batch job for
execution using the same executable referenced in the previous interactive example.

\begin{lstlisting}[language=bash,keywords={}]
# Copy example job script
[test@sms ~]$ cp /opt/ohpc/pub/examples/slurm/job.mpi .

# Examine contents (and edit to set desired job sizing characteristics)
[test@sms ~]$ cat job.mpi
#!/bin/bash

#SBATCH -J test               # Job name
#SBATCH -o job.%j.out         # Name of stdout output file (%j expands to %jobId)
#SBATCH -N 2                  # Total number of nodes requested
#SBATCH -n 16                 # Total number of mpi tasks #requested
#SBATCH -t 01:30:00           # Run time (hh:mm:ss) - 1.5 hours

# Launch MPI-based executable

prun ./a.out

# Submit job for batch execution
[test@sms ~]$ sbatch job.mpi
Submitted batch job 339
\end{lstlisting}

\begin{center}
\begin{tcolorbox}[]
\small
The use of the \texttt{\%j} option in the example batch job script shown is a convenient
way to track application output on an individual job basis. The \texttt{\%j} token
is replaced with the \SLURM{} job allocation number once assigned (job~\#339 in
this example).
\end{tcolorbox}
\end{center}

\clearpage
\appendix
{\bf \LARGE \centerline{Appendices}} \vspace*{0.2cm}

\addcontentsline{toc}{section}{Appendices}
\renewcommand{\thesubsection}{\Alph{subsection}}

\subsection{Installation Template}  \label{appendix:template_script}

This appendix highlights the availability of a companion installation script
that is included with \OHPC{} documentation.  This script, when combined with
local site inputs, can be used to implement a starting recipe for
bare-metal system installation and configuration. This template script is used
during validation efforts to test cluster installations and is provided as a
convenience for administrators as a starting point for potential site
customization. 

The template script relies on the use of a simple text file to
define local site variables that were outlined in \S\ref{sec:inputs}.
% The collection of command-line instructions that are in this guide, when
% combined with local site inputs,  To aid in direct usage of the
% commands called out in this particular recipe, and to also allow for potential
% site customization, the \OHPC{} documentation package includes a template script
% summarizing the commands used herein. This script can be used in conjunction
% with a simple text file to define the local site variables defined in the
% previous section (
By default, the template install script attempts to use local variable settings
sourced from the \path{/opt/ohpc/pub/doc/recipes/vanilla/input.local} file,
however, this choice can be overridden by the use of the
\texttt{OHPC\_INPUT\_LOCAL} environment variable. The template install script is
intended for execution on the SMS {\em master} host and is installed as part of
the \texttt{docs-ohpc} package into \path{/opt/ohpc/pub/doc/recipes/vanilla/recipe.sh}.
After enabling the \OHPC{} repository and reviewing the guide for additional information on the intent of the
commands, the general starting approach for using this template is as follows:

\begin{enumerate}
\item Install the \texttt{docs-ohpc} package

\begin{lstlisting}[language=bash,keywords={}]
[master]# (*\install*) docs-ohpc
\end{lstlisting}

\item Copy the provided template input file to use as a starting point to
  define local site settings:
\begin{lstlisting}
[master]# cp  /opt/ohpc/pub/doc/recipes/vanilla/input.local input.local
\end{lstlisting}

\item Update \path{input.local} with desired settings

\item Copy the template installation script which contains command-line
  instructions culled from this guide.

\begin{lstlisting}[language=bash,keywords={}]
[master]# cp -p /opt/ohpc/pub/doc/recipes/vanilla/recipe.sh .
\end{lstlisting}

\item Review and edit \path{recipe.sh} to suite.

\item Use environment variable to define local input file and execute
  \path{recipe.sh} to perform a local installation.

\begin{lstlisting}[language=bash,keywords={}]
[master]# export OHPC_INPUT_LOCAL=./input.local
[master]# ./recipe.sh
\end{lstlisting}
\end{enumerate}




\subsection{Upgrading OpenHPC Packages}  \label{appendix:upgrade}


As newer \OHPC{} releases are made available, users are encouraged to upgrade
their locally installed packages against the latest repository versions to
obtain access to bug fixes and newer component versions. This can be
accomplished with the underlying package manager as \OHPC{} packaging maintains
versioning state across releases. Also, package builds available from the
\OHPC{} repositories have ``\texttt{-ohpc}'' appended to their names so that
wild cards can be used as a simple way to obtain updates. The following general
procedure highlights a method for upgrading existing installations.
When upgrading from a minor release older than v\OHPCVerTree{}, you will first
need to update your local \OHPC{} repository configuration to point against the
v\OHPCVerTree{} release (or update your locally hosted mirror). Refer to
\S\ref{sec:enable_repo} for more details on enabling the latest
repository. In contrast, when upgrading between micro releases on the same
branch (e.g. from v\OHPCVerTree{} to \OHPCVerTree{}.2), there is no need to
adjust local package manager configurations when using the public repository as
rolling updates are pre-configured.

The initial step, when using a local mirror as described in
\S\ref{sec:enable_repo}, is downloading a new tarball from \texttt{http://build.openhpc.community/dist}.
After this, move (or remove) the old repository directory, unpack
the new tarball and run the \texttt{make\_repo.sh} script. In the following,
substitute \texttt{x.x.x} with the desired version number.

\begin{lstlisting}[language=bash,keywords={},basicstyle=\fontencoding{T1}\fontsize{7.6}{10}\ttfamily,
	literate={VER}{\OHPCVerTree{}}1 {OSREPO}{\OSTree{}}1 {TAG}{\OSTag{}}1 {ARCH}{\arch{}}1 {-}{-}1
        {VER2}{\OHPCVersion{}}1]
[sms](*\#*) wget http://repos.openhpc.community/dist/x.x.x/OpenHPC-x.x.x.OSREPO.ARCH.tar
[sms](*\#*) mv $ohpc_repo_dir "$ohpc_repo_dir"_old
[sms](*\#*) mkdir -p $ohpc_repo_dir
[sms](*\#*) tar xvf OpenHPC-x.x.x.OSREPO.ARCH.tar -C $ohpc_repo_dir
[sms](*\#*) $ohpc_repo_dir/make_repo.sh
\end{lstlisting}

The {\em compute} hosts will pickup new packages automatically as long as the
repository location on the SMS stays the same. If you prefer to use a different
location, setup repositories on the nodes as described in
\S\ref{sec:enable_repo}.

The actual update is performed as follows:
\begin{enumerate*}
\item (Optional) Ensure repo metadata is current (on head and compute nodes.)
  Package managers will naturally do this on their own over time,
  but if you are wanting to access updates immediately after a new release,
  the following can be used to sync to the latest.

\begin{lstlisting}[language=bash,keywords={}]
[sms](*\#*)  (*\clean*)
[sms](*\#*)  nodeshell compute (*\clean*)
\end{lstlisting}

\item Upgrade master (SMS) node

\begin{lstlisting}[language=bash,keywords={}]
[sms](*\#*)  (*\upgrade*) "*-ohpc"
\end{lstlisting}

\item Upgrade packages on compute nodes

\begin{lstlisting}[language=bash,keywords={}]
[sms](*\#*) nodeshell compute  (*\upgrade*) "*-ohpc"
\end{lstlisting}

\nottoggle{isxCATstateful}{\item Rebuild image(s)

\iftoggleverb{isWarewulf}
\begin{lstlisting}[language=bash,keywords={}]
[sms](*\#*) wwvnfs --chroot $CHROOT
\end{lstlisting}
\fi

\iftoggleverb{isxCAT}
\begin{lstlisting}[language=bash,keywords={},basicstyle=\fontencoding{T1}\fontsize{8.0}{10}\ttfamily,
    literate={-}{-}1 {BOSVER}{\baseos{}}1 {ARCH}{\arch{}}1]
[sms](*\#*) packimage BOSVER-x86_64-netboot-compute
\end{lstlisting}
\fi
}

\end{enumerate*}

\noindent Note that to update running services such as a resource manager a
service restart (or a full reboot) is required.

%%% \subsubsection{New component variants}
%%%
%%% As newer variants of key compiler/MPI stacks are released, \OHPC{} will
%%% periodically add toolchains enabling the latest variant. To stay consistent
%%% throughout the build hierarchy, minimize recompilation requirements for existing
%%% binaries, and allow for multiple variants to coexist, unique delimiters are
%%% used to distinguish RPM package names and module hierarchy.
%%%
%%% In the case of a fresh install, \OHPC{} recipes default to installation of the
%%% latest toolchains available in a given release branch. However, if upgrading a
%%% previously installed system, administrators can {\em opt-in} to enable new
%%% variants as they become available. To illustrate this point, consider the
%%% previous \OHPC{} 1.3.2 release as an example which contained an {``openmpi''}
%%% MPI variant providing OpenMPI 1.10.x along with runtimes and libraries compiled
%%% with this toolchain. That release also contained the {``llvm4''} compiler
%%% family which has been updated to {``llvm5''} in \OHPC{} 1.3.3.  In the case
%%% where an admin would like to enable the newer {``openmpi3''} toolchain,
%%% installation of these additions is simplified with the use of \OHPC{}'s
%%% meta-packages (see Table~\ref{table:groups} in Appendix
%%% \ref{appendix:manifest}).  The following example illustrates adding the
%%% complete ``openmpi3'' toolchain.  Note that we leverage the convenience
%%% meta-packages containing MPI-dependent builds, and we also update the
%%% modules environment to make it the default.
%%%
%%% \begin{lstlisting}[language=bash,keywords={}]
%%% # Update default environment
%%% [sms](*\#*) (*\remove*) lmod-defaults-gnu7-openmpi-ohpc
%%% [sms](*\#*) (*\install*) lmod-defaults-gnu7-openmpi3-ohpc
%%%
%%% # Install OpenMPI 3.x-compiled meta-packages with dependencies
%%% [sms](*\#*)  (*\install*) ohpc-gnu7-perf-tools \
%%%                          ohpc-gnu7-io-libs \
%%%                          ohpc-gnu7-python-libs \
%%%                          ohpc-gnu7-runtimes \
%%%                          ohpc-gnu7-openmpi3-parallel-libs
%%%
%%% # Install LLVM/Clang 5.x
%%% [sms](*\#*)  (*\install*) llvm5-compilers-ohpc
%%% \end{lstlisting}
%%%
%%%

\subsection{Integration Test Suite}  \label{appendix:test_suite}

This appendix details the installation and use of the OpenHPC validation test
suite. Each OpenHPC component is equiped with a set of scripts and applications
and the integration of these components is validated in a Jenkins CI 
environment. To facilitate local customization and extension of OpenHPC, we 
provide these tests in a standalone RPM. 

\begin{lstlisting}
[sms](*\#*) (*\install*) test-suite-ohpc
\end{lstlisting}

The RPM creates a user called ohpc-test, and inside that user's home directory 
are directories representing the functional areas of OpenHPC. GNU 
autotools-based configuration files control the building of the tests, and the
BATS framework is used to execute them and collect results. Most components can
be tested individually, but a default configuration is setup to enable 
collective testing.

\begin{lstlisting}
[sms](*\#*) ls /home/ohpc-test/tests/
aclocal.m4      bootstrap  config.guess  COPYRIGHT   libs     Makefile.am  mpi runtimes         user-env-oom
admin           bos        config.sub    dev-tools   LICENSE  Makefile.in  oob test-driver
apps            common     configure     install     lustre   missing perf-tools   test-driver-cmt
autom4te.cache  compilers  configure.ac  install-sh  m4       modules rms-harness  user-env
\end{lstlisting}

Some tests require priviledged execution, so a different set of tests will be
enabled depending in which user executes the configure script. The tests are
further devided in to a 'short' run and a 'long' run. The short run is a subset
of tests to demonstrate basic functionality and should complete in 10-20 
minutes. The long run is comprehensive and can take an hour or more to complete.
Results in Junit format are agregated to allow for ease of analysis.


\begin{lstlisting}
[sms](*\#*) su - ohpc-test
[test@sms ~]$ cd tests
[test@sms ~]$ ./configure
checking for a BSD-compatible install... /bin/install -c
checking whether build environment is sane... yes
checking for a thread-safe mkdir -p... /bin/mkdir -p
checking for gawk... gawk
checking whether make sets $(MAKE)... yes
checking whether make supports nested variables... yes
checking whether make supports nested variables... (cached) yes
checking build system type... x86_64-unknown-linux-gnu
checking host system type... x86_64-unknown-linux-gnu
checking for user root... no
checking if requesting longer-running tests... yes
checking for /etc/pbs.conf... no
checking if enable PXSE toolchain related tests... yes
checking that generated files are newer than configure... done
configure: creating ./config.status
config.status: creating Makefile
config.status: creating common/TEST_ENV
config.status: creating admin/Makefile
config.status: creating user-env/Makefile
config.status: creating bos/Makefile
config.status: creating perf-tools/scalasca/Makefile
config.status: creating perf-tools/tau/Makefile
config.status: creating oob/Makefile
config.status: creating lustre/Makefile
config.status: creating dev-tools/easybuild/Makefile
config.status: creating admin/spack/Makefile
config.status: creating dev-tools/numpy/Makefile
config.status: creating dev-tools/numpy/tests/Makefile
config.status: creating dev-tools/scipy/Makefile
config.status: creating dev-tools/scipy/tests/Makefile
config.status: creating dev-tools/tbb/Makefile
config.status: creating dev-tools/tbb/tests/Makefile
config.status: creating dev-tools/cilk/Makefile
config.status: creating dev-tools/R-base/Makefile
config.status: creating modules/Makefile

--------------------------------------------- SUMMARY
---------------------------------------------

Package version............... : test-suite-1.3.0

Build user.................... : ohpc-test
Build host.................... : sms001
Configure date................ : 2017-03-24 08:36
Build architecture............ : x86_64
MPI Families.................. : mpich mvapich2 openmpi impi
Resource manager ............. : SLURM
Test suite configuration...... : long

Submodule Configuration:

User Environment:
    Packaging tests........... : disabled
    RMS test harness.......... : enabled
    Munge..................... : enabled
    Compilers................. : enabled
    MPI....................... : enabled
    Modules................... : enabled
    OOM....................... : enabled
Dev Tools:
    Autotools................. : enabled
    EasyBuild................. : enabled
    Valgrind.................. : enabled
    R base package............ : enabled
    TBB....................... : enabled
    CILK...................... : enabled
Performance Tools:
    mpiP Profiler........ .... : enabled
    Papi...................... : enabled
    Scalasca.................. : enabled
    TAU....................... : enabled
Libraries:
    Adios .................... : enabled
    Boost .................... : enabled
    Boost MPI................. : enabled
    FFTW...................... : enabled
    GSL....................... : enabled
    HDF5...................... : enabled
    HYPRE..................... : enabled
    IMB....................... : enabled
    Metis..................... : enabled
    MUMPS..................... : enabled
    NetCDF.................... : enabled
    Numpy..................... : enabled
    OCR....................... : enabled
    OPENBLAS.................. : enabled
    PETSc..................... : enabled
    PHDF5..................... : enabled
    ScaLAPACK................. : enabled
    Scipy..................... : enabled
    Superlu................... : enabled
    Superlu_dist.............. : enabled
    Trilinos ................. : enabled
Apps:
    MiniFE.................... : enabled
    MiniDFT................... : disabled
    HPCG...................... : enabled
    PRK....................... : disabled
[test@sms ~]$ make check
Making check in dev-tools/easybuild
make  check-TESTS
PASS: EasyBuild
============================================================================
Testsuite summary for test-suite 1.3.0
============================================================================
# TOTAL: 1
# PASS:  1
# SKIP:  0
# XFAIL: 0
# FAIL:  0
# XPASS: 0
# ERROR: 0
============================================================================
Making check in modules
make  check-TESTS
PASS: init.sh
PASS: lmod_installed
PASS: interactive_commands
PASS: rm_execution
============================================================================
Testsuite summary for test-suite 1.3.0
============================================================================
# TOTAL: 4
# PASS:  4
# SKIP:  0
# XFAIL: 0
# FAIL:  0
# XPASS: 0
# ERROR: 0
============================================================================
Making check in user-env
make  check-TESTS
PASS: init.sh
PASS: mem_limits
PASS: pdsh
PASS: ompi_info
PASS: munge
============================================================================
Testsuite summary for test-suite 1.3.0
============================================================================
# TOTAL: 5
# PASS:  5
# SKIP:  0
# XFAIL: 0
# FAIL:  0
# XPASS: 0
# ERROR: 0
============================================================================
make --no-print-directory check-TESTS
PASS: rms-harness/ohpc-tests/test_mpi_families
PASS: apps/miniFE/ohpc-tests/test_mpi_families
PASS: apps/hpcg/run
PASS: compilers/ohpc-tests/test_compiler_families
PASS: dev-tools/valgrind/ohpc-tests/test_compiler_families
PASS: dev-tools/R-base/ohpc-tests/test_compiler_families
PASS: perf-tools/mpiP/ohpc-tests/test_mpi_families
PASS: libs/fftw/ohpc-tests/test_mpi_families
PASS: libs/adios/ohpc-tests/test_mpi_families
PASS: libs/boost/ohpc-tests/test_mpi_families
PASS: libs/boost-mpi/ohpc-tests/test_mpi_families
PASS: libs/gsl/ohpc-tests/test_compiler_families
PASS: libs/imb/ohpc-tests/test_mpi_families
PASS: libs/hdf5/ohpc-tests/test_compiler_families
PASS: libs/phdf5/ohpc-tests/test_mpi_families
PASS: libs/hypre/ohpc-tests/test_mpi_families
PASS: libs/metis/ohpc-tests/test_compiler_families
PASS: libs/mumps/ohpc-tests/test_mpi_families
PASS: libs/netcdf/ohpc-tests/test_mpi_families
PASS: runtimes/ocr/ohpc-tests/test_compiler_families
PASS: libs/openblas/ohpc-tests/test_compiler_families
PASS: user-env-oom/ohpc-tests/test_compiler_families
PASS: perf-tools/papi/ohpc-tests/test_compiler_families
PASS: libs/petsc/ohpc-tests/test_mpi_families
PASS: mpi/ohpc-tests/test_mpi_families
PASS: dev-tools/numpy/ohpc-tests/test_mpi_families
PASS: libs/scalapack/ohpc-tests/test_mpi_families
PASS: dev-tools/scipy/ohpc-tests/test_mpi_families
PASS: libs/superlu/ohpc-tests/test_compiler_families
PASS: libs/superlu_dist/ohpc-tests/test_mpi_families
PASS: perf-tools/scalasca/ohpc-tests/test_mpi_families
PASS: perf-tools/tau/ohpc-tests/test_mpi_families
PASS: dev-tools/autotools/run
PASS: dev-tools/tbb/ohpc-tests/test_compiler_families
PASS: dev-tools/cilk/ohpc-tests/test_compiler_families
PASS: libs/trilinos/ohpc-tests/test_mpi_families
============================================================================
Testsuite summary for test-suite 1.3.0
============================================================================
# TOTAL: 36
# PASS:  36
# SKIP:  0
# XFAIL: 0
# FAIL:  0
# XPASS: 0
# ERROR: 0
============================================================================

real    108m52.037s
user    55m18.016s
sys 18m57.938s
Install exit status = 0
\end{lstlisting}

\clearpage

\subsection{Customization} \label{appendix:customization}

\subsubsection{Adding local Lmod modules to \OHPC{} hierarchy} \label{appendix:modulefiles}
Locally installed applications can easily be integrated in to \OHPC{} systems by
following the Lmod convention laid out by the provided packages. Two sample
module files are included in the \texttt{examples-ohpc} package\textemdash one
representing an application with no compiler or MPI runtime dependencies, and
one dependent on OpenMPI and the \GNU{} toolchain. Simply copy these files to the
prescribed locations, and the \texttt{lmod} application should pick them up
automatically.

\begin{lstlisting}[alsoletter={/,.},morekeywords={example1/1.0, example2/1.0}]
[sms](*\#*) mkdir /opt/ohpc/pub/modulefiles/example1
[sms](*\#*) cp /opt/ohpc/pub/examples/example.modulefile \
    /opt/ohpc/pub/modulefiles/example1/1.0
[sms](*\#*) mkdir /opt/ohpc/pub/moduledeps/gnu7-openmpi3/example2
[sms](*\#*) cp /opt/ohpc/pub/examples/example-mpi-dependent.modulefile \
    /opt/ohpc/pub/moduledeps/gnu7-openmpi3/example2/1.0
[sms](*\#*) module avail

----------------------------------- /opt/ohpc/pub/moduledeps/gnu7-openmpi3 -----------------------------------
   adios/1.12.0    imb/2018.0          netcdf-fortran/4.4.4    ptscotch/6.0.4     sionlib/1.7.1
   boost/1.65.1    mpi4py/2.0.0        netcdf/4.4.1.1          scalapack/2.0.2    slepc/3.7.4
   example2/1.0    mpiP/3.4.1          petsc/3.7.6             scalasca/2.3.1     superlu_dist/4.2
   fftw/3.3.6      mumps/5.1.1         phdf5/1.10.1            scipy/0.19.1       tau/2.26.1
   hypre/2.11.2    netcdf-cxx/4.3.0    pnetcdf/1.8.1           scorep/3.1         trilinos/12.10.1

--------------------------------------- /opt/ohpc/pub/moduledeps/gnu7 ----------------------------------------
   R/3.4.2        metis/5.1.0     ocr/1.0.1              pdtoolkit/3.24    superlu/5.2.1
   gsl/2.4        mpich/3.2       openblas/0.2.20        plasma/2.8.0
   hdf5/1.10.1    numpy/1.13.1    openmpi3/3.0.0  (L)    scotch/6.0.4

---------------------------------------- /opt/ohpc/admin/modulefiles -----------------------------------------
   spack/0.10.0

----------------------------------------- /opt/ohpc/pub/modulefiles ------------------------------------------
   EasyBuild/3.4.1         cmake/3.9.2         hwloc/1.11.8        pmix/1.2.3             valgrind/3.13.0
   autotools        (L)    example1/1.0 (L)    llvm5/5.0.0         prun/1.2        (L)
   clustershell/1.8        gnu7/7.2.0   (L)    ohpc         (L)    singularity/2.4

  Where:
   L:  Module is loaded

Use "module spider" to find all possible modules.
Use "module keyword key1 key2 ..." to search for all possible modules matching any of the "keys".
\end{lstlisting}


\newpage
\subsubsection{Rebuilding Packages from Source}  \label{appendix:rpmbuild}
Users of \OHPC{} may find it desirable to rebuild one of the supplied packages
to apply build customizations or satisfy local requirements. One way to
accomplish this is to install the appropriate source RPM, modify the spec file
as needed, and rebuild to obtain an updated binary RPM. A brief example using
the FFTW library is highlighted below.  Note that the source RPMs can be downloaded from the
community build server at \href{https://build.openhpc.community}
{\color{blue}{https://build.openhpc.community}} via a web browser or directly
via \texttt{yum} as highlighted below. The \OHPC{} build system design
leverages several keywords to control the choice of compiler and MPI families
for relevant development libraries and the \texttt{rpmbuild} example
illustrates how to override the default mpi\_family.

\begin{lstlisting}[language=bash,keywords={},basicstyle=\fontencoding{T1}\footnotesize\ttfamily,
    literate={ARCH}{\arch{}}1 {-}{-}1]
# Install rpm-build package and yum tools from base OS distro
[test@sms ~]$ sudo (*\install*) rpm-build yum-utils

# Install FFTW's build dependencies
[test@sms ~]$ sudo yum-builddep fftw-gnu7-openmpi3-ohpc

# Download SRPM from OpenHPC repository and install locally
[test@sms ~]$ yumdownloader --source fftw-gnu7-openmpi3-ohpc
[test@sms ~]$ rpm -i ./fftw-gnu7-openmpi3-ohpc-3.3.6-28.11.src.rpm

# Modify spec file as desired
[test@sms ~]$ cd ~/rpmbuild/SPECS
[test@sms ~rpmbuild/SPECS]$ perl -pi -e "s/enable-static=no/enable-static=yes/" fftw.spec

# Increment RPM release so package manager will see an update
[test@sms ~rpmbuild/SPECS]$ perl -pi -e "s/Release:   28.11/Release:   29.1/" fftw.spec

# Rebuild binary RPM. Note that additional directives can be specified to modify build
[test@sms ~rpmbuild/SPECS]$ rpmbuild -bb --define "mpi_family mpich" fftw.spec

# Install the new package
[test@sms ~]$ sudo (*\install*) ~test/rpmbuild/RPMS/ARCH/fftw-gnu-mpich-ohpc-3.3.6-29.1.ARCH.rpm
\end{lstlisting}


\clearpage

\appendix
\section*{Appendix - Package Manifest}
\addcontentsline{toc}{section}{Appendix - Package Manifest}
\renewcommand{\thesubsection}{\Alph{subsection}}

%\subsection{Package Manifest}

This appendix provides a summary the underlying RPM packages that are available
as part of this \FSP{} release. These packages are presented in groupings
based on their general functionality that are organized as follows:

\begin{itemize*}
\item Administrative tools
\item Provisioning
\item Resource management
\item Compiler families
\item MPI families
\item Development tools
\item Performance analysis tools
\item Distro support packages and dependencies
\item IO Libraries
\item Serial Libraries
\item Parallel Libraries
\end{itemize*}

What follows in this Appendix are tables that summarize the packages in each
group including information on the RPM name, version, brief summary, and the web
URL where additional information can be contained for the component. Many of the 3rd
party community libraries that are pre-packaged with \FSP{} are built using
multiple compiler and MPI families. In these cases, the RPM package name
includes delimiters identifying the development environment for which each
package build is targeted.  Additional information on the \FSP{} package
naming scheme is presented in \S\ref{sec:3rdparty}.


\definecolor{Gray}{gray}{0.5}

\newcommand{\firstColWidth}{3.5cm}
\newcommand{\secondColWidth}{1.5cm}

% Administration Tools 

\captionsetup{justification=raggedright,singlelinecheck=false}

\vspace*{1.0cm}

\newcommand{\captionSpace}{-0.15cm}
\newcommand{\tabSpaceBot}{1.0cm}

\begin{table}[h]
\caption{\bf Administrative Tools} \vspace*{\captionSpace{}}
\input data/manifest/admin
\end{table}
\vspace*{0.5cm}

\renewcommand{\firstColWidth}{4.5cm}
\renewcommand{\secondColWidth}{2.0cm}

% Provisioning

\begin{table}[h!]
\caption{\bf Provisioning} \vspace*{\captionSpace{}}
\input data/manifest/provisioning
\vspace*{\tabSpaceBot{}}
\end{table} 

% Resource Management
\begin{table}[h!]
\caption{\bf Resource Management} \vspace*{\captionSpace{}}
\input data/manifest/rms
\vspace*{\tabSpaceBot{}}
\end{table}

% Compiler Families
\begin{table}[h!]
\caption{\bf Compiler Families} \vspace*{\captionSpace{}}
\input data/manifest/compiler-families
\vspace*{\tabSpaceBot{}}
\end{table}

% MPI Families
\begin{table}[h!]
\caption{\bf MPI Families} \vspace*{\captionSpace{}}
\input data/manifest/mpi-families
\vspace*{\tabSpaceBot{}}
\end{table}

\renewcommand{\firstColWidth}{4.5cm}

% Development Tools
\begin{table}[h!]
\caption{\bf Development Tools} \vspace*{\captionSpace{}}
\input data/manifest/dev-tools
\vspace*{\tabSpaceBot{}}
\end{table}

% Perf Tools
\begin{table}[h!]
\caption{\bf Performance Analysis Tools} \vspace*{\captionSpace{}}
\input data/manifest/perf-tools
\vspace*{\tabSpaceBot{}}
\end{table}

% Distro Packages
\begin{table}[h!]
\caption{\bf Distro Support Packages/Dependencies} \vspace*{\captionSpace{}}
\input data/manifest/distro-packages
\vspace*{\tabSpaceBot{}}
\end{table}

\renewcommand{\firstColWidth}{4.75cm}
\renewcommand{\secondColWidth}{1.75cm}

% Lustre
\begin{table}[h!]
\caption{\bf Lustre} \vspace*{\captionSpace{}}
\input data/manifest/lustre
\vspace*{\tabSpaceBot{}}
\end{table}

% IO Libs
\begin{table}[h!]
\caption{\bf IO Libraries} \vspace*{\captionSpace{}}
\input data/manifest/io-libs
\vspace*{\tabSpaceBot{}}
\end{table}

% Serial libs
\begin{table}[h!]
\caption{\bf Serial Libraries} \vspace*{\captionSpace{}} 
\input data/manifest/serial-libs
\vspace*{\tabSpaceBot{}}
\end{table}

% Parallel libs
\begin{table}[h!]
\caption{\bf Parallel Libraries} \vspace*{\captionSpace{}} 
\input data/manifest/parallel-libs
\vspace*{\tabSpaceBot{}}
\end{table}






\clearpage
\subsection{Package Signatures}
%\addcontentsline{toc}{section}{Appendix B - Package Signatures}

All of the RPMs provided via the \OHPC{} repository are signed with a GPG
signature. By default, the underlying package managers will verify these signatures during
installation to ensure that packages have not been altered. The RPMs can also
be manually verified and the public signing key fingerprint for the latest
repository is shown below: \\

\texttt{Fingerprint: 5392 744D 3C54 3ED5 7847  65E6 8A30 6019 {\bf DA565C6C}} \\

\noindent The following command can be used to verify an RPM once it
has been downloaded locally by confirming if the package is signed, and if so,
indicating which key was used to sign it. The example below highlights usage
for a local copy of the \texttt{docs-ohpc} package and illustrates how the {\em
key ID} matches the fingerprint shown above.

\begin{lstlisting}[language=bash,keywords={}]
[sms](*\#*) rpm --checksig -v docs-ohpc-*.rpm
docs-ohpc-2.0.0-72.1.ohpc.2.0.x86_64.rpm:
    Header V3 RSA/SHA1 Signature, key ID da565c6c: OK
    Header SHA256 digest: OK
    Header SHA1 digest: OK
    Payload SHA256 digest: OK
    V3 RSA/SHA1 Signature, key ID da565c6c: OK
    MD5 digest: OK

\end{lstlisting}






\end{document}

    



    

