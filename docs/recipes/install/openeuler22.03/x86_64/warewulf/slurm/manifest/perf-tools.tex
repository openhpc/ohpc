\newcolumntype{C}[1]{>{\centering}p{#1}}
\newcolumntype{L}[1]{>{\raggedleft}p{#1}}
\newcolumntype{C}[1]{>{\centering}p{#1}}
\newcolumntype{L}[1]{>{\raggedleft}p{#1}}
\small
\begin{tabularx}{\textwidth}{L{\firstColWidth{}}|C{\secondColWidth{}}|X}
\toprule
{\bf RPM Package Name} & {\bf Version} & {\bf Info/URL}  \\
\midrule

% <-- begin entry for dimemas
dimemas-gnu12-mpich-ohpc &
\multirow{6}{*}{5.4.2} &
\multirow{6}{\linewidth}{Dimemas tool. \newline {\color{logoblue} \url{https://tools.bsc.es}}}\\
dimemas-gnu12-mvapich2-ohpc &
& \\
dimemas-gnu12-openmpi4-ohpc &
& \\
dimemas-gnu13-mpich-ohpc &
& \\
dimemas-gnu13-mvapich2-ohpc &
& \\
dimemas-gnu13-openmpi5-ohpc &
& \\
\hline
% <-- end entry for dimemas

% <-- begin entry for extrae
extrae-gnu12-mpich-ohpc &
\multirow{6}{*}{3.8.3} &
\multirow{6}{\linewidth}{Extrae tool. \newline {\color{logoblue} \url{https://tools.bsc.es}}}\\
extrae-gnu12-mvapich2-ohpc &
& \\
extrae-gnu12-openmpi4-ohpc &
& \\
extrae-gnu13-mpich-ohpc &
& \\
extrae-gnu13-mvapich2-ohpc &
& \\
extrae-gnu13-openmpi5-ohpc &
& \\
\hline
% <-- end entry for extrae

% <-- begin entry for geopm
geopm-gnu12-mpich-ohpc &
\multirow{6}{*}{1.1.0} &
\multirow{6}{\linewidth}{Global Extensible Open Power Manager. \newline {\color{logoblue} \url{https://geopm.github.io}}}\\
geopm-gnu12-mvapich2-ohpc &
& \\
geopm-gnu12-openmpi4-ohpc &
& \\
geopm-gnu13-mpich-ohpc &
& \\
geopm-gnu13-mvapich2-ohpc &
& \\
geopm-gnu13-openmpi5-ohpc &
& \\
\hline
% <-- end entry for geopm

% <-- begin entry for imb
imb-gnu12-mpich-ohpc &
\multirow{6}{*}{2021.3} &
\multirow{6}{\linewidth}{Intel MPI Benchmarks (IMB). \newline {\color{logoblue} \url{https://software.intel.com/en-us/articles/intel-mpi-benchmarks}}}\\
imb-gnu12-mvapich2-ohpc &
& \\
imb-gnu12-openmpi4-ohpc &
& \\
imb-gnu13-mpich-ohpc &
& \\
imb-gnu13-mvapich2-ohpc &
& \\
imb-gnu13-openmpi5-ohpc &
& \\
\hline
% <-- end entry for imb

% <-- begin entry for likwid
likwid-gnu12-ohpc &
\multirow{1}{*}{5.2.2} &
\multirow{2}{\linewidth}{Performance tools for the Linux console. \newline {\color{logoblue} \url{https://github.com/RRZE-HPC/likwid}}}\\
\cline{1-2} likwid-gnu13-ohpc &\multirow{1}{*}{5.3.0}
& \\
\hline
% <-- end entry for likwid

% <-- begin entry for omb
omb-gnu12-mpich-ohpc &
\multirow{3}{*}{6.1} &
\multirow{6}{\linewidth}{OSU Micro-benchmarks. \newline {\color{logoblue} \url{https://mvapich.cse.ohio-state.edu/benchmarks}}}\\
omb-gnu12-mvapich2-ohpc &
& \\
omb-gnu12-openmpi4-ohpc &
& \\
\cline{1-2} omb-gnu13-mpich-ohpc &\multirow{3}{*}{7.3}
& \\
omb-gnu13-mvapich2-ohpc &
& \\
omb-gnu13-openmpi5-ohpc &
& \\
\hline
% <-- end entry for omb

% <-- begin entry for paraver-ohpc
\multirow{2}{*}{paraver-ohpc} &
\multirow{2}{*}{4.10.4} &
Paraver. \newline { \color{logoblue} \url{https://tools.bsc.es}}
\\ \hline
% <-- end entry for paraver-ohpc

% <-- begin entry for papi-ohpc
\multirow{2}{*}{papi-ohpc} &
\multirow{2}{*}{6.0.0} &
Performance Application Programming Interface. \newline { \color{logoblue} \url{http://icl.cs.utk.edu/papi}}
\\ \hline
% <-- end entry for papi-ohpc

% <-- begin entry for pdtoolkit
pdtoolkit-gnu12-ohpc &
\multirow{2}{*}{3.25.1} &
\multirow{2}{\linewidth}{PDT is a framework for analyzing source code. \newline {\color{logoblue} \url{http://www.cs.uoregon.edu/Research/pdt}}}\\
\bottomrule
\end{tabularx}
