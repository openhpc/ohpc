\subsection{Upgrading OpenHPC Packages}  \label{appendix:upgrade}


As newer \OHPC{} releases are made available, users are encouraged to upgrade
their locally installed packages against the latest repository versions to
obtain access to bug fixes and newer component versions. This can be
accomplished with the underlying package manager as \OHPC{} packaging maintains
versioning state across releases. Also, package builds available from the
\OHPC{} repositories have ``\texttt{-ohpc}'' appended to their names so
wild cards can be used as a simple way to obtain updates. The following general
procedure highlights a method for upgrading existing installations.  Note that
when upgrading from a minor release older than v\OHPCVerTree{}, you will first
need to update your local \OHPC{} repository configuration to point against the
v\OHPCVerTree{} release (or update your locally hosted mirror). Refer to
\S\ref{sec:enable_repo} for more details on enabling the latest
repository. Alternatively, when upgrading between micro releases on the same
branch (e.g. from v\OHPCVerTree{} to \OHPCVerTree{}.1), there is no need to
adjust local package manager configurations when using the public repository as
rolling updates are pre-configured.
 
\begin{enumerate*}
\item (Optional) Ensure repo metadata is current (on head node and in chroot
  location(s)). Package managers will naturally do this on their own over time,
  but if you are wanting to access updates immediately after a new release,
  the following can be used to sync to the latest.

\begin{lstlisting}[language=bash,keywords={}]
[sms](*\#*)  (*\clean*)
[sms](*\#*)  (*\chrootclean*)
\end{lstlisting}

\item Upgrade master (SMS) node

\begin{lstlisting}[language=bash,keywords={}]
[sms](*\#*)  (*\upgrade*) "*-ohpc"
\end{lstlisting}
  
\item Upgrade packages in compute image

\begin{lstlisting}[language=bash,keywords={}]
[sms](*\#*)  (*\chrootupgrade*) "*-ohpc"
\end{lstlisting}
  
\item Rebuild image(s)

\iftoggleverb{isWarewulf}
\begin{lstlisting}[language=bash,keywords={}]
[sms](*\#*) wwvnfs --chroot $CHROOT
\end{lstlisting}
\fi

\iftoggleverb{isxCAT}
\begin{lstlisting}[language=bash,keywords={},basicstyle=\fontencoding{T1}\fontsize{8.0}{10}\ttfamily,
    literate={-}{-}1 {BOSVER}{\baseos{}}1 {ARCH}{\arch{}}1]
[sms](*\#*) packimage centos7.3-x86_64-netboot-compute
\end{lstlisting}
\fi

\end{enumerate*}

\noindent In the case where packages were upgraded within the chroot compute image,
you will need to reboot the compute nodes when convenient to enable the
changes.

\iftoggleverb{isCentOS}
\input common/gcc7_upgrade_centos.tex
\else
\input common/gcc7_upgrade_sles.tex
\fi
