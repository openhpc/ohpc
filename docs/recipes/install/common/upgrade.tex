\subsection{Upgrading OpenHPC Packages}  \label{appendix:upgrade}


As newer \OHPC{} releases are made available, users are encouraged to upgrade
their locally installed packages against the latest repository versions to
obtain access to bug fixes and newer component versions. This can be
accomplished with the underlying package manager as \OHPC{} packaging maintains
versioning state across releases. Also, package builds available from the
\OHPC{} repositories have ``\texttt{-ohpc}'' appended to their names so
wildcards can be used as a simple way to obtain updates. The following general
procedure highlights a method for upgrading existing installations.  Note that
when upgrading from a minor release older than v\OHPCVerTree{}, you will first
need to update your local \OHPC{} repository configuration to point against the
v\OHPCVerTree{} release (or update your locally hosted mirror). Refer to
\S\ref{sec:enable_repo} for details on enabling the latest repository. Note
that when upgrading between micro releases on the same branch (e.g. from
v\OHPCVerTree{} to v\OHPCVerTree{}.1}), there is no need to adjust local
  package manager configurations when using the public repository as rolling
  updates are pre-configured.

\begin{enumerate}
\item Ensure repodata is current
\item Upgrade master (SMS) node
\item Upgrade packages in compute image
\item Rebuild vnfs image
\end{enumerate}
