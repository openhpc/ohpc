\conman{} is a serial console management program designed to support a large
number of console devices and simultaneous users. It supports logging console
device output and connecting to compute node consoles via IPMI
serial-over-lan. Installation and example configuration is outlined below.

% begin_ohpc_run
% ohpc_validation_newline
% ohpc_validation_comment Optionally, enable conman and configure
% ohpc_command if [[ ${enable_ipmisol} -eq 1 ]];then
% ohpc_indent 5
\begin{lstlisting}[language=bash,keywords={},upquote=true]
# Install conman to provide a front-end to compute consoles and log output
[sms](*\#*) (*\install*) conman-ohpc

# Configure conman for computes (note your IPMI password is required for console access)
[sms](*\#*) for ((i=0; i<$num_computes; i++)) ; do
              echo -n 'CONSOLE name="'${c_name[$i]}'" dev="ipmi:'${c_bmc[$i]}'" '
              echo 'ipmiopts="'U:${bmc_username},P:${IPMI_PASSWORD:-undefined},W:solpayloadsize'"'
        done >> /etc/conman.conf

# Enable and start conman
[sms](*\#*) systemctl enable conman
[sms](*\#*) systemctl start conman
\end{lstlisting}
% ohpc_indent 0
% ohpc_command fi
% end_ohpc_run

\iftoggleverb{isWarewulf}
\noindent Note that an additional kernel boot option is typically necessary to
enable serial console output. This option is highlighted in \S\ref{sec:optional_kargs} after
compute nodes have been registered with the provisioning system.
\fi

\iftoggleverb{isxCAT}
\noindent Note that additional options are typically necessary to
enable serial console output. These are setup during the node registration
process in \S\ref{sec:xcat_add_nodes} 
\fi


