To begin, enable use of the \OHPC{} repository by adding it to the local list
of available package repositories. Note that this requires network access from
your {\em master} server to the \OHPC{} repository, or alternatively, that
the \OHPC{} repository be mirrored locally.  In cases where network external
connectivity is available, \OHPC{} provides an \texttt{ohpc-release} package
that includes GPG keys for package signing and enabling the repository.  The
example which follows illustrates installation of the \texttt{ohpc-release}
package directly from the \OHPC{} build server.

\iftoggleverb{isCentOS}
% CentOS
\begin{lstlisting}[language=bash,keywords={},basicstyle=\fontencoding{T1}\fontsize{7.6}{10}\ttfamily,
	literate={VER}{\OHPCVerTree{}}1 {OSREPO}{\OSTree{}}1 {TAG}{\OSTag{}}1 {ARCH}{\arch{}}1 {-}{-}1]
[sms](*\#*) yum install https://support2.brightcomputing.com/ohpc/ohpc-reposetup-2.0_100008_cm9.0_590aab9b26.x86_64.rpm
\end{lstlisting}
\else
% non-CentOS
\begin{lstlisting}[language=bash,keywords={},basicstyle=\fontencoding{T1}\fontsize{7.8}{10}\ttfamily,
	literate={VER}{\OHPCVerTree{}}1 {OSREPO}{\OSTree{}}1 {TAG}{\OSTag{}}1 {ARCH}{\arch{}}1 {-}{-}1]
[sms](*\#*) rpm -ivh https://support2.brightcomputing.com/ohpc/ohpc-reposetup-2.0_100008_cm9.0_590aab9b26.x86_64.rpm
\end{lstlisting}
\fi

After installing the ohpc-reposetup Bright package on the {\em master} node, the 
ohpc-reposetup module needs to be loaded, then the ohpc-reposetup.sh script run 
to enable OpenHPC in the software images and the {\em master} node

\begin{lstlisting}[language=bash,keywords={},basicstyle=\fontencoding{T1}\fontsize{7.6}{10}\ttfamily,
        literate={VER}{\OHPCVerTree{}}1 {OSREPO}{\OSTree{}}1 {TAG}{\OSTag{}}1 {ARCH}{\arch{}}1 {-}{-}1]
	[sms](*\#*) module load ohpc-reposetup/2.0/ohpc-reposetup.module
	[sms](*\#*) ohpc-reposetup.sh
\end{lstlisting}
