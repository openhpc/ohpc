\subsection{Integration Test Suite}  \label{appendix:test_suite}

This appendix details the installation and use of the \OHPC validation test
suite. Each \OHPC component is equipped with a set of scripts and applications
to test the integration of these components in a Jenkins CI 
environment. To facilitate local customization and extension of \OHPC, we 
provide these tests in a standalone RPM. 

\begin{lstlisting}
[sms](*\#*) (*\install*) test-suite-ohpc
\end{lstlisting}

The RPM creates a user called ohpc-test, and inside that user's home directory 
are directories representing the functional areas of \OHPC. GNU 
autotools-based configuration files control the building of the tests, and the
BATS framework is used to execute them and collect results. 

Some tests require privileged execution, so a different set of tests will be
enabled depending in which user executes the configure script. Non-privileged
tests are submitted as jobs through the \rms{} resource manager. The tests are
further divided in to a 'short' run and a 'long' run. The short run is a subset
of tests to demonstrate basic functionality and should complete in 10-20 
minutes. The long run is comprehensive and can take an hour or more to complete.
Results in JUnit format are aggregated to allow for ease of analysis.

Most components can be tested individually, but a default configuration is setup 
to enable collective testing. To test a single component, use the \texttt{configure}
script to disable all tests, then re-enable the desired test to run. Some output
is omitted for the sake of brevity.

\begin{lstlisting}[literate={RMS}{\rms{}}1 {ARCH}{\arch{}}1]
[sms](*\#*) su - ohpc-test
[test@sms ~]$ cd tests
[test@sms ~]$ ./configure --disable-all --enable-fftw
checking for a BSD-compatible install... /bin/install -c
checking whether build environment is sane... yes
...
--------------------------------------------- SUMMARY ---------------------------------------------

Package version............... : test-suite-1.3.0

Build user.................... : ohpc-test
Build host.................... : sms001
Configure date................ : 2017-03-24 15:41
Build architecture............ : ARCH
Compiler Families............. : gnu
MPI Families.................. : mpich mvapich2 openmpi
Resource manager ............. : RMS
Test suite configuration...... : short
...
Libraries:
    Adios .................... : disabled
    Boost .................... : disabled
    Boost MPI................. : disabled
    FFTW...................... : enabled
    GSL....................... : disabled
    HDF5...................... : disabled
    HYPRE..................... : disabled
...
\end{lstlisting}

Many \OHPC components exist in multiple flavors to support multiple compiler
and MPI runtime permutations, and the test suite takes this in to account by
iterating through these combinations by default. If \texttt{make check} is executed
from the root test directory, all versions of a library will be exercised.

\begin{lstlisting}[literate={RMS}{\rms{}}1]
[test@sms ~]$ make check
make --no-print-directory check-TESTS
PASS: libs/fftw/ohpc-tests/test_mpi_families
============================================================================
Testsuite summary for test-suite 1.3.0
============================================================================
# TOTAL: 1
# PASS:  1
# SKIP:  0
# XFAIL: 0
# FAIL:  0
# XPASS: 0
# ERROR: 0
============================================================================
[test@sms ~]$ cat libs/fftw/tests/family-gnu-*/rm_execution.log 
1..3
ok 1 [libs/FFTW] Serial C binary runs under resource manager (RMS/gnu/mpich)
ok 2 [libs/FFTW] MPI C binary runs under resource manager (RMS/gnu/mpich)
ok 3 [libs/FFTW] Serial Fortran binary runs under resource manager (RMS/gnu/mpich)
PASS rm_execution (exit status: 0)
1..3
ok 1 [libs/FFTW] Serial C binary runs under resource manager (RMS/gnu/mvapich2)
ok 2 [libs/FFTW] MPI C binary runs under resource manager (RMS/gnu/mvapich2)
ok 3 [libs/FFTW] Serial Fortran binary runs under resource manager (RMS/gnu/mvapich2)
PASS rm_execution (exit status: 0)
1..3
ok 1 [libs/FFTW] Serial C binary runs under resource manager (RMS/gnu/openmpi)
ok 2 [libs/FFTW] MPI C binary runs under resource manager (RMS/gnu/openmpi)
ok 3 [libs/FFTW] Serial Fortran binary runs under resource manager (RMS/gnu/openmpi)
PASS rm_execution (exit status: 0)
\end{lstlisting}
