At this point, all of the packages necessary to use \xCAT{} on the {\em master}
host should be installed. Next, we enable support for local provisioning using
a second private interface (refer to Figure~\ref{fig:physical_arch}) and
register this network interface with \xCAT{}.  

% begin_ohpc_run
% ohpc_comment_header Complete basic xCAT setup for master node \ref{sec:setup_xcat}
%\begin{verbatim}
\begin{lstlisting}[language=bash,literate={-}{-}1,keywords={},upquote=true,keepspaces]
# Enable internal interface for provisioning
[sms](*\#*) ip link set dev ${sms_eth_internal} up
[sms](*\#*) ip address add ${sms_ip}/${internal_netmask} broadcast + dev ${sms_eth_internal}

# Register internal provisioning interface with xCAT for DHCP
[sms](*\#*) chdef -t site dhcpinterfaces="xcatmn|${sms_eth_internal}"

\end{lstlisting}
%\end{verbatim}
% end_ohpc_run


\noindent \xCAT{} requires a network domain name specification for system-wide name
resolution. This value can be set to match your local DNS schema or given a
unique identifier such as ``local''. In this recipe, we leverage the
\texttt{\$domain\_name} variable to define as follows:

% begin_ohpc_run
% ohpc_validation_newline
% ohpc_validation_comment Define local domainname
\begin{lstlisting}[language=bash,keywords={},upquote=true,basicstyle=\footnotesize\ttfamily,literate={BOSVER}{\baseos{}}1]
[sms](*\#*) chdef -t site domain=${domain_name}
\end{lstlisting}
