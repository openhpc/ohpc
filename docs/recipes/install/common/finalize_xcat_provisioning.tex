\subsection{Finalizing provisioning configuration} \label{sec:assemble_bootstrap}

To finalize the \xCAT{} provisioning configuration, this section first highlights
packing of the stateless image from the chroot environment followed by the
registration of desired compute nodes. To assemble the final compute image use
\texttt{packimage} as follows:

% begin_ohpc_run
% ohpc_validation_comment Pack compute image
\begin{lstlisting}[language=bash,keywords={},basicstyle=\fontencoding{T1}\fontsize{8.0}{10}\ttfamily,
    literate={-}{-}1 {BOSVER}{\baseos{}}1 {ARCH}{\arch{}}1]
[sms](*\#*) packimage BOSVER-ARCH-netboot-compute
\end{lstlisting}
% end_ohpc_run

%\subsubsection{Register nodes for provisioning}

Next, we define the desired network settings for each desired compute node
with the underlying provisioning system.  Note the use of variable names for
the desired compute hostnames, node IPs, and MAC addresses which should be
modified to accommodate local settings and hardware. These hosts are grouped
logically into an \xCAT{} group named {\em compute} to facilitate group-level
commands used later in the recipe.

%%% Also included in these steps are commands
%%% to enable \Warewulf{} to manage IPoIB settings and corresponding definitions of
%%% IPoIB addresses for the compute nodes. This is typically optional unless you
%%% are planning to include a \Lustre{} client mount over \InfiniBand{}. The final step
%%% in this process associates the VNFS image assembled in previous steps with the
%%% newly defined compute nodes, utilizing the user credential files and munge key
%%% that were imported in \S\ref{sec:file_import}.


