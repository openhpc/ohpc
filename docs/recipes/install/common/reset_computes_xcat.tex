Prior to booting the {\em compute} hosts, we set them up to the network boot
mode:

% begin_ohpc_run
% ohpc_comment_header Set nodes to net boot \ref{sec:boot_computes}
\begin{lstlisting}[language=bash,keywords={},upquote=true]
[sms](*\#*) rsetboot compute net
\end{lstlisting} 
% end_ohpc_run

At this point, the {\em master} server should be able to boot the newly defined
compute nodes. This is done by using the \texttt{rpower} \xCAT{} command
leveraging the IPMI protocol set up during the the {\em compute} node definition
in \S~\ref{sec:assemble_bootstrap}. The following power cycles each of the
desired hosts.


% begin_ohpc_run
% ohpc_comment_header Boot compute nodes \ref{sec:boot_computes}
\begin{lstlisting}[language=bash,keywords={},upquote=true]
[sms](*\#*) rpower compute reset
\end{lstlisting} 
% end_ohpc_run

Once kicked off, the boot process should take about \nottoggle{isstateful}{5
minutes (depending on BIOS post times)}{15 to 30 minutes}.  You can verify that
the compute hosts are available via ssh, or via parallel ssh tools to multiple
hosts.  \xCAT{} provides a \texttt{psh} command which accomplish this while
taking advantage of defined node names and groups. For example:

\begin{lstlisting}[language=bash]
[sms](*\#*) psh compute uptime
c1:  08:53:46 up 5 days, 18:30,  0 users,  load average: 0.00, 0.01, 0.05
c2:  08:53:51 up 5 days, 18:29,  0 users,  load average: 0.00, 0.01, 0.05
c3:  08:53:44 up 5 days, 18:30,  0 users,  load average: 0.00, 0.01, 0.05
c4:  08:53:36 up 5 days, 18:30,  0 users,  load average: 0.00, 0.01, 0.05
\end{lstlisting}
Another  useful way of monitoring provisioning is to use  the \texttt{rcons} command,
which displays serial console for a selected node. Note that the escape sequence
is \texttt{CTRL-e CTRL-c .} 
