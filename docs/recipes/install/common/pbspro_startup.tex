In section \S\ref{sec:basic_install}, the \rms{} workload manager was installed
and configured for use on both the {\em master} host and {\em compute} node
instances. With the cluster nodes up and functional, we can now startup the
resource manager services on the {\em master} host in preparation for running 
user jobs.

The following command can be used to startup the necessary services to support
resource management under \rms{}.

% begin_ohpc_run
% ohpc_comment_header Resource Manager Startup \ref{sec:rms_startup}
\begin{lstlisting}[language=bash,keywords={}]
# start pbspro daemons on master host
[sms](*\#*) systemctl enable pbs
[sms](*\#*) systemctl start pbs
\end{lstlisting}
% end_ohpc_run

In the default configuration, the {\em compute} hosts have not been added to 
the \rms{} resource manager. To add the {\em compute} hosts, execute 
the following for-loop:

% begin_ohpc_run
% ohpc_comment_header Resource Manager Startup \ref{sec:rms_startup}
\begin{lstlisting}[language=bash,keywords={}]
# configure compute hosts with pbspro
[sms](*\#*) for I in {1..4}; do
           qmgr -c "create node c${I}"
        done

\end{lstlisting}
% end_ohpc_run