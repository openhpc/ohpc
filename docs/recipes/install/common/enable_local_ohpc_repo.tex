\OHPC{} provides an online package repository, but here we setup and use a local
mirror on the SMS host for \OHPC{} package installation. This is preferable for
a stateful install, because {\em compute} nodes can access it on the local
network.  


Assuming that the {\em master} has outside network access, the mirror is
created  by downloading a tarball of the repository, unpacking it and running a
provided script to configure the package manager.

% begin_ohpc_run
% ohpc_validation_newline
% ohpc_comment_header Download OHPC repo and create local mirror \ref{sec:enable_repo}
\begin{lstlisting}[language=bash,keywords={},basicstyle=\fontencoding{T1}\fontsize{7.6}{10}\ttfamily,
	literate={VER}{\OHPCVerTree{}}1 {OSTREE}{\OSTree{}}1 {TAG}{\OSTag{}}1 {ARCH}{\arch{}}1 {-}{-}1 
        {VERLONG}{\OHPCVersion{}}1]
# Download OHPC tarball for local installation
[sms](*\#*) wget http://repos.openhpc.community/dist/VERLONG/OpenHPC-VERLONG.OSTREE.ARCH.tar
# Create directory if necessary
[sms](*\#*) mkdir -p $ohpc_repo_dir
#Unpack
[sms](*\#*) tar xvf OpenHPC-VERLONG.OSTREE.ARCH.tar -C $ohpc_repo_dir
#Create local mirror repository
[sms](*\#*) $ohpc_repo_dir/make_repo.sh
\end{lstlisting}
% end_ohpc_run
