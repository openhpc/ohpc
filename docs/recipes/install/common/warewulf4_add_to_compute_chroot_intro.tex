The process used in the previous step is designed to
provide a minimal \baseOS{} configuration. Next, we add additional components
to include resource management client services, NTP support, and
other additional packages to support the default \OHPC{} environment. This
process modifies the base provisioning image and will access the BOS and \OHPC{}
repositories to resolve package install requests. We begin by installing a few
common base packages:

% begin_ohpc_run
% ohpc_comment_header Add OpenHPC base components to compute image \ref{sec:add_components}
\begin{lstlisting}[language=bash,literate={-}{-}1,keywords={},upquote=true,literate={BOSVER}{\baseos{}}1]
# Install compute node base meta-package
[sms](*\#*) (*\containerinstall*)
  (*\install*) ohpc-base-compute
  /bin/false
EOF
\end{lstlisting}
% end_ohpc_run

\noindent Now, we can include additional required components to the compute
instance including resource manager client, NTP, and development environment modules support.

Adding packages can be done by entering the image with \texttt{wwctl container shell},
\texttt{wwctl container shell}, or using a CHROOT.

