Here we configure NTP for compute resources and enable \NFS{}  mounting of a
\$HOME file system and the public \OHPC{} install path (\texttt{/opt/ohpc/pub})
that will be hosted by the {\em master} host in this  example configuration.

\vspace*{0.15cm}
% begin_ohpc_run
% ohpc_comment_header Customize system configuration \ref{sec:master_customization}
\begin{lstlisting}[language=bash,literate={-}{-}1,keywords={},upquote=true]
# Create NFS client mounts of /home and /opt/ohpc/pub on compute hosts
[sms](*\#*) echo "echo "\""${sms_ip}:/home /home nfs nfsvers=3,rsize=1024,wsize=1024,cto 0 0"\"" \
        >> /etc/fstab" >> $postbootscript
[sms](*\#*) echo "echo "\""${sms_ip}:/opt/ohpc/pub /opt/ohpc/pub nfs nfsvers=3 0 0"\"" \
        >> /etc/fstab" >> $postbootscript
[sms](*\#*) echo "systemctl restart nfs" >> $postbootscript

# Mount NFS shares
[sms](*\#*) echo "mount /home" >> $postbootscript
[sms](*\#*) echo "mkdir -p /opt/ohpc/pub" >> $postbootscript
[sms](*\#*) echo "mount /opt/ohpc/pub" >> $postbootscript

# Export /home and OpenHPC public packages from master server
[sms](*\#*) echo "/home *(rw,no_subtree_check,fsid=10,no_root_squash)" >> /etc/exports
[sms](*\#*) echo "/opt/ohpc/pub *(ro,no_subtree_check,fsid=11)" >> /etc/exports
[sms](*\#*) exportfs -a
[sms](*\#*) systemctl restart nfs-server
[sms](*\#*) systemctl enable nfs-server

# Enable NTP time service on computes and identify master host as local NTP server
[sms](*\#*) echo "echo "\""server ${sms_ip}"\"" >> /etc/ntp.conf"  >> $postbootscript
[sms](*\#*) echo "systemctl enable ntpd" >> $postbootscript
[sms](*\#*) echo "systemctl start ntpd" >> $postbootscript

\end{lstlisting}
% end_ohpc_run

