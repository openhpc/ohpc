The following command adds Omni-Path support using base distro-provided drivers
to the chosen {\em master} host.

% begin_ohpc_run
% ohpc_comment_header Optionally add Omni-Path support services on master node \ref{sec:add_opa}
% ohpc_command if [[ ${enable_opa} -eq 1 ]];then
% ohpc_indent 5
\begin{lstlisting}[language=bash,keywords={}]
[sms](*\#*) (*\install*) opa-basic-tools

# Load RDMA services
[sms](*\#*) systemctl start rdma
\end{lstlisting}
% ohpc_indent 0
% ohpc_command fi
% end_ohpc_run

\begin{center}
  \begin{tcolorbox}[]
\OmniPath{} networks require a subnet management service that can typically be
run on either an administrative node, or on the switch itself. The optimal
placement and configuration of the subnet manager is beyond the scope of this
document, but \baseOS{} provides the \texttt{opa-fm} package should you choose
to run it on the {\em master} node.
\end{tcolorbox}
\end{center}

% begin_ohpc_run
% ohpc_validation_newline
% ohpc_validation_comment Optionally enable opensm subnet manager
% ohpc_command if [[ ${enable_opafm} -eq 1 ]];then
% ohpc_indent 5
% ohpc_command (*\install*) opa-fm
% ohpc_command systemctl enable opafm
% ohpc_command systemctl start opafm
% ohpc_indent 0
% ohpc_command fi


%%% With the \OmniPath{} drivers included, you can also enable (optional) IPoIB functionality
%%% which provides a mechanism to send IP packets over the IB network. If you plan
%%% to mount a \Lustre{} file system over \InfiniBand{} (see \S\ref{sec:lustre_client}
%%% for additional details), then having IPoIB enabled is a requirement for the
%%% \Lustre{} client. \OHPC{} provides a template configuration file to aid in setting up
%%% an {\em ib0} interface on the {\em master} host. To use, copy the template
%%% provided and update the \texttt{\$\{sms\_ipoib\}} and
%%% \texttt{\$\{ipoib\_netmask\}} entries to match local desired settings (alter ib0
%%% naming as appropriate if system contains dual-ported or multiple HCAs).
%%% 
%%% % begin_ohpc_run
%%% % ohpc_validation_newline
%%% % ohpc_validation_comment Optionally enable IPoIB interface on SMS
%%% % ohpc_command if [[ ${enable_ipoib} -eq 1 ]];then
%%% % ohpc_indent 5
%%% % ohpc_validation_comment Enable ib0
%%% \begin{lstlisting}[language=bash,literate={-}{-}1,keywords={},upquote=true]
%%% [sms](*\#*) cp /opt/ohpc/pub/examples/network/centos/ifcfg-ib0 /etc/sysconfig/network-scripts
%%% 
%%% # Define local IPoIB address and netmask
%%% [sms](*\#*) perl -pi -e "s/master_ipoib/${sms_ipoib}/" /etc/sysconfig/network-scripts/ifcfg-ib0
%%% [sms](*\#*) perl -pi -e "s/ipoib_netmask/${ipoib_netmask}/" /etc/sysconfig/network-scripts/ifcfg-ib0
%%% 
%%% # Initiate ib0
%%% [sms](*\#*) ifup ib0
%%% \end{lstlisting}
%%% % ohpc_indent 0
%%% % ohpc_command fi
%%% % end_ohpc_run

