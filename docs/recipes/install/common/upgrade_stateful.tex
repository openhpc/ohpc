\subsection{Upgrading OpenHPC Packages}  \label{appendix:upgrade}


As newer \OHPC{} releases are made available, users are encouraged to upgrade
their locally installed packages against the latest repository versions to
obtain access to bug fixes and newer component versions. This can be
accomplished with the underlying package manager as \OHPC{} packaging maintains
versioning state across releases. Also, package builds available from the
\OHPC{} repositories have ``\texttt{-ohpc}'' appended to their names so that
wild cards can be used as a simple way to obtain updates. The following general
procedure highlights a method for upgrading existing installations.

The first step, when using a local mirror as described in
\S\ref{sec:enable_repo}, is downloading a new tarball from \texttt{http://build.openhpc.community/dist}. 
After this, move (or remove) the old repository directory, unpack
the new tarball and run the \texttt{make\_repo.sh} script. In the following,
substitute \texttt{x.x.x} with the desired version number. 

\begin{lstlisting}[language=bash,keywords={},basicstyle=\fontencoding{T1}\fontsize{7.6}{10}\ttfamily,
	literate={VER}{\OHPCVerTree{}}1 {OSREPO}{\OSTree{}}1 {TAG}{\OSTag{}}1 {ARCH}{\arch{}}1 {-}{-}1 
        {VER2}{\OHPCVersion{}}1]
[sms](*\#*) wget http://build.openhpc.community/dist/x.x.x/OpenHPC-x.x.x.OSREPO.ARCH.tar
[sms](*\#*) mv $ohpc_repo_dir "$ohpc_repo_dir"_old
[sms](*\#*) mkdir -p $ohpc_repo_dir
[sms](*\#*) tar xvf OpenHPC-x.x.x.OSREPO.ARCH.tar -C $ohpc_repo_dir
[sms](*\#*) $ohpc_repo_dir/make_repo.sh
\end{lstlisting}

The {\em compute} hosts will pickup new packages automatically as long as the repository
location on the SMS stays the same. If you prefer to use a different location,
setup repositories on the nodes as described in \S\ref{sec:script_add_repos}.

The actual update is performed as follows:
\begin{enumerate*}
\item (Optional) Ensure repo metadata is current (on head and compute nodes.)
  Package managers will naturally do this on their own over time,
  but if you are wanting to access updates immediately after a new release,
  the following can be used to sync to the latest.

\begin{lstlisting}[language=bash,keywords={}]
[sms](*\#*)  (*\clean*)
[sms](*\#*)  psh compute (*\clean*)
\end{lstlisting}

\item Upgrade master (SMS) node

\begin{lstlisting}[language=bash,keywords={}]
[sms](*\#*)  (*\upgrade*) "*-ohpc"
\end{lstlisting}
  
\item Upgrade packages on compute nodes

\begin{lstlisting}[language=bash,keywords={}]
[sms](*\#*) psh compute  (*\upgrade*) "*-ohpc"
\end{lstlisting}
  

\end{enumerate*}

\noindent Note that to update running services such as a resource manager a
service restart (or a full reboot) is required.

\subsubsection{New component variants}

As newer variants of key compiler/MPI stacks are released, \OHPC{} will
periodically add toolchains enabling the latest variant. To stay consistent
throughout the build hierarchy, minimize recompilation requirements for existing
binaries, and allow for multiple variants to coexist, unique delimiters are
used to distinguish RPM package names and module hierarchy.

In the case of a fresh install, \OHPC{} recipes default to installation of the
latest toolchains available in a given release branch. However, if upgrading a
previously installed system, administrators can {\em opt-in} to enable new
variants as they become available. To illustrate this point, we use the \OHPC{}
1.3.1 release as an example which includes a new {``gnu7''} compiler variant
providing GCC 7.x along with runtimes and libraries compiled
with the newer toolchain. Note that prior to the 1.3.1 release, the {``gnu''} variant was
available which provided GCC 5.x versions.
%Component additions are also fair game for \OHPC{} micro releases, and the 1.3.1
%release in particular includes GCC 7.1.0, along with runtimes and libraries
%compiled with that toolchain.
In the case where an admin would like to enable the newer {gnu7} toolchain,
installation of these additions is simplified
with the use of \OHPC{}'s meta-packages (see Table~\ref{table:groups} in Appendix 
\ref{appendix:manifest}).
%and the toolchain package naming 
%convention allows the new compiler to be installed concurrently with older 
%versions.
The following example illustrates adding the complete ``gnu7'' toolchain
environment leveraging convenience meta-packages and also updates the modules environment to make it the default.
%configuring a GCC 7.1.0 default
%development environment and installing all correlated packages.

\begin{lstlisting}[language=bash,keywords={}]
# Update default environment
[sms](*\#*) (*\remove*) lmod-defaults-gnu-mvapich2-ohpc
[sms](*\#*) (*\install*) lmod-defaults-gnu7-mvapich2-ohpc

# Install GCC 7.x-compiled meta-packages with dependencies
[sms](*\#*)  (*\install*) ohpc-gnu7-perf-tools \
                         ohpc-gnu7-serial-libs \
                         ohpc-gnu7-io-libs \
                         ohpc-gnu7-python-libs \
                         ohpc-gnu7-runtimes \
                         ohpc-gnu7-mpich-parallel-libs \
                         ohpc-gnu7-openmpi-parallel-libs \
                         ohpc-gnu7-mvapich2-parallel-libs
\end{lstlisting}

