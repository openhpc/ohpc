% -*- mode: latex; fill-column: 120; -*- 

The first task performed by the post-install script is setting up access to
the \OHPC{} repository on the SMS. \xCAT{} preconfigures base OS repositories, but
additional repositories need to be set up separately. The following commands do
so by using \texttt{yum-config-manager}, a yum repository management tool.

% begin_ohpc_run
% ohpc_comment_header Install yum-utils, disable remote repositories \ref{sec:script_add_repos}
\begin{lstlisting}[language=bash,literate={-}{-}1,keywords={},upquote=true]
# Install yum-utils to have yum-config-manager available
[sms](*\#*) echo "yum --setopt=*.skip_if_unavailable=1 -y install yum-utils" >> $postbootscript
# Disable OS repositories pointing to outside network
[sms](*\#*) echo -e "yum-config-manager --disable CentOS\*" >> $postbootscript

# Add OpenHPC repo mirror hosted on SMS
[sms](*\#*) echo "yum-config-manager --add-repo=http://$sms_ip/$ohpc_repo_dir/OpenHPC.local.repo" \
        >> $postbootscript
# Replace local path with SMS URL
[sms](*\#*) echo "perl -pi -e 's/file:\/\/\@PATH\@/http:\/\/$sms_ip"${ohpc_repo_dir//\//"\/"}"/s' \
        /etc/yum.repos.d/OpenHPC.local.repo" >> $postbootscript
\end{lstlisting}
% end_ohpc_run


The {\em compute} nodes also need access to the EPEL repository, a required
dependency for \OHPC{} packages. A straightforward way to accomplish this is to
host a local mirror on the {\em master} host. However, the repository is very
large ($>$10 GB) containing thousands of packages, while only 3 (\texttt{fping,
qstat, libconfuse}) are required for this guide.  Therefore, here we only
download the required packages and put them into a repository accessible to the
nodes.  Users wanting to mirror the entire EPEL repository instead can take
advantage of the \texttt{reposync} utility and should expect a download time of
aver an hour depending on their network connection and a choice of EPEL mirror.

% begin_ohpc_run
% ohpc_comment_header Configure access to EPEL packages \ref{sec:script_add_repos}
\begin{lstlisting}[language=bash,literate={-}{-}1,keywords={},upquote=true]
# Create directory holding EPEL packages
[sms](*\#*) mkdir -p $epel_repo_dir
# Download required EPEL packages
[sms](*\#*) yumdownloader --destdir $epel_repo_dir fping qstat libconfuse
# Create a repository from the downloaded packages
[sms](*\#*) createrepo $epel_repo_dir

# Setup access to the repo from the compute hosts
[sms](*\#*) echo "yum-config-manager --add-repo=http://$sms_ip/$epel_repo_dir"  >> $postbootscript
[sms](*\#*) echo "echo gpgcheck=0 >> /etc/yum.repos.d/$sms_ip"${epel_repo_dir//\//"_"}".repo" >> $postbootscript
\end{lstlisting}
% end_ohpc_run

Next, we work around a problem with documentation files on {\em compute} hosts
that causes installation failure for certain packages.  This is because, by
default, documentation is installed to \texttt{/opt/ohpc/pub/doc}, but that
directory is a part of a read-only NFS share setup in
\S~\ref{sec:master_customization}. As a result, packages attempting to add to
this directory fail to install.  To avoid this, we setup \texttt{rpm} on the
{\em compute} nodes to exclude documentation files during installation.


% begin_ohpc_run
% ohpc_comment_header Exclude docs when installing compute nodes \ref{sec:script_add_repos}
\begin{lstlisting}[language=bash,literate={-}{-}1,keywords={},upquote=true]
[sms](*\#*) echo -e "echo -e "\""%_excludedocs 1"\"" >> ~/.rpmmacros" >> $postbootscript
\end{lstlisting}
% end_ohpc_run
