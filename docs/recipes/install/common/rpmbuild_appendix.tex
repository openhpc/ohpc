\subsection{RPM Rebuild}  \label{appendix:rpmbuild}

Users of OpenHPC may find it necessary to rebuild one of the packages to satisfy
local requirements. A simple way to accomplish this is to install the provided
source RPM, modify the specfile, and rebuild the binary RPM. A brief example
follows.

\begin{enumerate}
\item Install the \texttt{rpm-build} package

\begin{lstlisting}[language=bash,keywords={}]
[sms](*\#*) (*\install*) rpm-build
\end{lstlisting}

\item Download and install the source RPM 
\begin{lstlisting}[language=bash,keywords={},basicstyle=\fontencoding{T1}\footnotesize\ttfamily,
    literate={VER}{\OHPCVersion{}}1 {OSREPO}{\OSRepo{}}1 {-}{-}1]
[sms](*\#*) wget http://build.openhpc.community/OpenHPC:/VER:/Factory/OSREPO/src/openmpi-gnu-ohpc-1.10.1-8.1.src.rpm
[sms](*\#*) rpm -i openmpi-gnu-ohpc-1.10.1-8.1.src.rpm
\end{lstlisting}

\item Modify spec file to reflect any desired changes

\begin{lstlisting}[language=bash,keywords={}]
[sms](*\#*) cd ~/rpmbuild/SPECS
[sms](*\#*) perl -pi -e "s/BASEFLAGS=\"--prefix/BASEFLAGS=\"--enable-mpi-thread-multiple --prefix/" openmpi.spec
\end{lstlisting}

\item Copy necessary OpenHPC macro files (\texttt{OHPC\_macros}, 
\texttt{OHPC\_setup\_compiler}, etc.) to source directory

\begin{lstlisting}
[sms](*\#*) cd ~/rpmbuild/SOURCES
[sms](*\#*) wget https://raw.githubusercontent.com/openhpc/ohpc/obs/OpenHPC_1.0.1_Factory/components/OHPC_macros
[sms](*\#*) wget https://raw.githubusercontent.com/openhpc/ohpc/obs/OpenHPC_1.0.1_Factory/components/OHPC_setup_compiler
\end{lstlisting}

\item Provided the package's build dependencies are satisfied, rebuild the
package

\begin{lstlisting}
[sms](*\#*) rpmbuild -ba ~/rpmbuild/SPECS/openmpi.spec
\end{lstlisting}

\item Install the new package

\begin{lstlisting}[language=bash,keywords={}]
[sms](*\#*) (*\install*) rpmbuild/RPMS/openmpi-gnu-ohpc-1.10.1-8.1.x86_64
\end{lstlisting}
\end{enumerate}
