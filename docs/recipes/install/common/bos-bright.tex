In an external setting, installing the desired BOS and Bright on a
{\em master} SMS host typically involves booting from a DVD ISO image on a new
server. With this approach, insert the Bright/\baseOS{} DVD, power cycle the host, and
follow the Bright provided directions to install the BOS on your chosen {\em
master} host.  Alternatively, if choosing to use a pre-installed server, please
verify that it is provisioned with the required \baseOS{} distribution. If Bright
is not already installed on the {\em master} host, please follow the steps in the Bright
installation manual on performing an addon installation \\

In depth instructions of installing Bright/\baseOS{} are covered in the Bright 
\href{https://support.brightcomputing.com/manuals/9.0/installation-manual.pdf}{\color{blue}{Installation Manual}}.

If the SLURM workload manager is not selected during the Bright {\em master} installation
and you intend to install SLURM afterwards, please refer to the
\href{https://support.brightcomputing.com/manuals/9.0/admin-manual.pdf#section.7.3}{\color{blue}{Installation of Workload Managers}}
section of the Bright administration manual.

Prior to beginning the installation process of \OHPC{} components, several additional
considerations are noted here for the SMS host configuration. First,
the installation recipe herein assumes that
the SMS host name is resolvable locally. Since Bright manages this part of the
\path{/etc/hosts} file automatically, it should not be required to make manual modifications
to the \path{/etc/hosts} file.

While it is theoretically possible to enable SELinux on a cluster provisioned
with \provisioner{},
doing so is beyond the scope of this document. Even the use of
permissive mode can be problematic and we therefore recommend keeping SELinux disabled on the {\em
master} SMS host. If SELinux components are installed locally,
the \texttt{selinuxenabled} command can be used to determine if SELinux is
currently enabled. If enabled, consult the distro documentation for information
on how to disable. \\

Finally, the Bright node-installer provisioning rely on DHCP, TFTP, HTTP and rsync
(or optionally rsync over SSH) network protocols. For the internal network, Bright
is automatically configured to allow these protocols to the {\em master}, so out of
the box, no additional manual configuration should be required.
