With the provisioning services enabled, the next step is to define and
customize a system image that can subsequently be used to provision one or more
{\em compute} nodes. The following subsections highlight this process.

\subsubsection{Build initial BOS image} \label{sec:assemble_bos}
\Warewulf{} 4 supports using container images as the base file system for 
provisioning, and it can import these images directly from an OCI registry like
Docker Hub. Container images must be created especially for use with \Warewulf{}
since they need to include things like a kernel and an init system. In this 
example we will import our base image from a set maintained by the \Warewulf{}
community on the GitHub container registry.

The \texttt{wwctl container exec} command runs the commands below it, these commands 
also be run interactively one a time with the command \texttt{wwctl container 
shell \baseos{}}. The final command \texttt{/bin/false} prevents the image from
rebuilding, this is also true for any final command that fails.

% begin_ohpc_run
% ohpc_comment_header Create compute image for Warewulf \ref{sec:assemble_bos}
\begin{lstlisting}[language=bash,literate={-}{-}1,keywords={},upquote=true,keepspaces,literate={BOSVER}{\baseos{}}1{BOSTAG}{\baseostag{}}1]
# Import the base image from ghcr
[sms](*\#*) wwctl container import docker://ghcr.io/warewulf/warewulf-rockylinux:BOSTAG BOSVER --syncuser

# Enable OpenHPC inside image and update container
[sms](*\#*) (*\containerinstall*)
  (*\install*) http://repos.openhpc.community/OpenHPC/3/EL_9/x86_64/ohpc-release-3-1.el9.x86_64.rpm
  (*\pkgmgr*) -y update
  /bin/false
EOF

# Define chroot location 
[sms](*\#*) export CHROOT=/srv/warewulf/chroots/BOSVER/rootfs
\end{lstlisting}
% end_ohpc_run
