\OHPC{} provides pre-packaged builds for a number of popular open-source
tools and libraries used by HPC applications and developers. For
example, \OHPC{} provides builds for \FFTW{} and \hdffive{} (including serial and parallel
I/O support), and the \GNU{} Scientific Library (GSL). Again, multiple builds of
each package are available in the \OHPC{} repository to support multiple compiler
and MPI family combinations where appropriate. Note, however, that not all
combinatorial permutations may be available for components where there are known
license incompatibilities. The general naming convention
for builds provided by \OHPC{} is to append the compiler and MPI family name that
the library was built against directly into the package name. For example,
libraries that do not require MPI as part of the build process adopt the
following RPM name: \\

\noindent
\texttt{package-<compiler\_family>-ohpc-<package\_version>-<release>.rpm} \\

\noindent Packages that do require MPI as part of the build expand upon this convention to
additionally include the MPI family name as follows: \\

\noindent
\texttt{package-<compiler\_family>-<mpi\_family>-ohpc-<package\_version>-<release>.rpm} \\

To illustrate this further, the command below queries the locally configured
repositories to identify all of the available PETSc packages that were built
with the \GNU{} toolchain. The resulting output that is included shows that
pre-built versions are available for each of the supported MPI families
presented in \S\ref{sec:mpi}.
