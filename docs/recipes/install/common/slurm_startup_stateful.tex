In section \S\ref{sec:basic_install}, the \SLURM{} resource manager was
installed and configured on the  {\em master} host. \SLURM{} clients were also
installed, but have not been configured yet. To do so, \SLURM{} and
\texttt{munge} configuration files need to be copied to the nodes. This can be
accomplished as follows:

% begin_ohpc_run
% ohpc_comment_header Resource Manager Startup \ref{sec:rms_startup}
\begin{lstlisting}[language=bash,keywords={}]
[sms](*\#*) xdcp compute /etc/slurm/slurm.conf /etc/slurm/slurm.conf
[sms](*\#*) xdcp compute /etc/munge/munge.key /etc/munge/munge.key
\end{lstlisting}
% end_ohpc_run


With \SLURM{} configured, we can now startup the
resource manager services in preparation for running user jobs. Generally, this
is a two-step process that requires starting up the controller daemons on the {\em
 master} host and the client daemons on each of the {\em compute} hosts.
%Since the {\em compute} hosts were booted into an image that included the SLURM client
%components, the daemons should already be running on the {\em compute}
%hosts. 
Note that \SLURM{} leverages the use of the {\em munge} library to provide
authentication services and this daemon also needs to be running on all hosts
within the resource management pool. 
%The munge daemons should already
%be running on the {\em compute} subsystem at this point, 
The following commands can be used to startup the necessary services to support
resource management under \SLURM{}.

%\iftoggle{isCentOS}{\clearpage}

% begin_ohpc_run
\begin{lstlisting}[language=bash,keywords={}]
# Start munge and slurm controller on master host
[sms](*\#*) systemctl enable munge
[sms](*\#*) systemctl enable slurmctld
[sms](*\#*) systemctl start munge
[sms](*\#*) systemctl start slurmctld

# Start slurm clients on compute hosts
[sms](*\#*) psh compute systemctl enable munge
[sms](*\#*) psh compute systemctl enable slurmd
[sms](*\#*) psh compute systemctl start munge
[sms](*\#*) psh compute systemctl start slurmd
\end{lstlisting}
% end_ohpc_run

%%% In the default configuration, the {\em compute} hosts will be initialized in an
%%% {\em unknown} state. To place the hosts into production such that they are
%%% eligible to schedule user jobs, issue the following:

%%% % begin_ohpc_run
%%% \begin{lstlisting}[language=bash]
%%% [sms](*\#*) scontrol update partition=normal state=idle
%%% \end{lstlisting}
%%% % end_ohpc_run

After this, check status of the nodes within \SLURM{} by using the
\texttt{sinfo} command. All compute nodes should be in an {\em idle} state
(without asterisk). If the state is reported as {\em unknown}, the following
might help:

%%% % begin_ohpc_run
\begin{lstlisting}[language=bash]
[sms](*\#*) scontrol update partition=normal state=idle
\end{lstlisting}
%%% % end_ohpc_run

In case of additional \SLURM{} issues, ensure that the configuration file fits
your hardware and that it is identical across the nodes. Also, verify
that \SLURM{} user id is the same on the SMS and {\em compute} nodes. You may
also consult
\href{https://slurm.schedmd.com/troubleshoot.html}{\color{blue}\SLURM{}
Troubleshooting Guide}.
