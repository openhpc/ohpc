In addition to running the VNFS image from memory, Warewulf can also partition
and format persistent storage on a node and maintain the VNFS state across
reboots. This requires installing additional packages in the VNFS image to
support a boot loader (GRUB), and it also requires some modifictation to the
node definition in the warewulf data store. 

% begin_fsp_run
% fsp_validation_comment Add stateful provisioning support
\begin{lstlisting}[language=bash,literate={-}{-}1,keywords={},upquote=true]
# Add GRUB2 Bootloader
[sms]# (*\chrootinstall*) grub2

# Re-assemble VNFS image
[sms]# wwvnfs  -y --chroot $CHROOT
\end{lstlisting}
% end_fsp_run 

Stateful nodes require addition parameters in the Warewulf configuration. In
particular, we instruct Warewulf where to install the GRUB bootloader, which
disk to partition, which partition to format, and what the filesystem layout
will look like. These changes are necessary for each compute node we wish to
provision statefully.

% begin_fsp_run
% fsp_validation_comment Add stateful node object parameters
\begin{lstlisting}[language=bash,literate={-}{-}1,keywords={},upquote=true]
# Update node object parameters
[sms]# export filesystems="mountpoint=/boot:dev=sda1:type=ext3:size=500," \
              "dev=sda2:type=swap:size=32768," \
              "mountpoint=/:dev=sda3:type=ext3:size=fill"
[sms]# for ((i=0; i<$num_computes; i++)) ; do 
              wwsh -y object modify -s bootloader=sda ${c_name[$i]};
              wwsh -y object modify -s diskpartition=sda ${c_name[$i]};
              wwsh -y object modify -s diskformat=sda1,sda2,sda3 ${c_name[$i]};
              wwsh -y object modify -s filesystems="$filesystems" ${c_name[$i]};
           done
\end{lstlisting}
% end_fsp_run 

Upon the node's next reboot, Warewulf will partition and format the disk. It
will also run grub2-mkconfig and grub2-install to place grub in the proscribed
boot sector, and it will install the VNFS locally. In order to ensure this 
process happens only once per node we must instruct warewulf to next boot from 
the local storage.

% begin_fsp_run
% fsp_validation_comment Add bootlocal node object parameters
\begin{lstlisting}[language=bash,literate={-}{-}1,keywords={},upquote=true]
# Update node object parameters
[sms]# for ((i=0; i<$num_computes; i++)) ; do 
              wwsh -y object modify -s bootlocal=EXIT ${c_name[$i]};
           done
\end{lstlisting}
% end_fsp_run 

Deleting the bootlocal object parameter will cause Warewulf once againg to
re-partion and format the local storage.
