\noindent Similarly, to distribute the global Slurm configuration file and the
cryptographic key that is required by the {\em munge} authentication library to
every host in the resource management pool, issue the following:

% begin_ohpc_run
\begin{lstlisting}[language=bash,literate={-}{-}1,keywords={},upquote=true]
[sms](*\#*) echo "/etc/slurm/slurm.conf -> /etc/slurm/slurm.conf" >> $synclist
[sms](*\#*) echo "/etc/munge/munge.key  -> /etc/munge/munge.key"  >> $synclist
\end{lstlisting}
% \end_ohpc_run

\begin{center}
\begin{tcolorbox}[]
\small
The ``\texttt{updatenode compute -F}'' command can be used to distribute changes made to any
defined synchronization files on the SMS host. Users wishing to automate this process may
want to consider adding a crontab entry to perform this action at defined intervals.

\iftoggleverb{isstateful}
Note that under certain circumstances full user and group synchronization can 
cause problems. For example, a different order of installation between {\em
master} and {\em compute} nodes can cause system created groups and users to have
different ID numbers. In such a case, it is better to only synchronize
non-system users and groups.  
\fi
\end{tcolorbox}
\end{center}
