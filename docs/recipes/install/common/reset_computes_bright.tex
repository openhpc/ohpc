Prior to booting the {\em compute} hosts, we configure them to use PXE as their
next boot mode. After the initial PXE, since Bright installs iPXE on the node 
disk drive, ensuing boots will continue to attempt to PXE boot from the network.

During the initial boot of the nodes, when the Bright node installer runs,
you will be asked to assign each of the compute nodes an identity. This only
needs to occur on the initial boot as Bright will record the node MAC address
in it's database so that on subsequent boots, Bright will know which node is
being provisioned. For more information regarding the node boot process,
please refer to the
\href{https://support.brightcomputing.com/manuals/9.0/admin-manual.pdf#section.5.4}{\color{blue}{Node-Installer}}
section of the Bright admin manual.

If you wish to define additional compute nodes than what was defined during
the {\em master} installation, the steps to add additional compute nodes to Bright
can be found in the
\href{https://support.brightcomputing.com/manuals/9.0/admin-manual.pdf#section.5.7}{\color{blue}{Adding New Nodes}}
section of the Bright admin manual.

After the initial boot, providing the nodes BMC interfaces have been configured 
per section 3.7 of the Bright administration manual, the {\em master} server 
should be able to boot the compute nodes. This is done by using the \texttt{power} 
Bright command leverging IPMI protocol. The following power cycles a Bright
node.

% begin_ohpc_run
% ohpc_comment_header Power cycle a single compute node
\begin{lstlisting}[language=bash,keywords={},upquote=true]
[sms](*\#*) cmsh
% device use <node_name>
% power reset
\end{lstlisting} 
% end_ohpc_run

It is also possible to power cycle multiple nodes at once, for example,
to power cycle all nodes in the default category, the following can be performed.

% begin_ohpc_run
% ohpc_comment_header Power cycle multiple compute nodes in a category
\begin{lstlisting}[language=bash,keywords={},upquote=true]
[sms](*\#*) cmsh
% device
% power reset -c default
\end{lstlisting} 
% end_ohpc_run
