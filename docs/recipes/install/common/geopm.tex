Global Extensible Open Power Manager (\GEOPM{}) is a framework for exploring
power and energy optimizations targeting high performance computing.  The
library can be extended to support new control algorithms and new
hardware-specific power management features.  The \GEOPM{} package provides
built-in features ranging from static management of power policy for each
individual compute node, to dynamic coordination of power policy and
performance across all of the compute nodes hosting one MPI job on a portion of
a distributed computing system.  The dynamic coordination is implemented as a
hierarchical control system for scalable communication and decentralized
control.

\GEOPM{} will be installed with the Performance Tools meta package
discussed in Section~\ref{sec:install_perf_tools}.  Once installed, arguments
must be added to the kernel command line for the compute nodes:

% begin_ohpc_run
% ohpc_validation_newline
% ohpc_command if [[ ${enable_geopm} -eq 1 ]];then
% ohpc_indent 5
\begin{lstlisting}[language=bash,keywords={},upquote=true]
# Disable Intel pstate driver because it interferes with GEOPM's operation.
[sms](*\#*) export kargs="${kargs} intel_pstate=disable"
\end{lstlisting}
% ohpc_indent 0
% ohpc_command fi
% end_ohpc_run

\noindent The \GEOPM{} package uses \OHPC{}'s msr-safe kernel module
to enable users to directly read and write whitelisted Model Specific
Registers (MSRs).  For documentation on how to use \GEOPM{} please see
the \GEOPM{} man pages which are all linked from the geopm(7) overview
man page available in html here:
\href{http://geopm.github.io/man/geopm.7.html}
{\color{blue}{http://geopm.github.io/man/geopm.7.html}}.
Please see the \GEOPM{} tutorials and for working examples using the
\GEOPM{} runtime here: \href{https://github.com/geopm/geopm/tree/dev/tutorial}
{\color{blue}{https://github.com/geopm/geopm/tree/dev/tutorial}}.
