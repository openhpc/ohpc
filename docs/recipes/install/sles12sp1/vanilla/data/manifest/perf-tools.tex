\newcolumntype{C}[1]{>{\centering}p{#1}} 
\newcolumntype{L}[1]{>{\raggedleft}p{#1}} 
\small
\begin{tabularx}{\textwidth}{L{\firstColWidth{}}|C{\secondColWidth{}}|X}
\toprule
{\bf RPM Package Name} & {\bf Version} & {\bf Info/URL}  \\ 
\midrule

% <-- begin entry for imb
imb-gnu-mvapich2-ohpc & 
\multirow{2}{*}{4.1} & 
\multirow{2}{\linewidth}{Intel MPI Benchmarks (IMB). \newline {\color{logoblue} \url{https://software.intel.com/en-us/articles/intel-mpi-benchmarks}}} \\imb-gnu-openmpi-ohpc & 
& \\ 
\hline
% <-- end entry for imb

% <-- begin entry for mpiP
mpiP-gnu-mvapich2-ohpc & 
\multirow{2}{*}{3.4.1} & 
\multirow{2}{\linewidth}{mpiP: a lightweight profiling library for MPI applications. \newline {\color{logoblue} \url{http://mpip.sourceforge.net}}} \\ 
mpiP-gnu-openmpi-ohpc & 
& \\ 
\hline
% <-- end entry for mpiP

% <-- begin entry for papi-ohpc
\multirow{2}{*}{papi-ohpc} & 
\multirow{2}{*}{5.4.1} & 
Performance Application Programming Interface. \newline { \color{logoblue} \url{http://icl.cs.utk.edu/papi}} 
\\
\hline 
% <-- end entry for papi-ohpc

% <-- begin entry for pdtoolkit
\multirow{2}{*}{pdtoolkit-gnu-ohpc} & 
\multirow{2}{*}{3.21} & 
PDT is a framework for analyzing source code. \newline {\color{logoblue} \url{http://www.cs.uoregon.edu/Research/pdt}} \\ 
\hline
% <-- end entry for pdtoolkit

% <-- begin entry for tau
tau-gnu-mvapich2-ohpc & 
\multirow{2}{*}{2.25} & 
\multirow{2}{\linewidth}{Tuning and Analysis Utilities Profiling Package. \newline {\color{logoblue} \url{http://www.cs.uoregon.edu/research/tau/home.php}}} \\ 
tau-gnu-openmpi-ohpc & 
& \\ 
\hline
% <-- end entry for tau

\bottomrule
\end{tabularx}
