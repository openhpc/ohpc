Launched initially in November 2015, and formalized as a collaborative Linux
Foundation~\cite{LinuxFoundation_url} project in June 2016, OpenHPC is a
community driven project currently comprised of over 25 member organizations
with representation from academia, research labs, and industry. To date, the
OpenHPC software stack aggregates over 60 components ranging from tools for
bare-metal provisioning, administration, and resource management to end-user
development libraries that span a range of scientific/numerical uses. OpenHPC
adopts a familiar repository delivery model with HPC-centric packaging in mind,
and provides customizable recipes for installing and configuring reference
designs of compute clusters.  OpenHPC is intended both to make available
current best practices and provide a framework for delivery of future
innovation in cluster computing system software.
